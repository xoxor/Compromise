\RequirePackage[l2tabu, orthodox]{nag}
\documentclass[version=3.21, pagesize, notitlepage, twoside=off, bibliography=totoc, DIV=calc, fontsize=12pt, a4paper]{scrartcl}
%Permits to copy eg x ⪰ y ⇔ v(x) ≥ v(y) from PDF to unicode data, and to search. From pdfTeX users manual. See https://tex.stackexchange.com/posts/comments/1203887.
	\input glyphtounicode
	\pdfgentounicode=1
%Latin Modern has more glyphs than Computer Modern, such as diacritical characters. fntguide commands to load the font before fontenc, to prevent default loading of cmr.
	\usepackage{lmodern}
%Encode resulting accented characters correctly in resulting PDF, permits copy from PDF.
	\usepackage[T1]{fontenc}
%UTF8 seems to be the default in recent TeX installations, but not all, see https://tex.stackexchange.com/a/370280.
	\usepackage[utf8]{inputenc}
%Provides \newunicodechar for easy definition of supplementary UTF8 characters such as → or ≤ for use in source code.
	\usepackage{newunicodechar}
%Text Companion fonts, much used together with CM-like fonts. Provides \texteuro and commands for text mode characters such as \textminus, \textrightarrow, \textlbrackdbl.
	\usepackage{textcomp}
%St Mary’s Road symbol font, used for ⟦ = \llbracket.
	\usepackage{stmaryrd}
\usepackage{centernot}
%Solves bug in lmodern, https://tex.stackexchange.com/a/261188; probably useful only for unusually big font sizes; and probably better to use exscale instead. Note that the authors of exscale write against this trick.
	%\DeclareFontShape{OMX}{cmex}{m}{n}{
		%<-7.5> cmex7
		%<7.5-8.5> cmex8
		%<8.5-9.5> cmex9
		%<9.5-> cmex10
	%}{}
	%\SetSymbolFont{largesymbols}{normal}{OMX}{cmex}{m}{n}
%More symbols (such as \sum) available in bold version, see https://github.com/latex3/latex2e/issues/71.
	\DeclareFontShape{OMX}{cmex}{bx}{n}{%
	   <->sfixed*cmexb10%
	   }{}
	\SetSymbolFont{largesymbols}{bold}{OMX}{cmex}{bx}{n}
%For small caps also in italics, see https://tex.stackexchange.com/questions/32942/italic-shape-needed-in-small-caps-fonts, https://tex.stackexchange.com/questions/284338/italic-small-caps-not-working.
	\usepackage{slantsc}
	\AtBeginDocument{%
		%“Since nearly no font family will contain real italic small caps variants, the best approach is to substitute them by slanted variants.” -- slantsc doc
		%\DeclareFontShape{T1}{lmr}{m}{scit}{<->ssub*lmr/m/scsl}{}%
		%There’s no bold small caps in Latin Modern, we switch to Computer Modern for bold small caps, see https://tex.stackexchange.com/a/22241
		%\DeclareFontShape{T1}{lmr}{bx}{sc}{<->ssub*cmr/bx/sc}{}%
		%\DeclareFontShape{T1}{lmr}{bx}{scit}{<->ssub*cmr/bx/scsl}{}%
	}
%Warn about missing characters.
	\tracinglostchars=2
%Nicer tables: provides \toprule, \midrule, \bottomrule.
	%\usepackage{booktabs}
%For new column type X which stretches; can be used together with booktabs, see https://tex.stackexchange.com/a/97137. “tabularx modifies the widths of the columns, whereas tabular* modifies the widths of the inter-column spaces.” Loads array.
	%\usepackage{tabularx}
%math-mode version of "l" column type. Requires \usepackage{array}.
	%\usepackage{array}
	%\newcolumntype{L}{>{$}l<{$}}
%Provides \xpretocmd and loads etoolbox which provides \apptocmd, \patchcmd, \newtoggle… Also loads xparse, which provides \NewDocumentCommand and similar commands intended as replacement of \newcommand in LaTeX3 for defining commands (see https://tex.stackexchange.com/q/98152 and https://github.com/latex3/latex2e/issues/89).
	\usepackage{xpatch}
%ntheorem doc says: “empheq provides an enhanced vertical placement of the endmarks”; must be loaded before ntheorem. Loads the mathtools package, which loads and fixes some bugs in amsmath and provides \DeclarePairedDelimiter. amsmath is considered a basic, mandatory package nowadays (Grätzer, More Math Into LaTeX).
	\usepackage[ntheorem]{empheq}
%Package frenchb asks to load natbib before babel-french. Package hyperref asks to load natbib before hyperref.
	\usepackage{natbib}

\newtoggle{LCpres}
	\newtoggle{LCart}
	\newtoggle{LCposter}
	\makeatletter
	\@ifclassloaded{beamer}{
		\toggletrue{LCpres}
		\togglefalse{LCart}
		\togglefalse{LCposter}
		\wlog{Presentation mode}
	}{
		\@ifclassloaded{tikzposter}{
			\toggletrue{LCposter}
			\togglefalse{LCpres}
			\togglefalse{LCart}
			\wlog{Poster mode}
		}{
			\toggletrue{LCart}
			\togglefalse{LCpres}
			\togglefalse{LCposter}
			\wlog{Article mode}
		}
	}
	\makeatother%

%Language options ([french, english]) should be on the document level (last is main); except with tikzposter: put [french, english] options next to \usepackage{babel} to avoid warning. beamer uses the \translate command for the appendix: omitting babel results in a warning, see https://github.com/josephwright/beamer/issues/449. Babel also seems required for \refname.
	\iftoggle{LCpres}{
		\usepackage{babel}
	}{
	}
	%\frenchbsetup{AutoSpacePunctuation=false}
%listings (1.7) does not allow multi-byte encodings. listingsutf8 works around this only for characters that can be represented in a known one-byte encoding and only for \lstinputlisting. Other workarounds: use literate mechanism; or escape to LaTeX (but breaks alignment).
	%\usepackage{listings}
	%\lstset{tabsize=2, basicstyle=\ttfamily, escapechar=§, literate={é}{{\'e}}1}
%I favor acro over acronym because the former is more recently updated (2018 VS 2015 at time of writing); has a longer user manual (about 40 pages VS 6 pages if not counting the example and implementation parts); has a command for capitalization; and acronym suffers a nasty bug when ac used in section, see https://tex.stackexchange.com/q/103483 (though this might be the fault of the silence package and might be solved in more recent versions, I do not know) and from a bug when used with cleveref, see https://tex.stackexchange.com/q/71364. However, loading it makes compilation time (one pass on this template) go from 0.6 to 1.4 seconds, see https://bitbucket.org/cgnieder/acro/issues/115. Option short-format not usable in the package options as it is fragile, see https://tex.stackexchange.com/q/466882.
	\usepackage[single]{acro}
	%\acsetup{short-format = {\MakeUppercase}}
	\DeclareAcronym{AMCD}{short=amcd, long={Aide Multicritère à la Décision}}
\DeclareAcronym{AR}{short=ar, long={Argumentative Recommender}}
\DeclareAcronym{DA}{short=da, long={Decision Analysis}}
\DeclareAcronym{DJ}{short=dj, long={Deliberated Judgment}}
\DeclareAcronym{DM}{short=dm, long={Decision Maker}}
\DeclareAcronym{DP}{short=dp, long={Deliberated Preference}}
\DeclareAcronym{MAVT}{short=mavt, long={Multiple Attribute Value Theory}}
\DeclareAcronym{MCDA}{short=mcda, long={Multicriteria Decision Aid}}
\DeclareAcronym{MIP}{short=mip, long={Mixed Integer Program}}


\iftoggle{LCpres}{
	%I favor fmtcount over nth because it is loaded by datetime anyway; and fmtcount warns about possible conflicts when loaded after nth.
	\usepackage{fmtcount}
	%For nice input of date of presentation. Must be loaded after the babel package. Has possible problems with srcletter: https://golatex.de/verwendung-von-babel-und-datetime-in-scrlttr2-schlaegt-fehlt-t14779.html.
	\usepackage[nodayofweek]{datetime}
}{
}
%For presentations, Beamer implicitely uses the pdfusetitle option. ntheorem doc says to load hyperref “before the first use of \newtheorem”. autonum doc mandates option hypertexnames=false. I want to highlight links only if necessary for the reader to recognize it as a link, to reduce distraction. In presentations, this is already taken care of by beamer (https://tex.stackexchange.com/a/262014). If using colorlinks=true in a presentation, see https://tex.stackexchange.com/q/203056. Crashes the first compilation with tikzposter, just compile again and the problem disappears, see https://tex.stackexchange.com/q/254257.
\makeatletter
\iftoggle{LCpres}{
	\usepackage{hyperref}
}{
	\usepackage[hypertexnames=false, pdfusetitle, linkbordercolor={1 1 1}, citebordercolor={1 1 1}, urlbordercolor={1 1 1}]{hyperref}
	%https://tex.stackexchange.com/a/466235
	\pdfstringdefDisableCommands{%
		\let\thanks\@gobble
	}
}
\makeatother
%urlbordercolor is used both for \url and \doi, which I think shouldn’t be colored, and for \href, thus might want to color manually when required. Requires xcolor.
	\NewDocumentCommand{\hrefblue}{mm}{\textcolor{blue}{\href{#1}{#2}}}
%hyperref doc says: “Package bookmark replaces hyperref’s bookmark organization by a new algorithm (...) Therefore I recommend using this package”.
	\usepackage{bookmark}
%Need to invoke hyperref explicitly to link to line numbers: \hyperlink{lintarget:mylinelabel}{\ref*{lin:mylinelabel}}, with \ref* to disable automatic link. Also see https://tex.stackexchange.com/q/428656 for referencing lines from another document.
	%\usepackage{lineno}
	%\NewDocumentCommand{\llabel}{m}{\hypertarget{lintarget:#1}{}\linelabel{lin:#1}}
	%\setlength\linenumbersep{9mm}
%For complex authors blocks. Seems like authblk wants to be later than hyperref, but sooner than silence. See https://tex.stackexchange.com/q/475513 for the patch to hyperref pdfauthor.
	\ExplSyntaxOn
	\seq_new:N \g_oc_hrauthor_seq
	\NewDocumentCommand{\addhrauthor}{m}{
		\seq_gput_right:Nn \g_oc_hrauthor_seq { #1 }
	}
	%Should be \NewExpandableDocumentCommand, but this is not yet provided by my version of xparse
	\DeclareExpandableDocumentCommand{\hrauthor}{}{
		\seq_use:Nn \g_oc_hrauthor_seq {,~}
	}
	\ExplSyntaxOff
	{
		\catcode`#=11\relax
		\gdef\fixauthor{\xpretocmd{\author}{\addhrauthor{#2}}{}{}}%
	}
	\iftoggle{LCart}{
		\usepackage{authblk}
		\renewcommand\Affilfont{\small}
		\fixauthor
		\AtBeginDocument{
		    \hypersetup{pdfauthor={\hrauthor}}
		}
	}{
	}
%I do not use floatrow, because it requires an ugly hack for proper functioning with KOMA script (see scrhack doc). Instead, the following command centers all floats (using \centering, as the center environment adds space, http://texblog.net/latex-archive/layout/center-centering/), and I manually place my table captions above and figure captions below their contents (https://tex.stackexchange.com/a/3253).
	\makeatletter
	\g@addto@macro\@floatboxreset\centering
	\makeatother
%Permits to customize enumeration display and references
	%\nottoggle{LCpres}{
		\usepackage{enumitem} %follow list environments by a string to customize enumeration, example: \begin{description}[itemindent=8em, labelwidth=!] or \begin{enumerate}[label=({\roman*}), ref={\roman*}].
	%}{
	%}
%Provides \Cen­ter­ing, \RaggedLeft, and \RaggedRight and en­vi­ron­ments Cen­ter, FlushLeft, and FlushRight, which al­low hy­phen­ation. With tikzposter, seems to cause 1=1 to be printed in the middle of the poster.
	%\usepackage{ragged2e}
%To typeset units by closely following the “official” rules.
	%\usepackage[strict]{siunitx}
%Turns the doi provided by some bibliography styles into URLs. However, uses old-style dx.doi url (see 3.8 DOI system Proxy Server technical details, “Users may resolve DOI names that are structured to use the DOI system Proxy Server (https://doi.org (current, preferred) or earlier syntax http://dx.doi.org).”, https://www.doi.org/doi_handbook/3_Resolution.html). The patch solves this.
	\usepackage{doi}
	\makeatletter
	\patchcmd{\@doi}{http://dx.doi.org}{https://doi.org}{}{}
	\makeatother
%Makes sure upper case greek letters are italic as well.
	\usepackage{fixmath}
%Provides \mathbb; obsoletes latexsym (see http://tug.ctan.org/macros/latex/base/latexsym.dtx). Relatedly, \usepackage{eucal} to change the mathcal font and \usepackage[mathscr]{eucal} (apparently equivalent to \usepackage[mathscr]{euscript}) to supplement \mathcal with \mathscr. This last option is not very useful as both fonts are similar, and the intent of the authors of eucal was to provide a replacement to mathcal (see doc euscript). Also provides \mathfrak for supplementary letters.
	\usepackage{amsfonts}
%Provides a beautiful (IMHO) \mathscr and really different than \mathcal, for supplementary uppercase letters. But there is no bold version. Alternative: mathrsfs (more slanted), but when used with tikzposter, it warns about size substitution, see https://tex.stackexchange.com/q/495167.
	\usepackage[scr]{rsfso}
%Multiple means to produce bold math: \mathbf, \boldmath (defined to be \mathversion{bold}, see fntguide), \pmb, \boldsymbol (all legacy, from LaTeX base and AMS), \bm (the most recommended one), \mathbold from package fixmath (I don’t see its advantage over \boldsymbol).
%“The \boldsymbol command is obtained preferably by using the bm package, which provides a newer, more powerful version than the one provided by the amsmath package. Generally speaking, it is ill-advised to apply \boldsymbol to more than one symbol at a time.” — AMS Short math guide. “If no bold font appears to be available for a particular symbol, \bm will use ‘poor man’s bold’” — bm. It is “best to load the package after any packages that define new symbol fonts” – bm. bm defines \boldsymbol as synonym to \bm. \boldmath accesses the correct font if it exists; it is used by \bm when appropriate. See https://tex.stackexchange.com/a/10643 and https://github.com/latex3/latex2e/issues/71 for some difficulties with \bm.
	\usepackage{bm}
	\nottoggle{LCpres}{
	%https://ctan.org/pkg/amsmath recommends ntheorem, which supersedes amsthm, which corrects the spacing of proclamations and allows for theoremstyle. Option standard loads amssymb and latexsym. Must be loaded after amsmath (from ntheorem doc). From cleveref doc, “ntheorem is fully supported and even recommended”; says to load cleveref after ntheorem. When used with tikzposter, warns about size substitution for the lasy (latexsym) font when using \url, because ntheorem loads latexsym; relatedly (but not directly related to ntheorem), size substitution warning with the cmex font happens when loading amsmath and using \url.
		\usepackage[thmmarks, amsmath, standard, hyperref]{ntheorem}
		%empheq doc says to do this after loading ntheorem
		\usetagform{default}
	%Provides \cref. Unfortunately, cref fails when the language is French and referring to a label whose name contains a colon (https://tex.stackexchange.com/q/83798). Use \cref{sec\string:intro} to work around this. cleveref should go “laster” than hyperref.
		\usepackage[capitalise]{cleveref}
	}{
	}
	\nottoggle{LCposter}{
	%Equations get numbers iff they are referenced. Loading order should be “amsmath → hyperref → cleveref → autonum”, according to autonum doc. Use this in preference to the showonlyrefs option from mathtools, see https://tex.stackexchange.com/q/459918 and autonum doc. See https://tex.stackexchange.com/a/285953 for the etex line. Incompatible with my version of tikzposter (produces “! Improper \prevdepth”).
		\expandafter\def\csname ver@etex.sty\endcsname{3000/12/31}\let\globcount\newcount
		\usepackage{autonum}
	}{
	}
%Also loaded by tikz.
	\usepackage{xcolor}
\iftoggle{LCpres}{
	\usepackage{tikz}
	%\usetikzlibrary{babel, matrix, fit, plotmarks, calc, trees, shapes.geometric, positioning, plothandlers, arrows, shapes.multipart}
}{
}
%Vizualization, on top of TikZ
	%\usepackage{pgfplots}
	%\pgfplotsset{compat=1.14}
\usepackage{graphicx}
	\graphicspath{{graphics/}}

%Provides \print­length{length}, useful for debugging.
	%\usepackage{printlen}
	%\uselengthunit{mm}

\iftoggle{LCpres}{
	\usepackage{appendixnumberbeamer}
	%I have yet to see anyone actually use these navigation symbols; let’s disable them
	\setbeamertemplate{navigation symbols}{} 
	\usepackage{preamble/beamerthemeParisFrance}
	\setcounter{tocdepth}{10}
}{
}

%Do not use the displaymath environment: use equation. Do not use the eqnarray or eqnarray* environments: use align(*). This improves spacing. (See l2tabu or amsldoc.)


%Requires package xcolor.
\newcommand{\commentOC}[1]{\textcolor{blue}{\small$\big[$OC: #1$\big]$}}
%Requires package babel and option [french]. According to babel doc, need two braces around \selectlanguage to make the changes really local.
\newcommand{\commentOCf}[1]{\textcolor{blue}{{\small\selectlanguage{french}$\big[$OC : #1$\big]$}}}
\newcommand{\commentYM}[1]{\textcolor{red}{\small$\big[$YM: #1$\big]$}}
\newcommand{\commentYMf}[1]{\textcolor{red}{{\small\selectlanguage{french}$\big[$YM : #1$\big]$}}}

\bibliographystyle{abbrvnat}

%https://tex.stackexchange.com/a/467188 - uncomment if one of those symbols is used.
%\DeclareFontFamily{U} {MnSymbolD}{}
%\DeclareFontShape{U}{MnSymbolD}{m}{n}{
%  <-6> MnSymbolD5
%  <6-7> MnSymbolD6
%  <7-8> MnSymbolD7
%  <8-9> MnSymbolD8
%  <9-10> MnSymbolD9
%  <10-12> MnSymbolD10
%  <12-> MnSymbolD12}{}
%\DeclareFontShape{U}{MnSymbolD}{b}{n}{
%  <-6> MnSymbolD-Bold5
%  <6-7> MnSymbolD-Bold6
%  <7-8> MnSymbolD-Bold7
%  <8-9> MnSymbolD-Bold8
%  <9-10> MnSymbolD-Bold9
%  <10-12> MnSymbolD-Bold10
%  <12-> MnSymbolD-Bold12}{}
%\DeclareSymbolFont{MnSyD} {U} {MnSymbolD}{m}{n}
%\DeclareMathSymbol{\ntriplesim}{\mathrel}{MnSyD}{126}
%\DeclareMathSymbol{\nlessgtr}{\mathrel}{MnSyD}{192}
%\DeclareMathSymbol{\ngtrless}{\mathrel}{MnSyD}{193}
%\DeclareMathSymbol{\nlesseqgtr}{\mathrel}{MnSyD}{194}
%\DeclareMathSymbol{\ngtreqless}{\mathrel}{MnSyD}{195}
%\DeclareMathSymbol{\nlesseqgtrslant}{\mathrel}{MnSyD}{198}
%\DeclareMathSymbol{\ngtreqlessslant}{\mathrel}{MnSyD}{199}
%\DeclareMathSymbol{\npreccurlyeq}{\mathrel}{MnSyD}{228}
%\DeclareMathSymbol{\nsucccurlyeq}{\mathrel}{MnSyD}{229}

%03B3 Greek Small Letter Gamma
\newunicodechar{γ}{\gamma}
%03B4 Greek Small Letter Delta
\newunicodechar{δ}{\delta}
%2115 Double-Struck Capital N
\newunicodechar{ℕ}{\mathbb{N}}
%211D Double-Struck Capital R
\newunicodechar{ℝ}{\mathbb{R}}
%21CF Rightwards Double Arrow with Stroke
\newunicodechar{⇏}{\nRightarrow}
%21D2 Rightwards Double Arrow
\newunicodechar{⇒}{\ensuremath{\Rightarrow}}
%21D4 Left Right Double Arrow
\newunicodechar{⇔}{\Leftrightarrow}
%21DD Rightwards Squiggle Arrow
\newunicodechar{⇝}{\rightsquigarrow}
%2212 Minus Sign
\newunicodechar{−}{\ifmmode{-}\else\textminus\fi}
%2227 Logical And
\newunicodechar{∧}{\land}
%2228 Logical Or
\newunicodechar{∨}{\lor}
%2229 Intersection
\newunicodechar{∩}{\cap}
%222A Union
\newunicodechar{∪}{\cup}
%2260 Not Equal To (handy also as text in informal writing)
\newunicodechar{≠}{\ensuremath{\neq}}
%2264 Less-Than or Equal To
\newunicodechar{≤}{\leq}
%2265 Greater-Than or Equal To
\newunicodechar{≥}{\geq}
%2270 Neither Less-Than nor Equal To
\newunicodechar{≰}{\nleq}
%2271 Neither Greater-Than nor Equal To
\newunicodechar{≱}{\ngeq}
%2272 Less-Than or Equivalent To
\newunicodechar{≲}{\lesssim}
%2273 Greater-Than or Equivalent To
\newunicodechar{≳}{\gtrsim}
%2274 Neither Less-Than nor Equivalent To – also, from MnSymbol: \nprecsim, a more exact match to the Unicode symbol; and \npreccurlyeq, too small
\newunicodechar{≴}{\not\preccurlyeq}
%2275 Neither Greater-Than nor Equivalent To
\newunicodechar{≵}{\not\succcurlyeq}
%2279 Neither Greater-Than nor Less-Than – requires MnSymbol; also \nlessgtr from txfonts/pxfonts, \ngtreqless from MnSymbol (but much higher), \ngtrless from MnSymbol (a more exact match to the Unicode symbol); for incomparability (not matching this Unicode symbol), may also consider \ntriplesim from MnSymbol,\nparallelslant from fourier, \between from mathabx, or ⋈
\newunicodechar{≹}{\ngtreqlessslant}
%227A Precedes
\newunicodechar{≺}{\prec}
%227B Succeeds
\newunicodechar{≻}{\succ}
%227C Precedes or Equal To
\newunicodechar{≼}{\preccurlyeq}
%227D Succeeds or Equal To
\newunicodechar{≽}{\succcurlyeq}
%227E Precedes or Equivalent To
\newunicodechar{≾}{\precsim}
%227F Succeeds or Equivalent To
\newunicodechar{≿}{\succsim}
%2280 Does Not Precede
\newunicodechar{⊀}{\nprec}
%2281 Does Not Succeed
\newunicodechar{⊁}{\nsucc}
%22B2 Normal Subgroup Of – \triangleleft is too small compared to \trianglelefteq and the like
\newunicodechar{⊲}{\lhd}
%22B3 Contains as Normal Subgroup
\newunicodechar{⊳}{\rhd}
%22B4 Normal Subgroup of or Equal To
\newunicodechar{⊴}{\trianglelefteq}
%22B5 Contains as Normal Subgroup or Equal To
\newunicodechar{⊵}{\trianglerighteq}
%22C8 Bowtie
\newunicodechar{⋈}{\bowtie}
%22EA Not Normal Subgroup Of
\newunicodechar{⋪}{\ntriangleleft}
%22EB Does Not Contain As Normal Subgroup
\newunicodechar{⋫}{\ntriangleright}
%22EC Not Normal Subgroup of or Equal To
\newunicodechar{⋬}{\ntrianglelefteq}
%22ED Does Not Contain as Normal Subgroup or Equal
\newunicodechar{⋭}{\ntrianglerighteq}
%25A1 White Square
\newunicodechar{□}{\Box}
%27E6 Mathematical Left White Square Bracket – there’s also \llbracket from stmaryrd
\newunicodechar{⟦}{\text{\textlbrackdbl}}
%27E7 Mathematical Right White Square Bracket – there’s also \rrbracket from stmaryrd
\newunicodechar{⟧}{\text{\textrbrackdbl}}
%27FC Long Rightwards Arrow from Bar
\newunicodechar{⟼}{\longmapsto}
%2AB0 Succeeds Above Single-Line Equals Sign
\newunicodechar{⪰}{\succeq}
%301A Left White Square Bracket
\newunicodechar{〚}{\textlbrackdbl}
%301B Right White Square Bracket
\newunicodechar{〛}{\textrbrackdbl}
%→ is defined by default as \textrightarrow, which is invalid in math mode. Same thing for the three other commands. I redefine those four using \DeclareUnicodeCharacter instead of \newunicodechar because the latter warns about the previous definition.
%→ Rightwards Arrow
\DeclareUnicodeCharacter{2192}{\ifmmode\rightarrow\else\textrightarrow\fi}
%¬ Not Sign
\DeclareUnicodeCharacter{00AC}{\ifmmode\lnot\else\textlnot\fi}
%… Horizontal Ellipsis
\DeclareUnicodeCharacter{2026}{\ifmmode\dots\else\textellipsis\fi}
%× Multiplication Sign
\DeclareUnicodeCharacter{00D7}{\ifmmode\times\else\texttimes\fi}


\NewDocumentCommand{\R}{}{ℝ}
\NewDocumentCommand{\N}{}{ℕ}
%\mathscr is rounder than \mathcal.
\NewDocumentCommand{\powerset}{m}{\mathscr{P}(#1)}
%Powerset without zero.
\NewDocumentCommand{\powersetz}{m}{\mathscr{P}^*(#1)}
%https://tex.stackexchange.com/a/45732, works within both \set and \set*, same spacing than \mid (https://tex.stackexchange.com/a/52905).
\NewDocumentCommand{\suchthat}{}{\;\ifnum\currentgrouptype=16 \middle\fi|\;}
%Integer interval.
\NewDocumentCommand{\intvl}{m}{⟦#1⟧}
%Allows for \abs and \abs*, which resizes the delimiters.
\DeclarePairedDelimiter\abs{\lvert}{\rvert}
\DeclarePairedDelimiter\card{\lvert}{\rvert}
%Perhaps should use U+2016 ‖ DOUBLE VERTICAL LINE here?
\DeclarePairedDelimiter\norm{\lVert}{\rVert}
%Better than using the package braket because braket introduces possibly undesirable space. Then: \begin{equation}\set*{x \in \R^2 \suchthat \norm{x}<5}\end{equation}.
\DeclarePairedDelimiter\set{\{}{\}}
\DeclarePairedDelimiter\ceil{\lceil}{\rceil}
\DeclarePairedDelimiter\floor{\lfloor}{\rfloor}
\DeclareMathOperator*{\argmax}{arg\,max}
\DeclareMathOperator*{\argmin}{arg\,min}

%We want the straight form of \phi for mathematics, as recommended in UTR #25: Unicode support for mathematics, and thus use \phi for the mathematical symbol and not \varphi; and similarly \epsilon is preferred to \varepsilon for the mathematical symbol.

%The amssymb solution.
%\NewDocumentCommand{\restr}{mm}{{#1}_{\restriction #2}}
%Another acceptable solution.
%\NewDocumentCommand{\restr}{mm}{{#1|}_{#2}}
%https://tex.stackexchange.com/a/278631; drawback being that sometimes the text collides with the line below.
\NewDocumentCommand\restr{mm}{#1\raisebox{-.5ex}{$|$}_{#2}}


%Decision Theory (MCDA and SC)
\NewDocumentCommand{\allalts}{}{A}
\NewDocumentCommand{\allcrits}{}{\mathscr{C}}
\NewDocumentCommand{\alts}{}{A}
\NewDocumentCommand{\dm}{}{i}
\NewDocumentCommand{\allF}{}{\mathscr{F}}
\NewDocumentCommand{\allvoters}{}{\mathscr{N}}
\NewDocumentCommand{\voters}{}{N}
\NewDocumentCommand{\allprofs}{}{\boldsymbol{\mathcal{R}}}
\NewDocumentCommand{\prof}{}{P}
\NewDocumentCommand{\ibar}{}{\overline{i}}
\NewDocumentCommand{\lprof}{}{\lambda_P}
\NewDocumentCommand{\lprofi}{O{x}}{\lambda_P(#1)_i}
\NewDocumentCommand{\lprofibar}{O{x}}{\lambda_P(#1)_{\overline{i}}}
\NewDocumentCommand{\ineq}{}{(\sigma \circ \lambda_P)}

\NewDocumentCommand{\linors}{}{\mathcal{L}(\allalts)}
%Thanks to https://tex.stackexchange.com/q/154549
	%\makeatletter
	%\def\@myRgood@#1#2{\mathrel{R^X_{#2}}}
	%\def\myRgood{\@ifnextchar_{\@myRgood@}{\mathrel{R^X}}}
	%\makeatother
\NewDocumentCommand{\pref}{}{\succ}
\NewDocumentCommand{\prefi}{O{i}}{\succ_{#1}}
\NewDocumentCommand{\paretopt}{}{\text{PO}}
\NewDocumentCommand{\SPPd}{}{\Sigma^\text{PPd}}
\NewDocumentCommand{\SAll}{}{\Sigma^\text{All}}
\NewDocumentCommand{\SThreshold}{}{\Sigma_\text{threshold}}
\NewDocumentCommand{\vpr}{}{\boldsymbol{v}}

\NewDocumentCommand{\musigma}{O{\sigma}O{P}}{\argmin_{A}({#1}\circ\lambda_{{#2}})}
\NewDocumentCommand{\mustar}{O{\sigma}O{P}}{\argmin_{\paretopt({#2})} ({#1} \circ \lambda_{#2})}
\NewDocumentCommand{\minineq}{O{\allalts}}{\argmin_{#1}(\sigma \circ \lambda)}
\NewDocumentCommand{\FBP}{}{\text{FB}(P)}
\NewDocumentCommand{\POP}{}{\text{PO}(P)}

\NewDocumentCommand{\alllosses}{}{\intvl{0, m-1}^N}

\NewDocumentCommand{\Ptop}{}{\bar{P}}
\NewDocumentCommand{\sigmatop}{}{\bar{\sigma}}

\NewDocumentCommand{\fltwo}{}{\floor{\bar{l_2}}}
\NewDocumentCommand{\bltwo}{}{\bar{l_2}}

\newtheorem{conjecture}{Conjecture}

%\newcommand{\tikzmark}[1]{%
	\tikz[overlay, remember picture, baseline=(#1.base)] \node (#1) {};%
}

\newlength{\GraphsDNodeSep}
\setlength{\GraphsDNodeSep}{7mm}
\tikzset{/GraphsD/dot/.style={
	shape=circle, fill=black, inner sep=0, minimum size=1mm
}}

% MCDA Drawing Sorting
\newlength{\MCDSCatHeight}
\setlength{\MCDSCatHeight}{6mm}
\newlength{\MCDSAltHeight}
\setlength{\MCDSAltHeight}{4mm}
%separation between two vertical alts
\newlength{\MCDSAltSep}
\setlength{\MCDSAltSep}{2mm}
\newlength{\MCDSCatWidth}
\setlength{\MCDSCatWidth}{3cm}
\newlength{\MCDSAltWidth}
\setlength{\MCDSAltWidth}{2.5cm}
\newlength{\MCDSEvalRowHeight}
\setlength{\MCDSEvalRowHeight}{6mm}
\newlength{\MCDSAltsToCatsSep}
\setlength{\MCDSAltsToCatsSep}{1.5cm}
\newcounter{MCDSNbAlts}
\newcounter{MCDSNbCats}
\newlength{\MCDSArrowDownOffset}
\setlength{\MCDSArrowDownOffset}{0mm}
\tikzset{/MCD/S/alt/.style={
	shape=rectangle, draw=black, inner sep=0, minimum height=\MCDSAltHeight, minimum width=\MCDSAltWidth
}}
\tikzset{/MCD/S/pref/.style={
	shape=ellipse, draw=gray, thick
}}
\tikzset{/MCD/S/cat/.style={
	shape=rectangle, draw=black, inner sep=0, minimum height=\MCDSCatHeight, minimum width=\MCDSCatWidth
}}
\tikzset{/MCD/S/evals matrix/.style={
	matrix, row sep=-\pgflinewidth, column sep=-\pgflinewidth, nodes={shape=rectangle, draw=black, inner sep=0mm, text depth=0.5ex, text height=1em, minimum height=\MCDSEvalRowHeight, minimum width=12mm}, nodes in empty cells, matrix of nodes, inner sep=0mm, outer sep=0mm, row 1/.style={nodes={draw=none, minimum height=0em, text height=, inner ysep=1mm}}
}}

%Git
\newlength{\GitDCommitSep}
\setlength{\GitDCommitSep}{13mm}
\tikzset{/GitD/commit/.style={
	shape=rectangle, draw, minimum width=4em, minimum height=0.6cm
}}
\tikzset{/GitD/branch/.style={
	shape=ellipse, draw, red
}}
\tikzset{/GitD/head/.style={
	shape=ellipse, draw, fill=yellow
}}

%Social Choice
\tikzset{/SCD/profile matrix/.style={
	matrix of math nodes, column sep=3mm, row sep=2mm, nodes={inner sep=0.5mm, anchor=base}
}}
\tikzset{/SCD/rank-profile matrix/.style={
	matrix of math nodes, column sep=3mm, row sep=2mm, nodes={anchor=base}, column 1/.style={nodes={inner sep=0.5mm}}, row 1/.style={nodes={inner sep=0.5mm}}
}}
\tikzset{/SCD/rank-vector/.style={
	draw, rectangle, inner sep=0, outer sep=1mm
}}
\tikzset{/SCD/isolated rank-vector/.style={
	draw, matrix of math nodes, column sep=3mm, inner sep=0, matrix anchor=base, nodes={anchor=base, inner sep=.33em}, ampersand replacement=\&
}}

% GUI
\tikzset{/GUID/button/.style={
	rectangle, very thick, rounded corners, draw=black, fill=black!40%, top color=black!70, bottom color=white
}}

% Logger objects
\tikzset{/loggerD/main/.style={
	shape=rectangle, draw=black, inner sep=1ex, minimum height=7mm
}}
\tikzset{/loggerD/helper/.style={
	shape=rectangle, draw=black, dashed, minimum height=7mm
}}
\tikzset{/loggerD/helper line/.style={
	<->, draw, dotted
}}

% Beliefs
\tikzset{/BeliefsD/attacker/.style={
	shape=rectangle, draw, minimum size=8mm
}}
\tikzset{/BeliefsD/supporter/.style={
	shape=circle, draw
}}


%\DeclareAcronym{AMCD}{short=amcd, long={Aide Multicritère à la Décision}}
\DeclareAcronym{AR}{short=ar, long={Argumentative Recommender}}
\DeclareAcronym{DA}{short=da, long={Decision Analysis}}
\DeclareAcronym{DJ}{short=dj, long={Deliberated Judgment}}
\DeclareAcronym{DM}{short=dm, long={Decision Maker}}
\DeclareAcronym{DP}{short=dp, long={Deliberated Preference}}
\DeclareAcronym{MAVT}{short=mavt, long={Multiple Attribute Value Theory}}
\DeclareAcronym{MCDA}{short=mcda, long={Multicriteria Decision Aid}}
\DeclareAcronym{MIP}{short=mip, long={Mixed Integer Program}}


%\input{preamble/refAPIcmds}

%I find these settings useful in draft mode. Should be removed for final versions.
	%Which line breaks are chosen: accept worse lines, therefore reducing risk of overfull lines. Default = 200.
		\tolerance=2000
	%Accept overfull hbox up to...
		\hfuzz=2cm
	%Reduces verbosity about the bad line breaks.
		\hbadness 5000
	%Reduces verbosity about the underful vboxes.
		\vbadness=1300

%\title{Title \thanks{Thanks.}}
%\author{Name1}
%\author{Name2}
%\affil{Université Paris-Dauphine, PSL Research University, CNRS, LAMSADE, 75016 PARIS, FRANCE\\
%	\href{mailto:olivier.cailloux@dauphine.fr}{olivier.cailloux@dauphine.fr}
%}
%\author{Name3}
%\affil{Affil2}
%\hypersetup{
%	pdfsubject={},
%	pdfkeywords={},
%}

\setcounter{MaxMatrixCols}{10}
%TCIDATA{OutputFilter=LATEX.DLL}
%TCIDATA{Version=5.50.0.2953}
%TCIDATA{<META NAME="SaveForMode" CONTENT="1">}
%TCIDATA{BibliographyScheme=Manual}
%TCIDATA{Created=Tuesday, August 09, 2005 15:34:58}
%TCIDATA{LastRevised=Thursday, March 07, 2019 13:36:58}
%TCIDATA{<META NAME="GraphicsSave" CONTENT="32">}
%TCIDATA{<META NAME="DocumentShell" CONTENT="Scientific Notebook\Blank with Theorem Tags">}
%TCIDATA{Language=American English}
%TCIDATA{CSTFile=Math.cst}
%TCIDATA{PageSetup=14,14,57,57,0}
%TCIDATA{AllPages=
%H=36
%F=36,\PARA{038<p type="texpara" tag="Body Text" >\hfill \thepage}
%}

%\newtheorem{theorem}{Theorem}[section]
\newtheorem{acknowledgement}[theorem]{Acknowledgement}
\newtheorem{algorithm}[theorem]{Algorithm}
\newtheorem{axiom}{Axiom}[section]
\newtheorem{case}{Case}[section]
\newtheorem{claim}{Claim}[section]
\newtheorem{conclusion}{Conclusion}[section]
\newtheorem{condition}{Condition}[section]
\newtheorem{conjecture}{Conjecture}[section]
%\newtheorem{corollary}{Corallary}[section]
\newtheorem{criterion}{Criterion}[section]
%\newtheorem{definition}{Definition}[section]
%\newtheorem{example}{Example}[section]
\newtheorem{exercise}{Exercise}[section]
%\newtheorem{lemma}{Lemma}[section]
\newtheorem{notation}{Notation}[section]
\newtheorem{problem}{Problem}[section]
%\newtheorem{proposition}{Proposition}[section]
%\newtheorem{remark}{Remark}[section]
\newtheorem{solution}{Solution}[section]
\newtheorem{summary}{Summary}[section]
%\newenvironment{proof}[1][Proof]{\noindent\textbf{#1.} }{\ \rule{0.5em}{0.5em}}

\usepackage{blkarray}
\usepackage{booktabs}
\usepackage{amsmath}

\title{Ex-Ante versus Ex-Post \\ Compromise}
\author{}
\date{}

\newcommand{\commentBN}[1]{\textcolor{magenta}{\small$\big[$BN: #1$\big]$}}
\newcommand{\commentRS}[1]{\textcolor{red}{\small$\big[$RS: #1$\big]$}}
\newcommand{\paretopt}{\mathit{PO}}
\newcommand{\SPPd}{\Sigma^\text{PPd}}
\newcommand{\SAll}{\Sigma^\text{All}}
\newcommand{\SThreshold}{\Sigma_\text{threshold}}
\newcommand{\vpr}{\mathbf{v}}

\begin{document}

\maketitle
\thispagestyle{empty}

\begin{abstract}
	A classical social choice setting is composed of a group of individuals, or voters, that express their preferences over a set of alternatives. The social choice problem consists in defining a procedure able to determine a collective choice for this group of voters, starting from their individual preferences. Such procedure is called social choice rule and it can be defined as a function mapping preference profiles to alternatives. Depending on the properties that this function satisfies, very different outcomes can be produced starting from the same initial profile. The plurality rule is one of the most common social choice rule and it consists in selecting, as a winner, the alternative that is considered the best by the largest number of voters forming the society. Yet, this rule can pick, as a winner, an alternative that is considered the worst by a strict majority of voters. Such outcome may be undesirable. Several procedures, the so-called compromise rules, have been proposed in the literature that aim to find a compromise. Nevertheless, all those rules can be defined as \textit{ex-ante compromises} or \textit{procedural compromises}, i.e., they impose over individuals a willingness to compromise but they do not ensure an outcome where everyone has effectively compromised. In this work, we approach the problem of compromise from an \textit{ex-post} perspective, favoring an outcome where every voter gives up her most preferred positions if this increases equality. We propose a new notion of compromise in the social choice context, considering ordinal utilities. 
\end{abstract}

\pagebreak

\section{Introduction and Related Works}

In a social choice situation where individuals have preferences over several alternatives there is no unique procedure for selecting a common agreement between them. One of the way to define such a procedure is to prefer the quantity of support behind an alternative (i.e. the number of voters who favor it) to the quality of its support (i.e. until which level in voters’ preferences is legitimate to descend) or vice-versa. Deciding to give more importance to the first may result in a lack of a majority support or, worst, into a strict majority of individuals which strongly dislikes the chosen candidates. Vice versa, favoring the number of voters behind an alternative implies descending levels in individual preferences potentially ending on electing a generally disliked candidate because of the opposition of very few individuals. 

Using the plurality rule, for example, the alternatives which are ranked the best by the largest number of voters are picked as outcome. The problem of this procedure is that, as long as there are more than two alternatives, the outcome may lack the support of any majority. In fact, it may also be opposed by a strict majority of agents each of whom ranks the chosen candidate as their least preferred choice. 

All the rules existing in literature aiming at finding a compromise between voters end up facing the problem of balancing the trade-off between the quality and quantity of support behind alternatives. \citet{Merlin2019} gather them in the class of Compromise Rules. The Majoritarian Compromise \citep{Sertel1999} selects candidates who have the support of a majority in the best degree possible. 
We call BK-compromises the class of $q$-approval fallback bargaining rules introduced by \citet{Brams2001}: the $q$-approval fallback bargaining rule picks ranks 1, 2, …, and stops as soon as at least $q$ voters rank some alternative at the chosen rank or better. All such alternatives form the compromise set, and, among them, the ones which receive the highest support are elected.
We start by an example, based on an observation made by Laslier at Buyukada, that will help us to define the problem and to give the motivation for our approach.


\begin{example}
	\label{ex:ex1}
	Consider the following preference profile $P\in L(A)^{N}$ with a set $N$ of $n$ voters and a set $A$ of alternatives. $L(A)$ represents the set of linear orders over $A$.
	\begin{center}
		$
		\begin{array}{cccc}
		\mathbf{n-1} \quad &a&b&c\\
		\mathbf{1} \quad &c&b&a\\
		\end{array}
		$
	\end{center}
	which represents $n-1$ individuals who prefer $a$ to $b$, $b$ to $c$, hence $a$ to $c$; and one individual who prefers $c$ to $b$, $b$ to $a$, hence $c$ to $a $. At $P$ all BK-compromises, except fallback bargaining i.e. when $q=n$, pick $a$ as the winner, which does not appear as a compromise as a voter ranks it as her least preferred alternative.
\end{example}

Please note that the group of ignored voters could be even larger.
\begin{example}
	Consider the following preference profile with $n=100$:
	\begin{center}
		$
		\begin{array}{cccc}
		\mathbf{51} \quad &a&b&c\\
		\mathbf{49} \quad &c&b&a\\
		\end{array}
		$
	\end{center}
	When $q\in \left\{ 1,..., \frac{n}{2} +1\right\} $, all BK-compromises pick $a$, and, again, it does not appear as a compromise as 51 voters reach their best alternative while the remaining 49 voters have to be contented with their worst one. Note that for $q\in \left\{ 1,..., \frac{n}{2} -1 \right\} $ the set of possible common agreements determined by the fallback bargaining procedure is $\{a,c\}$. Nevertheless, $a$ receives the highest support thus it is elected.
\end{example}

These two examples illustrate that BK-compromises (except fallback bargaining (FB) where $q=n$) are “ex-ante compromises” or “procedural compromises”, i.e., they impose over individuals a willingness to compromise but they don't ensure an outcome where everyone has effectively compromised. So what is a compromise?

%In the case where the candidates are only two we talk about bargaining \citep{Thomson1994}. The bargaining problem was first axiomatized by \citet{Nash1950}, who also proposed a solution for the problem that is obtained by maximizing the product of the differences between the utility from an agreement and the one from a disagreement (namely the status quo). This is the only bargaining solution that satisfies all the following axioms: independence of equivalent utility transformations, independence of irrelevant alternatives, pareto optimality and symmetry.

\section{Basic notions and notation}
Consider a finite set $N$ of individuals with $\#N=n\geq 2$ and a finite set $A$ of alternatives with $\#A=m\geq 2$. We write $L(A)$ for the set of linear orders over $A$ and $P_{i}\in L(A)$ stands for the preference of $i\in N$. An ordinal social choice rule (SCR) is a mapping $f:L(A)^{N}\rightarrow 2^{A} \setminus \{\emptyset \}$. For each $x \in A$ and $i \in N$ let $r_i(x)$ be the rank of the alternative $x$ in the preference order of $i$; we write with $\lambda^{P}_i(x)=r_{i}(x)-1$ the loss of ranks for $i$ when electing $x$ instead of her favorite alternative. This value can also be seen as the number of pairwise flips we need to perform in order to place $x$ at top position for $i$. Thus, $\lambda ^{P}(x)=(\lambda _{i}^{P}(x))_{i\in N}\in \R_+^N$ represents the vector of losses when electing $x$. 
We define a spread measure $\sigma \in \SAll$ as a function $\sigma : \R_{+}^{N}\longrightarrow \R_{+}$ such that $\sigma(l_1, \dots, l_n)=0 \iff l_i=l_j \forall i,j \in \{1, \dots, n\}$ given a vector $l \in R_{+}^{N}$. When all the values are the same, a coefficient of zero expresses perfect equality.

\section{Compromises}


\begin{definition}[Compromise]
	A SCR $f$ is a compromise iff $f(P) \subseteq \argmin_{x \in A} (\sigma(\lambda^P(x))$, $\forall P \in L(A)^N$.
\end{definition}


	Note that $\paretopt(P)$ be the set of alternatives which are Pareto optimal at $P$. Let $f(P)$ be the set of the alternatives elected by a compromise SCR $f$, $\paretopt(P)$ and $f(P)$ might be disjoint.
	\begin{example} Consider the following profile $P$
		\begin{center}
			$
			\begin{array}{cccc}
			\mathbf{k} \quad &a&b&c\\
			\mathbf{n-k} \quad &b&a&c\\
			\end{array}
			$
		\end{center}
	For each $\sigma \in \SAll$ we have $\sigma(\lambda^P(a))>0$, $\sigma(\lambda^P(b))>0$, $\sigma(\lambda^P(c))=0$. A compromise rule would, in this case, elect $c$ which is a pareto dominated alternative.
	\end{example}

\begin{definition}[Efficient Compromise]
	A SCR $f$ is an efficient compromise iff $f(P) \subseteq \argmin_{x \in \paretopt(P)} (\sigma(\lambda^P(x))$, $\forall P \in L(A)^N$.
\end{definition}

\noindent Consider again the profile $P$ of the Example \ref{ex:ex1}. 
	\begin{center}
		$
		\begin{array}{cccc}
		\mathbf{n-1} \quad &a&b&c\\
		\mathbf{1} \quad &c&b&a\\
		\end{array}
		$
	\end{center}
	At $P$, $b$ is the only efficient \commentBN{efficient egalitarian alternative}, and we can say that 
\begin{itemize}
	\item Condorcet is not an efficient compromise since it picks $a$ as the winner.
	\item No scoring rule except anti-plurality is an efficient compromise (see Proof below).
	\item No BK-compromise is an efficient compromise, unless $q=n$ which is fallback bargaining. Indeed for $q=\{1,\dots, n-1\}$ the alternative selected by the BK-compromise rules is $a$.
	\item No preferential runoff method is an efficient compromise since it always selects $a$ in the first round.
\end{itemize}
As mentioned above, for the considered example no scoring rule except anti-plurality is an efficient compromise.
\begin{proof} Without loss of generality we can consider the scoring function $s:\{1,\dots,m\}\rightarrow [0,1]$ as a function assigning $1$ to the alternative ranked at the first position, and $0$ to the one ranked at the last. Therefore, the alternative ranked in the second position will have a score $0 \leq \lambda<1$. We exclude $\lambda = 1$ which is anti-plurality.
	\begin{center}
		$
		\begin{array}{ccc}
		\quad & \mathbf{n-1} &\mathbf{1}\\
		1 \quad &a&c\\
		\lambda \quad &b&b\\
		0 \quad &c&a\\
		\end{array}
		$
	\end{center}
	We have that $s(a)=n-1$, $s(b)=(n-1+1)\cdot \lambda$, $s(c)=1$. We can see that the alternative $a$ will always be selected as winner because $s(a)$ is always greater than $s(b)$ and $s(c)$. For $c$ is trivial, for $b$ we have that $n-1>n\cdot\lambda$ so $n-1>\frac{\lambda}{(1-\lambda)}$ and since $\lambda \neq 1$ that is always the case.
\end{proof}

\commentBN{The following definitions have been characterized on the 15/11 and are susceptible to changes.}
\begin{definition}
	An \textit{unequivocal compromise alternative} is an alternative ranked at the same position by all voters. If it is not Pareto dominated then it is also efficient.
\end{definition}
\begin{conjecture}
	A SCR f is a compromise $\iff$ if an unequivocal compromise alternative exists than f only picks unequivocal compromise alternatives.
\end{conjecture}
\begin{conjecture}
	A SCR f is an efficient compromise $\iff$ if an efficient unequivocal compromise alternative exists than it is picked as a sole winner.
\end{conjecture}
\begin{definition}[Compromise]
	\[ COMP(P) = \argmin_x \sigma(\lambda^P(x)) \]
	\[f(P) \subseteq COMP(P) \qquad \forall P \text{ for some } \sigma\]
\end{definition}
\begin{definition}[Efficient Compromise]
	\[ EffCOMP(P) = \argmin_{x \in \paretopt(P)} \sigma(\lambda^P(x)) \]
	\[f(P) \subseteq EffCOMP(P) \qquad \forall P \text{ for some } \sigma\]
\end{definition}
\begin{definition}[Compromise Compatible]
	A SCR f is compromise compatible $\iff$
	$f(P) \cap COMP(P) \neq 0 \qquad \forall P \text{ for some } \sigma$.
\end{definition}
\begin{definition}[Compromise Compatible]
	A SCR f is efficient compromise compatible $\iff$
	$f(P) \cap EffCOMP(P) \neq 0 \qquad \forall P \text{ for some } \sigma$.
\end{definition}
\begin{proposition}
	Condorcet, Scoring Rules (except Anti-Plurality), BK-Compromise (except FB) are neither compromise nor compromise compatible.
\end{proposition}

Antiplurality and FB are closer to be compromise.

\begin{proposition}
	Anti-plurality is not compromise compatible.
\end{proposition}
\begin{proof} Consider the following profile $P$
	\begin{center}
		$
		\begin{array}{cc}
		a&b\\
		b&a\\
		c&c\\
		\end{array}
		$
	\end{center}
	$COMP(P)=\{c\}$ and $\textit{antiplr}(P)=\{a,b\}$, so $COMP(P) \cap \textit{antyplr}(P)=0$ at $P$.
\end{proof}

\begin{definition}
	\label{def:effcomprcmp}
	A SCR f is efficient compromise compatible $\iff$ if an efficient unequivocal compromise alternative exists it must be among the winner.
\end{definition}

\begin{proposition}
	Efficient anti-plurality is efficient compromise compatible.
\end{proposition}

\begin{proof}
	It follows from \ref{def:effcomprcmp}. TODO
\end{proof}
\begin{example}
	Consider the following profile $P$
	\begin{center}
		$
		\begin{array}{cc}
		a&b\\
		c&c\\
		b&a\\
		d&d\\
		\end{array}
		$
	\end{center}
	$EffCOMP(P)=\{c\}$ and $\textit{effantiplr}(P)=\{a,b,c\}$.
\end{example}
\begin{proposition}
	FB is neither compromise compatible nor efficient compromise compatible.
\end{proposition}
\begin{proof} Consider the following profile $P$
	\begin{center}
		$
		\begin{array}{cc}
		\mathbf{i_1}& \mathbf{i_2} \\
		x &b_1\\
		a_1 &\cdot\\
		\cdot &\cdot\\
		\cdot &b_{k-1}\\
		\cdot &y\\
		a_k &x\\
		y &b_k\\
		b_1 &a_1\\
		\cdot &\cdot\\
		\cdot &\cdot\\
		b_k &a_k\\
		\end{array}
		$
	\end{center}
	$COMP(P)=EffCOMP(P)=\{c\}$ and $\textit{FB}(P)=\{x\}$.
\end{proof}


\bibliography{biblio}
\bibliographystyle{abbrv}

\pagebreak
\appendix
\section{Spread Measures}
In our idea of compromise, we aim to determine the “most equally distributed” vector within $\left\{ \lambda^P(x)\right\} _{x\in A}$. We thus need to adopt a spread measure which is a mapping $\sigma \in \SAll: \R_{+}^{N}\longrightarrow \R_{+}$ which satisfies $\sigma(r_i, \dots, r_n)=0 \iff r_i=r_j \forall i,j$. In what follows, we will consider various classes of possible spread measures.

\begin{definition}[Pairwise Pareto dominance]
	\label{def:PPD}
	For all $r$, $s\in \R_{+}^{N}$: 
	\[\left[\left\vert r_{i}-r_{j}\right\vert \leq \left\vert s_{i}-s_{j}\right\vert \forall i, j\in N\right] ⇒ \sigma (r)\leq \sigma (s).\] 
\end{definition}
We write $\SPPd \subseteq \SAll$ for the class of spread measures that satisfy also PPd.
Given a vector $r \in \R^N$ of $n$ elements, some examples of spread measures are the following.

\subsection{Mean Absolute Difference}
\[\sigma_{mad}(r)= \frac{1}{n^2} \sum_{i=1}^{n}\sum_{j=1}^{n}|r_i-r_j| \]
The mean absolute difference is a measure of dispersion calculated as the arithmetic mean of the absolute value of all possible differences. 
\begin{proposition}
	$\sigma_{mad}$ is a spread measure and it satisfies PPd.
\end{proposition}
\begin{proof} 
	In order to be a spread measure as we defined $\sigma_{mad}$ should be zero when all the elements of $r$ are equals, $r_i-r_j = 0 \ \forall i,j\in {1, \dots, n}$. In this case the sum term $\sum_{i=1}^{n}\sum_{j=1}^{n}|r_i-r_j|$ will always be zero, and thus $\sigma_{mad}(r)=0$ for such $r$. Hence $\sigma_{mad}$ is a spread measure.
	In order to prove the satisfaction of PPd we can observe that, given two vectors $r,s \in \R^N$, if \mbox{$|r_{i}-r_{j}| \leq |s_{i}-s_{j}| \ \forall i,j \in N$} then also \[\sum_{i=1}^{n}\sum_{j=1}^{n}|r_i-r_j| \leq \sum_{i=1}^{n}\sum_{j=1}^{n}|s_i-s_j| \]
	The remaining term $\frac{1}{n^2}$ is constant and it is equal for both sides, so $\sigma_{mad}(r) \leq \sigma_{mad}(s)$ and PPd is satisfied.
\end{proof}
\begin{example}
	\label{ex:spreadVectors}
	\begin{align}
	s=(3,3,3,3) \\
	t=(1,2,3,4) \\
	w=(1,3,5,7) \\
	\end{align}
	In order for a function $\sigma$ to be a spread measure it must be that $\sigma(s)=0$, and to satisfy PPd it must be $\sigma(t)<\sigma(w)$.
	
	Compute now $\sigma_{mad}$ for these vectors, we have:
	\begin{align}
	\sigma_{mad}(s) &= \frac{1}{4^2} \cdot 0 =0 \\
	\sigma_{mad}(t) &= \frac{1+2+3+1+1+2+2+1+1+3+2+1}{4^2}=\frac{20}{16} \\
	\sigma_{mad}(w) &= \frac{2+4+6+2+2+4+4+2+2+6+4+2}{4^2}=\frac{40}{16} \\
	\end{align}
	\begin{align}
	\sigma_{mad}(s) = 0 \ \surd \\
	\sigma_{mad}(t) \leq \sigma_{mad}(w) \rightarrow PPd \ \surd \\
	\end{align}
\end{example}




\subsection{Average Absolute Deviation}
\[\sigma_{ad}(r)= \frac{\sum_{i=1}^{n}|r_i-\bar{r}|}{n}\]
The average absolute deviation considers the differences between the values of the vector and their arithmetic mean $\bar{r}$.

\begin{proposition}
	\label{prop:ad}
	$\sigma_{ad}$ is a spread measure and it satisfies PPd.
\end{proposition}
\begin{proof}
	When $r_i=r_j \ \forall i,j\in {1, \dots, n}$ then $\bar{r}=\frac{\sum_{i=1}^{n}r_i}{n}=r_i$ for any $i \in N$. It follows that $\sum_{i=1}^{n}|r_i-\bar{r}|=0$ and $\sigma_{ad}(r)=0$.
	
	In order to prove that $\sigma_{ad}$ satisfies PPd we have to show that:
	
	\begin{align}
	\sigma_{ad}(r) &\leq \sigma_{ad}(s) \\
	\frac{\sum_{i=1}^{n}|r_i-\bar{r}|}{n} &\leq \frac{\sum_{i=1}^{n}|s_i-\bar{s}|}{n} \\
	\sum_{i=1}^{n}|r_i-\bar{r}| &\leq \sum_{i=1}^{n}|s_i-\bar{s}| \\
	\sum_{i=1}^{n}|r_i-\frac{\sum_{j=1}^{n}r_{j}}{n}| &\leq \sum_{i=1}^{n}|s_i-\frac{\sum_{j=1}^{n}s_{j}}{n}| \\
	\sum_{i=1}^{n}|\frac{n \cdot r_i}{n}-\frac{\sum_{j=1}^{n}r_{j}}{n}| &\leq \sum_{i=1}^{n}|\frac{n \cdot s_i}{n}-\frac{\sum_{j=1}^{n}s_{j}}{n}| \\
	\sum_{i=1}^{n}|\frac{\sum_{j=1}^{n}r_i}{n}-\frac{\sum_{j=1}^{n}r_{j}}{n}| &\leq \sum_{i=1}^{n}|\frac{\sum_{j=1}^{n}s_i}{n}-\frac{\sum_{j=1}^{n}s_{j}}{n}| \\
	\sum_{i=1}^{n}|\frac{\sum_{j=1}^{n}r_i - r_j}{n}| &\leq \sum_{i=1}^{n}|\frac{\sum_{j=1}^{n}s_i - s_j}{n}| \\
	\sum_{i=1}^{n}|\sum_{j=1}^{n}r_i - r_j| &\leq \sum_{i=1}^{n}|\sum_{j=1}^{n}s_i - s_j| \\
	\end{align}
	But we can consider that 
	\begin{itemize}
		\item $\sum_{i=1}^{n}|\sum_{j=1}^{n}r_i - r_j| \geq |\sum_{i=1}^{n}\sum_{j=1}^{n}r_i - r_j|$
		\item $\sum_{i=1}^{n}|\sum_{j=1}^{n}s_i - s_j| \leq \sum_{i=1}^{n}\sum_{j=1}^{n}|s_i - s_j|$
	\end{itemize}
	and, therefore, go back to our proof showing that
	\begin{align}
	|\sum_{i=1}^{n}\sum_{j=1}^{n}r_i - r_j| \leq \sum_{i=1}^{n}|\sum_{j=1}^{n}r_i - r_j| &\leq \sum_{i=1}^{n}|\sum_{j=1}^{n}s_i - s_j| \leq \sum_{i=1}^{n}\sum_{j=1}^{n}|s_i - s_j| \\
	|\sum_{i=1}^{n}\sum_{j=1}^{n}r_i - r_j| &\leq \sum_{i=1}^{n}\sum_{j=1}^{n}|s_i - s_j|
	\end{align}
	By assumption	
	\begin{align}
	|r_{i}-r_{j}| &\leq |s_{i}-s_{j}| \ \forall i,j \in N \\ \sum_{i=1}^{n}\sum_{j=1}^{n}|r_i-r_j| &\leq \sum_{i=1}^{n}\sum_{j=1}^{n}|s_i-s_j| \\
	\end{align}
	and by triangle inequality
	\begin{align}
	|\sum_{i=1}^{n}\sum_{j=1}^{n}r_i-r_j| &\leq \sum_{i=1}^{n}\sum_{j=1}^{n}|r_i-r_j| \\
	\end{align}
	it follows that 
	\begin{align}
	|\sum_{i=1}^{n}\sum_{j=1}^{n}r_i-r_j| \leq \sum_{i=1}^{n}\sum_{j=1}^{n}|r_i-r_j| \leq \sum_{i=1}^{n}\sum_{j=1}^{n}|s_i-s_j| \\
	\end{align}
	This concludes the proof.
\end{proof}

Consider the vectors of the Example \ref{ex:spreadVectors}.
If we compute $\sigma_{ad}$ for those vectors we have:
\begin{align}
\sigma_{ad}(s) &= \frac{\sum_{i=1}^{4} |3-3|}{4} =0 \\
\sigma_{ad}(t) &= \frac{|1-2.5|+|2-2.5|+|3-2.5|+|4-2.5|}{4}=\frac{4}{4}=1 \\
\sigma_{ad}(w) &= \frac{|1-4|+|3-4|+|5-4|+|7-4|}{4}=\frac{8}{4}=2 \\
\end{align}
\begin{align}
\sigma_{ad}(s) = 0 \ \surd \\
\sigma_{ad}(t) \leq \sigma_{ad}(w) \rightarrow PPd \ \surd \\
\end{align}

\subsection{Standard Deviation}
\[\sigma_{sd}(r)= \sqrt{\frac{\sum_{i=1}^{n}(r_i-\bar{r})^2}{n}}\]
\begin{proposition}
	$\sigma_{sd}$ is a spread measure and it satisfies PPd.
\end{proposition}
\begin{proof}
	If $r_i=r_j \ \forall i,j\in {1, \dots, n}$ then $\bar{r}=\frac{\sum_{i=1}^{n}r_i}{n}=r_i$ for any $i \in N$, thus the sum will be zero and $\sigma_{sd}(r)=0$.
	\\ To prove that $\sigma_{sd}$ satisfies PPd we need to show that
	\[\sum_{i=1}^{n}(r_i-\bar{r})^2 \leq \sum_{i=1}^{n}(s_i-\bar{s})^2\]
	Since 
	\[(r_i-\bar{r})^2 = |r_i-\bar{r}|^2 \text{ and } (s_i-\bar{s})^2 = |s_i-\bar{s}|^2 \] 
	we can rewrite it as
	\[\sum_{i=1}^{n}|r_i-\bar{r}|^2 \leq \sum_{i=1}^{n}|s_i-\bar{s}|^2\]
	We know that 
	\[ \sum_{i=1}^{n}|r_i-\bar{r}|^2 \leq (\sum_{i=1}^{n}|r_i-\bar{r}|)^2 \text{ and }
	\sum_{i=1}^{n}|s_i-\bar{s}|^2 \leq (\sum_{i=1}^{n}|s_i-\bar{s}|)^2 \]
	indeed, in case $n=2$, $(a-b)^2 = a^2+2ab+b^2$ so $(a-b)^2 \geq a^2+b^2$ only if $2ab > 0$; in our case $a$ and $b$ are the terms of the summation, respectively $|r_1-\bar{r}|$ and $|r_2-\bar{r}|$ (or $|s_1-\bar{s}|$ and $|s_2-\bar{s}|$) so they are always positive. So we can rewrite our expression as 
	\[\sum_{i=1}^{n}|r_i-\bar{r}|^2 \leq \sum_{i=1}^{n}|s_i-\bar{s}|^2  \leq (\sum_{i=1}^{n}|s_i-\bar{s}|)^2 \]
	and if we show that $\sum_{i=1}^{n}|r_i-\bar{r}|^2 \leq (\sum_{i=1}^{n}|s_i-\bar{s}|)^2 $ then the proof is concluded.
	From the proof of Proposition \ref{prop:ad}, we know that
	\begin{align}
	\sum_{i=1}^{n}|r_i-\bar{r}| &\leq \sum_{i=1}^{n}|s_i-\bar{s}| \\
	(\sum_{i=1}^{n}|r_i-\bar{r}|)^2 &\leq (\sum_{i=1}^{n}|s_i-\bar{s}|)^2
	\end{align}
	so we can affirm that 
	\[\sum_{i=1}^{n}|r_i-\bar{r}|^2 \leq (\sum_{i=1}^{n}|r_i-\bar{r}|)^2 \leq (\sum_{i=1}^{n}|s_i-\bar{s}|)^2 \]
	Therefore $\sum_{i=1}^{n}|r_i-\bar{r}|^2 \leq (\sum_{i=1}^{n}|s_i-\bar{s}|)^2 $ and this concludes the proof.
\end{proof}
Consider the vectors of Example \ref{ex:spreadVectors}.
If we compute $\sigma_{sd}$ for those vectors we have:
\begin{align}
\sigma_{sd}(s) &= \sqrt{\frac{\sum_{i=1}^{4} (3-3)^2}{4}}=0 \\
\sigma_{sd}(t) &= \sqrt{\frac{(1-2.5)^2+(2-2.5)^2+(3-2.5)^2+(4-2.5)^2}{4}}=\sqrt{\frac{5}{4}} \\
\sigma_{sd}(w) &= \sqrt{\frac{(1-4)^2+(3-4)^2+(5-4)^2+(7-4)^2}{4}}=\sqrt{\frac{20}{4}}=\sqrt{5} \\
\end{align}
\begin{align}
\sigma_{sd}(s) = 0 \ \surd \\
\sigma_{sd}(t) \leq \sigma_{sd}(w) \rightarrow PPd \ \surd \\
\end{align}

\subsection{Gini Coefficient}
\[\sigma_{G}(r)= \frac{\sum_{i=1}^{n}\sum_{j=1}^{n}|r_i-r_j|}{2 \cdot n \cdot \sum_{i=1}^{n} r_i}\]
\begin{proposition}
	$\sigma_{G}$ is a spread measure but it does not satisfy PPd.
\end{proposition}
\begin{proof}
	When $r_i=r_j \ \forall i,j\in {1, \dots, n}$ then the numerator of $\sigma_{G}$ is $0$, thus $\sigma_{G}(r)=0$. When $|r_{i}-r_{j}| \leq |s_{i}-s_{j}| \ \forall i,j \in N $ then the numerator of $\sigma_{G}(r)$ is less than or equal to the numerator of $\sigma_{G}(s)$ but it could be that the sum of the elements of $s$ is bigger than the one of $r$, resulting into $\sigma_{G}(s)$ having a greater denominator than $\sigma_{G}(r)$ and therefore $\sigma_{G}(s) \leq \sigma_{G}(r)$. To see that consider the following vectors:
	\begin{align}
	t'=(1,1,2,2) \\
	w'=(2,2,3,3) \\
	\end{align}
	\begin{align}
	\sigma_{G}(t') &= \frac{(1+1)\cdot 4}{2 \cdot 4 \cdot 6}=\frac{1}{6} \\
	\sigma_{G}(w') &= \frac{(1+1)\cdot 4}{2 \cdot 4 \cdot 10}=\frac{1}{10} \\
	\end{align}
	$|t'_{i}-t'_{j}| \leq |w'_{i}-w'_{j}| \forall i, j$ but $\sigma_{G}(t') \nleq \sigma_{G}(w')$
\end{proof}

TODO: Give reference to the relevant literature, also to the Pareto dominance condition.

\end{document}
