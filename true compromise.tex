\RequirePackage[l2tabu, orthodox]{nag}
\documentclass[version=3.21, pagesize, notitlepage, twoside=off, bibliography=totoc, DIV=calc, fontsize=12pt, a4paper]{scrartcl}
%Permits to copy eg x ⪰ y ⇔ v(x) ≥ v(y) from PDF to unicode data, and to search. From pdfTeX users manual. See https://tex.stackexchange.com/posts/comments/1203887.
	\input glyphtounicode
	\pdfgentounicode=1
%Latin Modern has more glyphs than Computer Modern, such as diacritical characters. fntguide commands to load the font before fontenc, to prevent default loading of cmr.
	\usepackage{lmodern}
%Encode resulting accented characters correctly in resulting PDF, permits copy from PDF.
	\usepackage[T1]{fontenc}
%UTF8 seems to be the default in recent TeX installations, but not all, see https://tex.stackexchange.com/a/370280.
	\usepackage[utf8]{inputenc}
%Provides \newunicodechar for easy definition of supplementary UTF8 characters such as → or ≤ for use in source code.
	\usepackage{newunicodechar}
%Text Companion fonts, much used together with CM-like fonts. Provides \texteuro and commands for text mode characters such as \textminus, \textrightarrow, \textlbrackdbl.
	\usepackage{textcomp}
%St Mary’s Road symbol font, used for ⟦ = \llbracket.
	\usepackage{stmaryrd}
\usepackage{centernot}
%Solves bug in lmodern, https://tex.stackexchange.com/a/261188; probably useful only for unusually big font sizes; and probably better to use exscale instead. Note that the authors of exscale write against this trick.
	%\DeclareFontShape{OMX}{cmex}{m}{n}{
		%<-7.5> cmex7
		%<7.5-8.5> cmex8
		%<8.5-9.5> cmex9
		%<9.5-> cmex10
	%}{}
	%\SetSymbolFont{largesymbols}{normal}{OMX}{cmex}{m}{n}
%More symbols (such as \sum) available in bold version, see https://github.com/latex3/latex2e/issues/71.
	\DeclareFontShape{OMX}{cmex}{bx}{n}{%
	   <->sfixed*cmexb10%
	   }{}
	\SetSymbolFont{largesymbols}{bold}{OMX}{cmex}{bx}{n}
%For small caps also in italics, see https://tex.stackexchange.com/questions/32942/italic-shape-needed-in-small-caps-fonts, https://tex.stackexchange.com/questions/284338/italic-small-caps-not-working.
	\usepackage{slantsc}
	\AtBeginDocument{%
		%“Since nearly no font family will contain real italic small caps variants, the best approach is to substitute them by slanted variants.” -- slantsc doc
		%\DeclareFontShape{T1}{lmr}{m}{scit}{<->ssub*lmr/m/scsl}{}%
		%There’s no bold small caps in Latin Modern, we switch to Computer Modern for bold small caps, see https://tex.stackexchange.com/a/22241
		%\DeclareFontShape{T1}{lmr}{bx}{sc}{<->ssub*cmr/bx/sc}{}%
		%\DeclareFontShape{T1}{lmr}{bx}{scit}{<->ssub*cmr/bx/scsl}{}%
	}
%Warn about missing characters.
	\tracinglostchars=2
%Nicer tables: provides \toprule, \midrule, \bottomrule.
	%\usepackage{booktabs}
%For new column type X which stretches; can be used together with booktabs, see https://tex.stackexchange.com/a/97137. “tabularx modifies the widths of the columns, whereas tabular* modifies the widths of the inter-column spaces.” Loads array.
	%\usepackage{tabularx}
%math-mode version of "l" column type. Requires \usepackage{array}.
	%\usepackage{array}
	%\newcolumntype{L}{>{$}l<{$}}
%Provides \xpretocmd and loads etoolbox which provides \apptocmd, \patchcmd, \newtoggle… Also loads xparse, which provides \NewDocumentCommand and similar commands intended as replacement of \newcommand in LaTeX3 for defining commands (see https://tex.stackexchange.com/q/98152 and https://github.com/latex3/latex2e/issues/89).
	\usepackage{xpatch}
%ntheorem doc says: “empheq provides an enhanced vertical placement of the endmarks”; must be loaded before ntheorem. Loads the mathtools package, which loads and fixes some bugs in amsmath and provides \DeclarePairedDelimiter. amsmath is considered a basic, mandatory package nowadays (Grätzer, More Math Into LaTeX).
	\usepackage[ntheorem]{empheq}
%Package frenchb asks to load natbib before babel-french. Package hyperref asks to load natbib before hyperref.
	\usepackage{natbib}

\newtoggle{LCpres}
	\newtoggle{LCart}
	\newtoggle{LCposter}
	\makeatletter
	\@ifclassloaded{beamer}{
		\toggletrue{LCpres}
		\togglefalse{LCart}
		\togglefalse{LCposter}
		\wlog{Presentation mode}
	}{
		\@ifclassloaded{tikzposter}{
			\toggletrue{LCposter}
			\togglefalse{LCpres}
			\togglefalse{LCart}
			\wlog{Poster mode}
		}{
			\toggletrue{LCart}
			\togglefalse{LCpres}
			\togglefalse{LCposter}
			\wlog{Article mode}
		}
	}
	\makeatother%

%Language options ([french, english]) should be on the document level (last is main); except with tikzposter: put [french, english] options next to \usepackage{babel} to avoid warning. beamer uses the \translate command for the appendix: omitting babel results in a warning, see https://github.com/josephwright/beamer/issues/449. Babel also seems required for \refname.
	\iftoggle{LCpres}{
		\usepackage{babel}
	}{
	}
	%\frenchbsetup{AutoSpacePunctuation=false}
%listings (1.7) does not allow multi-byte encodings. listingsutf8 works around this only for characters that can be represented in a known one-byte encoding and only for \lstinputlisting. Other workarounds: use literate mechanism; or escape to LaTeX (but breaks alignment).
	%\usepackage{listings}
	%\lstset{tabsize=2, basicstyle=\ttfamily, escapechar=§, literate={é}{{\'e}}1}
%I favor acro over acronym because the former is more recently updated (2018 VS 2015 at time of writing); has a longer user manual (about 40 pages VS 6 pages if not counting the example and implementation parts); has a command for capitalization; and acronym suffers a nasty bug when ac used in section, see https://tex.stackexchange.com/q/103483 (though this might be the fault of the silence package and might be solved in more recent versions, I do not know) and from a bug when used with cleveref, see https://tex.stackexchange.com/q/71364. However, loading it makes compilation time (one pass on this template) go from 0.6 to 1.4 seconds, see https://bitbucket.org/cgnieder/acro/issues/115. Option short-format not usable in the package options as it is fragile, see https://tex.stackexchange.com/q/466882.
	\usepackage[single]{acro}
	%\acsetup{short-format = {\MakeUppercase}}
	\DeclareAcronym{AMCD}{short=amcd, long={Aide Multicritère à la Décision}}
\DeclareAcronym{AR}{short=ar, long={Argumentative Recommender}}
\DeclareAcronym{DA}{short=da, long={Decision Analysis}}
\DeclareAcronym{DJ}{short=dj, long={Deliberated Judgment}}
\DeclareAcronym{DM}{short=dm, long={Decision Maker}}
\DeclareAcronym{DP}{short=dp, long={Deliberated Preference}}
\DeclareAcronym{MAVT}{short=mavt, long={Multiple Attribute Value Theory}}
\DeclareAcronym{MCDA}{short=mcda, long={Multicriteria Decision Aid}}
\DeclareAcronym{MIP}{short=mip, long={Mixed Integer Program}}


\iftoggle{LCpres}{
	%I favor fmtcount over nth because it is loaded by datetime anyway; and fmtcount warns about possible conflicts when loaded after nth.
	\usepackage{fmtcount}
	%For nice input of date of presentation. Must be loaded after the babel package. Has possible problems with srcletter: https://golatex.de/verwendung-von-babel-und-datetime-in-scrlttr2-schlaegt-fehlt-t14779.html.
	\usepackage[nodayofweek]{datetime}
}{
}
%For presentations, Beamer implicitely uses the pdfusetitle option. ntheorem doc says to load hyperref “before the first use of \newtheorem”. autonum doc mandates option hypertexnames=false. I want to highlight links only if necessary for the reader to recognize it as a link, to reduce distraction. In presentations, this is already taken care of by beamer (https://tex.stackexchange.com/a/262014). If using colorlinks=true in a presentation, see https://tex.stackexchange.com/q/203056. Crashes the first compilation with tikzposter, just compile again and the problem disappears, see https://tex.stackexchange.com/q/254257.
\makeatletter
\iftoggle{LCpres}{
	\usepackage{hyperref}
}{
	\usepackage[hypertexnames=false, pdfusetitle, linkbordercolor={1 1 1}, citebordercolor={1 1 1}, urlbordercolor={1 1 1}]{hyperref}
	%https://tex.stackexchange.com/a/466235
	\pdfstringdefDisableCommands{%
		\let\thanks\@gobble
	}
}
\makeatother
%urlbordercolor is used both for \url and \doi, which I think shouldn’t be colored, and for \href, thus might want to color manually when required. Requires xcolor.
	\NewDocumentCommand{\hrefblue}{mm}{\textcolor{blue}{\href{#1}{#2}}}
%hyperref doc says: “Package bookmark replaces hyperref’s bookmark organization by a new algorithm (...) Therefore I recommend using this package”.
	\usepackage{bookmark}
%Need to invoke hyperref explicitly to link to line numbers: \hyperlink{lintarget:mylinelabel}{\ref*{lin:mylinelabel}}, with \ref* to disable automatic link. Also see https://tex.stackexchange.com/q/428656 for referencing lines from another document.
	%\usepackage{lineno}
	%\NewDocumentCommand{\llabel}{m}{\hypertarget{lintarget:#1}{}\linelabel{lin:#1}}
	%\setlength\linenumbersep{9mm}
%For complex authors blocks. Seems like authblk wants to be later than hyperref, but sooner than silence. See https://tex.stackexchange.com/q/475513 for the patch to hyperref pdfauthor.
	\ExplSyntaxOn
	\seq_new:N \g_oc_hrauthor_seq
	\NewDocumentCommand{\addhrauthor}{m}{
		\seq_gput_right:Nn \g_oc_hrauthor_seq { #1 }
	}
	%Should be \NewExpandableDocumentCommand, but this is not yet provided by my version of xparse
	\DeclareExpandableDocumentCommand{\hrauthor}{}{
		\seq_use:Nn \g_oc_hrauthor_seq {,~}
	}
	\ExplSyntaxOff
	{
		\catcode`#=11\relax
		\gdef\fixauthor{\xpretocmd{\author}{\addhrauthor{#2}}{}{}}%
	}
	\iftoggle{LCart}{
		\usepackage{authblk}
		\renewcommand\Affilfont{\small}
		\fixauthor
		\AtBeginDocument{
		    \hypersetup{pdfauthor={\hrauthor}}
		}
	}{
	}
%I do not use floatrow, because it requires an ugly hack for proper functioning with KOMA script (see scrhack doc). Instead, the following command centers all floats (using \centering, as the center environment adds space, http://texblog.net/latex-archive/layout/center-centering/), and I manually place my table captions above and figure captions below their contents (https://tex.stackexchange.com/a/3253).
	\makeatletter
	\g@addto@macro\@floatboxreset\centering
	\makeatother
%Permits to customize enumeration display and references
	%\nottoggle{LCpres}{
		\usepackage{enumitem} %follow list environments by a string to customize enumeration, example: \begin{description}[itemindent=8em, labelwidth=!] or \begin{enumerate}[label=({\roman*}), ref={\roman*}].
	%}{
	%}
%Provides \Cen­ter­ing, \RaggedLeft, and \RaggedRight and en­vi­ron­ments Cen­ter, FlushLeft, and FlushRight, which al­low hy­phen­ation. With tikzposter, seems to cause 1=1 to be printed in the middle of the poster.
	%\usepackage{ragged2e}
%To typeset units by closely following the “official” rules.
	%\usepackage[strict]{siunitx}
%Turns the doi provided by some bibliography styles into URLs. However, uses old-style dx.doi url (see 3.8 DOI system Proxy Server technical details, “Users may resolve DOI names that are structured to use the DOI system Proxy Server (https://doi.org (current, preferred) or earlier syntax http://dx.doi.org).”, https://www.doi.org/doi_handbook/3_Resolution.html). The patch solves this.
	\usepackage{doi}
	\makeatletter
	\patchcmd{\@doi}{http://dx.doi.org}{https://doi.org}{}{}
	\makeatother
%Makes sure upper case greek letters are italic as well.
	\usepackage{fixmath}
%Provides \mathbb; obsoletes latexsym (see http://tug.ctan.org/macros/latex/base/latexsym.dtx). Relatedly, \usepackage{eucal} to change the mathcal font and \usepackage[mathscr]{eucal} (apparently equivalent to \usepackage[mathscr]{euscript}) to supplement \mathcal with \mathscr. This last option is not very useful as both fonts are similar, and the intent of the authors of eucal was to provide a replacement to mathcal (see doc euscript). Also provides \mathfrak for supplementary letters.
	\usepackage{amsfonts}
%Provides a beautiful (IMHO) \mathscr and really different than \mathcal, for supplementary uppercase letters. But there is no bold version. Alternative: mathrsfs (more slanted), but when used with tikzposter, it warns about size substitution, see https://tex.stackexchange.com/q/495167.
	\usepackage[scr]{rsfso}
%Multiple means to produce bold math: \mathbf, \boldmath (defined to be \mathversion{bold}, see fntguide), \pmb, \boldsymbol (all legacy, from LaTeX base and AMS), \bm (the most recommended one), \mathbold from package fixmath (I don’t see its advantage over \boldsymbol).
%“The \boldsymbol command is obtained preferably by using the bm package, which provides a newer, more powerful version than the one provided by the amsmath package. Generally speaking, it is ill-advised to apply \boldsymbol to more than one symbol at a time.” — AMS Short math guide. “If no bold font appears to be available for a particular symbol, \bm will use ‘poor man’s bold’” — bm. It is “best to load the package after any packages that define new symbol fonts” – bm. bm defines \boldsymbol as synonym to \bm. \boldmath accesses the correct font if it exists; it is used by \bm when appropriate. See https://tex.stackexchange.com/a/10643 and https://github.com/latex3/latex2e/issues/71 for some difficulties with \bm.
	\usepackage{bm}
	\nottoggle{LCpres}{
	%https://ctan.org/pkg/amsmath recommends ntheorem, which supersedes amsthm, which corrects the spacing of proclamations and allows for theoremstyle. Option standard loads amssymb and latexsym. Must be loaded after amsmath (from ntheorem doc). From cleveref doc, “ntheorem is fully supported and even recommended”; says to load cleveref after ntheorem. When used with tikzposter, warns about size substitution for the lasy (latexsym) font when using \url, because ntheorem loads latexsym; relatedly (but not directly related to ntheorem), size substitution warning with the cmex font happens when loading amsmath and using \url.
		\usepackage[thmmarks, amsmath, standard, hyperref]{ntheorem}
		%empheq doc says to do this after loading ntheorem
		\usetagform{default}
	%Provides \cref. Unfortunately, cref fails when the language is French and referring to a label whose name contains a colon (https://tex.stackexchange.com/q/83798). Use \cref{sec\string:intro} to work around this. cleveref should go “laster” than hyperref.
		\usepackage[capitalise]{cleveref}
	}{
	}
	\nottoggle{LCposter}{
	%Equations get numbers iff they are referenced. Loading order should be “amsmath → hyperref → cleveref → autonum”, according to autonum doc. Use this in preference to the showonlyrefs option from mathtools, see https://tex.stackexchange.com/q/459918 and autonum doc. See https://tex.stackexchange.com/a/285953 for the etex line. Incompatible with my version of tikzposter (produces “! Improper \prevdepth”).
		\expandafter\def\csname ver@etex.sty\endcsname{3000/12/31}\let\globcount\newcount
		\usepackage{autonum}
	}{
	}
%Also loaded by tikz.
	\usepackage{xcolor}
\iftoggle{LCpres}{
	\usepackage{tikz}
	%\usetikzlibrary{babel, matrix, fit, plotmarks, calc, trees, shapes.geometric, positioning, plothandlers, arrows, shapes.multipart}
}{
}
%Vizualization, on top of TikZ
	%\usepackage{pgfplots}
	%\pgfplotsset{compat=1.14}
\usepackage{graphicx}
	\graphicspath{{graphics/}}

%Provides \print­length{length}, useful for debugging.
	%\usepackage{printlen}
	%\uselengthunit{mm}

\iftoggle{LCpres}{
	\usepackage{appendixnumberbeamer}
	%I have yet to see anyone actually use these navigation symbols; let’s disable them
	\setbeamertemplate{navigation symbols}{} 
	\usepackage{preamble/beamerthemeParisFrance}
	\setcounter{tocdepth}{10}
}{
}

%Do not use the displaymath environment: use equation. Do not use the eqnarray or eqnarray* environments: use align(*). This improves spacing. (See l2tabu or amsldoc.)


%Requires package xcolor.
\newcommand{\commentOC}[1]{\textcolor{blue}{\small$\big[$OC: #1$\big]$}}
%Requires package babel and option [french]. According to babel doc, need two braces around \selectlanguage to make the changes really local.
\newcommand{\commentOCf}[1]{\textcolor{blue}{{\small\selectlanguage{french}$\big[$OC : #1$\big]$}}}
\newcommand{\commentYM}[1]{\textcolor{red}{\small$\big[$YM: #1$\big]$}}
\newcommand{\commentYMf}[1]{\textcolor{red}{{\small\selectlanguage{french}$\big[$YM : #1$\big]$}}}

\bibliographystyle{abbrvnat}

%https://tex.stackexchange.com/a/467188 - uncomment if one of those symbols is used.
%\DeclareFontFamily{U} {MnSymbolD}{}
%\DeclareFontShape{U}{MnSymbolD}{m}{n}{
%  <-6> MnSymbolD5
%  <6-7> MnSymbolD6
%  <7-8> MnSymbolD7
%  <8-9> MnSymbolD8
%  <9-10> MnSymbolD9
%  <10-12> MnSymbolD10
%  <12-> MnSymbolD12}{}
%\DeclareFontShape{U}{MnSymbolD}{b}{n}{
%  <-6> MnSymbolD-Bold5
%  <6-7> MnSymbolD-Bold6
%  <7-8> MnSymbolD-Bold7
%  <8-9> MnSymbolD-Bold8
%  <9-10> MnSymbolD-Bold9
%  <10-12> MnSymbolD-Bold10
%  <12-> MnSymbolD-Bold12}{}
%\DeclareSymbolFont{MnSyD} {U} {MnSymbolD}{m}{n}
%\DeclareMathSymbol{\ntriplesim}{\mathrel}{MnSyD}{126}
%\DeclareMathSymbol{\nlessgtr}{\mathrel}{MnSyD}{192}
%\DeclareMathSymbol{\ngtrless}{\mathrel}{MnSyD}{193}
%\DeclareMathSymbol{\nlesseqgtr}{\mathrel}{MnSyD}{194}
%\DeclareMathSymbol{\ngtreqless}{\mathrel}{MnSyD}{195}
%\DeclareMathSymbol{\nlesseqgtrslant}{\mathrel}{MnSyD}{198}
%\DeclareMathSymbol{\ngtreqlessslant}{\mathrel}{MnSyD}{199}
%\DeclareMathSymbol{\npreccurlyeq}{\mathrel}{MnSyD}{228}
%\DeclareMathSymbol{\nsucccurlyeq}{\mathrel}{MnSyD}{229}

%03B3 Greek Small Letter Gamma
\newunicodechar{γ}{\gamma}
%03B4 Greek Small Letter Delta
\newunicodechar{δ}{\delta}
%2115 Double-Struck Capital N
\newunicodechar{ℕ}{\mathbb{N}}
%211D Double-Struck Capital R
\newunicodechar{ℝ}{\mathbb{R}}
%21CF Rightwards Double Arrow with Stroke
\newunicodechar{⇏}{\nRightarrow}
%21D2 Rightwards Double Arrow
\newunicodechar{⇒}{\ensuremath{\Rightarrow}}
%21D4 Left Right Double Arrow
\newunicodechar{⇔}{\Leftrightarrow}
%21DD Rightwards Squiggle Arrow
\newunicodechar{⇝}{\rightsquigarrow}
%2212 Minus Sign
\newunicodechar{−}{\ifmmode{-}\else\textminus\fi}
%2227 Logical And
\newunicodechar{∧}{\land}
%2228 Logical Or
\newunicodechar{∨}{\lor}
%2229 Intersection
\newunicodechar{∩}{\cap}
%222A Union
\newunicodechar{∪}{\cup}
%2260 Not Equal To (handy also as text in informal writing)
\newunicodechar{≠}{\ensuremath{\neq}}
%2264 Less-Than or Equal To
\newunicodechar{≤}{\leq}
%2265 Greater-Than or Equal To
\newunicodechar{≥}{\geq}
%2270 Neither Less-Than nor Equal To
\newunicodechar{≰}{\nleq}
%2271 Neither Greater-Than nor Equal To
\newunicodechar{≱}{\ngeq}
%2272 Less-Than or Equivalent To
\newunicodechar{≲}{\lesssim}
%2273 Greater-Than or Equivalent To
\newunicodechar{≳}{\gtrsim}
%2274 Neither Less-Than nor Equivalent To – also, from MnSymbol: \nprecsim, a more exact match to the Unicode symbol; and \npreccurlyeq, too small
\newunicodechar{≴}{\not\preccurlyeq}
%2275 Neither Greater-Than nor Equivalent To
\newunicodechar{≵}{\not\succcurlyeq}
%2279 Neither Greater-Than nor Less-Than – requires MnSymbol; also \nlessgtr from txfonts/pxfonts, \ngtreqless from MnSymbol (but much higher), \ngtrless from MnSymbol (a more exact match to the Unicode symbol); for incomparability (not matching this Unicode symbol), may also consider \ntriplesim from MnSymbol,\nparallelslant from fourier, \between from mathabx, or ⋈
\newunicodechar{≹}{\ngtreqlessslant}
%227A Precedes
\newunicodechar{≺}{\prec}
%227B Succeeds
\newunicodechar{≻}{\succ}
%227C Precedes or Equal To
\newunicodechar{≼}{\preccurlyeq}
%227D Succeeds or Equal To
\newunicodechar{≽}{\succcurlyeq}
%227E Precedes or Equivalent To
\newunicodechar{≾}{\precsim}
%227F Succeeds or Equivalent To
\newunicodechar{≿}{\succsim}
%2280 Does Not Precede
\newunicodechar{⊀}{\nprec}
%2281 Does Not Succeed
\newunicodechar{⊁}{\nsucc}
%22B2 Normal Subgroup Of – \triangleleft is too small compared to \trianglelefteq and the like
\newunicodechar{⊲}{\lhd}
%22B3 Contains as Normal Subgroup
\newunicodechar{⊳}{\rhd}
%22B4 Normal Subgroup of or Equal To
\newunicodechar{⊴}{\trianglelefteq}
%22B5 Contains as Normal Subgroup or Equal To
\newunicodechar{⊵}{\trianglerighteq}
%22C8 Bowtie
\newunicodechar{⋈}{\bowtie}
%22EA Not Normal Subgroup Of
\newunicodechar{⋪}{\ntriangleleft}
%22EB Does Not Contain As Normal Subgroup
\newunicodechar{⋫}{\ntriangleright}
%22EC Not Normal Subgroup of or Equal To
\newunicodechar{⋬}{\ntrianglelefteq}
%22ED Does Not Contain as Normal Subgroup or Equal
\newunicodechar{⋭}{\ntrianglerighteq}
%25A1 White Square
\newunicodechar{□}{\Box}
%27E6 Mathematical Left White Square Bracket – there’s also \llbracket from stmaryrd
\newunicodechar{⟦}{\text{\textlbrackdbl}}
%27E7 Mathematical Right White Square Bracket – there’s also \rrbracket from stmaryrd
\newunicodechar{⟧}{\text{\textrbrackdbl}}
%27FC Long Rightwards Arrow from Bar
\newunicodechar{⟼}{\longmapsto}
%2AB0 Succeeds Above Single-Line Equals Sign
\newunicodechar{⪰}{\succeq}
%301A Left White Square Bracket
\newunicodechar{〚}{\textlbrackdbl}
%301B Right White Square Bracket
\newunicodechar{〛}{\textrbrackdbl}
%→ is defined by default as \textrightarrow, which is invalid in math mode. Same thing for the three other commands. I redefine those four using \DeclareUnicodeCharacter instead of \newunicodechar because the latter warns about the previous definition.
%→ Rightwards Arrow
\DeclareUnicodeCharacter{2192}{\ifmmode\rightarrow\else\textrightarrow\fi}
%¬ Not Sign
\DeclareUnicodeCharacter{00AC}{\ifmmode\lnot\else\textlnot\fi}
%… Horizontal Ellipsis
\DeclareUnicodeCharacter{2026}{\ifmmode\dots\else\textellipsis\fi}
%× Multiplication Sign
\DeclareUnicodeCharacter{00D7}{\ifmmode\times\else\texttimes\fi}


\NewDocumentCommand{\R}{}{ℝ}
\NewDocumentCommand{\N}{}{ℕ}
%\mathscr is rounder than \mathcal.
\NewDocumentCommand{\powerset}{m}{\mathscr{P}(#1)}
%Powerset without zero.
\NewDocumentCommand{\powersetz}{m}{\mathscr{P}^*(#1)}
%https://tex.stackexchange.com/a/45732, works within both \set and \set*, same spacing than \mid (https://tex.stackexchange.com/a/52905).
\NewDocumentCommand{\suchthat}{}{\;\ifnum\currentgrouptype=16 \middle\fi|\;}
%Integer interval.
\NewDocumentCommand{\intvl}{m}{⟦#1⟧}
%Allows for \abs and \abs*, which resizes the delimiters.
\DeclarePairedDelimiter\abs{\lvert}{\rvert}
\DeclarePairedDelimiter\card{\lvert}{\rvert}
%Perhaps should use U+2016 ‖ DOUBLE VERTICAL LINE here?
\DeclarePairedDelimiter\norm{\lVert}{\rVert}
%Better than using the package braket because braket introduces possibly undesirable space. Then: \begin{equation}\set*{x \in \R^2 \suchthat \norm{x}<5}\end{equation}.
\DeclarePairedDelimiter\set{\{}{\}}
\DeclarePairedDelimiter\ceil{\lceil}{\rceil}
\DeclarePairedDelimiter\floor{\lfloor}{\rfloor}
\DeclareMathOperator*{\argmax}{arg\,max}
\DeclareMathOperator*{\argmin}{arg\,min}

%We want the straight form of \phi for mathematics, as recommended in UTR #25: Unicode support for mathematics, and thus use \phi for the mathematical symbol and not \varphi; and similarly \epsilon is preferred to \varepsilon for the mathematical symbol.

%The amssymb solution.
%\NewDocumentCommand{\restr}{mm}{{#1}_{\restriction #2}}
%Another acceptable solution.
%\NewDocumentCommand{\restr}{mm}{{#1|}_{#2}}
%https://tex.stackexchange.com/a/278631; drawback being that sometimes the text collides with the line below.
\NewDocumentCommand\restr{mm}{#1\raisebox{-.5ex}{$|$}_{#2}}


%Decision Theory (MCDA and SC)
\NewDocumentCommand{\allalts}{}{A}
\NewDocumentCommand{\allcrits}{}{\mathscr{C}}
\NewDocumentCommand{\alts}{}{A}
\NewDocumentCommand{\dm}{}{i}
\NewDocumentCommand{\allF}{}{\mathscr{F}}
\NewDocumentCommand{\allvoters}{}{\mathscr{N}}
\NewDocumentCommand{\voters}{}{N}
\NewDocumentCommand{\allprofs}{}{\boldsymbol{\mathcal{R}}}
\NewDocumentCommand{\prof}{}{P}
\NewDocumentCommand{\ibar}{}{\overline{i}}
\NewDocumentCommand{\lprof}{}{\lambda_P}
\NewDocumentCommand{\lprofi}{O{x}}{\lambda_P(#1)_i}
\NewDocumentCommand{\lprofibar}{O{x}}{\lambda_P(#1)_{\overline{i}}}
\NewDocumentCommand{\ineq}{}{(\sigma \circ \lambda_P)}

\NewDocumentCommand{\linors}{}{\mathcal{L}(\allalts)}
%Thanks to https://tex.stackexchange.com/q/154549
	%\makeatletter
	%\def\@myRgood@#1#2{\mathrel{R^X_{#2}}}
	%\def\myRgood{\@ifnextchar_{\@myRgood@}{\mathrel{R^X}}}
	%\makeatother
\NewDocumentCommand{\pref}{}{\succ}
\NewDocumentCommand{\prefi}{O{i}}{\succ_{#1}}
\NewDocumentCommand{\paretopt}{}{\text{PO}}
\NewDocumentCommand{\SPPd}{}{\Sigma^\text{PPd}}
\NewDocumentCommand{\SAll}{}{\Sigma^\text{All}}
\NewDocumentCommand{\SThreshold}{}{\Sigma_\text{threshold}}
\NewDocumentCommand{\vpr}{}{\boldsymbol{v}}

\NewDocumentCommand{\musigma}{O{\sigma}O{P}}{\argmin_{A}({#1}\circ\lambda_{{#2}})}
\NewDocumentCommand{\mustar}{O{\sigma}O{P}}{\argmin_{\paretopt({#2})} ({#1} \circ \lambda_{#2})}
\NewDocumentCommand{\minineq}{O{\allalts}}{\argmin_{#1}(\sigma \circ \lambda)}
\NewDocumentCommand{\FBP}{}{\text{FB}(P)}
\NewDocumentCommand{\POP}{}{\text{PO}(P)}

\NewDocumentCommand{\alllosses}{}{\intvl{0, m-1}^N}

\NewDocumentCommand{\Ptop}{}{\bar{P}}
\NewDocumentCommand{\sigmatop}{}{\bar{\sigma}}

\NewDocumentCommand{\fltwo}{}{\floor{\bar{l_2}}}
\NewDocumentCommand{\bltwo}{}{\bar{l_2}}

\newtheorem{conjecture}{Conjecture}

%\newcommand{\tikzmark}[1]{%
	\tikz[overlay, remember picture, baseline=(#1.base)] \node (#1) {};%
}

\newlength{\GraphsDNodeSep}
\setlength{\GraphsDNodeSep}{7mm}
\tikzset{/GraphsD/dot/.style={
	shape=circle, fill=black, inner sep=0, minimum size=1mm
}}

% MCDA Drawing Sorting
\newlength{\MCDSCatHeight}
\setlength{\MCDSCatHeight}{6mm}
\newlength{\MCDSAltHeight}
\setlength{\MCDSAltHeight}{4mm}
%separation between two vertical alts
\newlength{\MCDSAltSep}
\setlength{\MCDSAltSep}{2mm}
\newlength{\MCDSCatWidth}
\setlength{\MCDSCatWidth}{3cm}
\newlength{\MCDSAltWidth}
\setlength{\MCDSAltWidth}{2.5cm}
\newlength{\MCDSEvalRowHeight}
\setlength{\MCDSEvalRowHeight}{6mm}
\newlength{\MCDSAltsToCatsSep}
\setlength{\MCDSAltsToCatsSep}{1.5cm}
\newcounter{MCDSNbAlts}
\newcounter{MCDSNbCats}
\newlength{\MCDSArrowDownOffset}
\setlength{\MCDSArrowDownOffset}{0mm}
\tikzset{/MCD/S/alt/.style={
	shape=rectangle, draw=black, inner sep=0, minimum height=\MCDSAltHeight, minimum width=\MCDSAltWidth
}}
\tikzset{/MCD/S/pref/.style={
	shape=ellipse, draw=gray, thick
}}
\tikzset{/MCD/S/cat/.style={
	shape=rectangle, draw=black, inner sep=0, minimum height=\MCDSCatHeight, minimum width=\MCDSCatWidth
}}
\tikzset{/MCD/S/evals matrix/.style={
	matrix, row sep=-\pgflinewidth, column sep=-\pgflinewidth, nodes={shape=rectangle, draw=black, inner sep=0mm, text depth=0.5ex, text height=1em, minimum height=\MCDSEvalRowHeight, minimum width=12mm}, nodes in empty cells, matrix of nodes, inner sep=0mm, outer sep=0mm, row 1/.style={nodes={draw=none, minimum height=0em, text height=, inner ysep=1mm}}
}}

%Git
\newlength{\GitDCommitSep}
\setlength{\GitDCommitSep}{13mm}
\tikzset{/GitD/commit/.style={
	shape=rectangle, draw, minimum width=4em, minimum height=0.6cm
}}
\tikzset{/GitD/branch/.style={
	shape=ellipse, draw, red
}}
\tikzset{/GitD/head/.style={
	shape=ellipse, draw, fill=yellow
}}

%Social Choice
\tikzset{/SCD/profile matrix/.style={
	matrix of math nodes, column sep=3mm, row sep=2mm, nodes={inner sep=0.5mm, anchor=base}
}}
\tikzset{/SCD/rank-profile matrix/.style={
	matrix of math nodes, column sep=3mm, row sep=2mm, nodes={anchor=base}, column 1/.style={nodes={inner sep=0.5mm}}, row 1/.style={nodes={inner sep=0.5mm}}
}}
\tikzset{/SCD/rank-vector/.style={
	draw, rectangle, inner sep=0, outer sep=1mm
}}
\tikzset{/SCD/isolated rank-vector/.style={
	draw, matrix of math nodes, column sep=3mm, inner sep=0, matrix anchor=base, nodes={anchor=base, inner sep=.33em}, ampersand replacement=\&
}}

% GUI
\tikzset{/GUID/button/.style={
	rectangle, very thick, rounded corners, draw=black, fill=black!40%, top color=black!70, bottom color=white
}}

% Logger objects
\tikzset{/loggerD/main/.style={
	shape=rectangle, draw=black, inner sep=1ex, minimum height=7mm
}}
\tikzset{/loggerD/helper/.style={
	shape=rectangle, draw=black, dashed, minimum height=7mm
}}
\tikzset{/loggerD/helper line/.style={
	<->, draw, dotted
}}

% Beliefs
\tikzset{/BeliefsD/attacker/.style={
	shape=rectangle, draw, minimum size=8mm
}}
\tikzset{/BeliefsD/supporter/.style={
	shape=circle, draw
}}


%\DeclareAcronym{AMCD}{short=amcd, long={Aide Multicritère à la Décision}}
\DeclareAcronym{AR}{short=ar, long={Argumentative Recommender}}
\DeclareAcronym{DA}{short=da, long={Decision Analysis}}
\DeclareAcronym{DJ}{short=dj, long={Deliberated Judgment}}
\DeclareAcronym{DM}{short=dm, long={Decision Maker}}
\DeclareAcronym{DP}{short=dp, long={Deliberated Preference}}
\DeclareAcronym{MAVT}{short=mavt, long={Multiple Attribute Value Theory}}
\DeclareAcronym{MCDA}{short=mcda, long={Multicriteria Decision Aid}}
\DeclareAcronym{MIP}{short=mip, long={Mixed Integer Program}}


%\input{preamble/refAPIcmds}

%I find these settings useful in draft mode. Should be removed for final versions.
	%Which line breaks are chosen: accept worse lines, therefore reducing risk of overfull lines. Default = 200.
		\tolerance=2000
	%Accept overfull hbox up to...
		\hfuzz=2cm
	%Reduces verbosity about the bad line breaks.
		\hbadness 5000
	%Reduces verbosity about the underful vboxes.
		\vbadness=1300

%\title{Title \thanks{Thanks.}}
%\author{Name1}
%\author{Name2}
%\affil{Université Paris-Dauphine, PSL Research University, CNRS, LAMSADE, 75016 PARIS, FRANCE\\
%	\href{mailto:olivier.cailloux@dauphine.fr}{olivier.cailloux@dauphine.fr}
%}
%\author{Name3}
%\affil{Affil2}
%\hypersetup{
%	pdfsubject={},
%	pdfkeywords={},
%}

\setcounter{MaxMatrixCols}{10}
%TCIDATA{OutputFilter=LATEX.DLL}
%TCIDATA{Version=5.50.0.2953}
%TCIDATA{<META NAME="SaveForMode" CONTENT="1">}
%TCIDATA{BibliographyScheme=Manual}
%TCIDATA{Created=Tuesday, August 09, 2005 15:34:58}
%TCIDATA{LastRevised=Thursday, March 07, 2019 13:36:58}
%TCIDATA{<META NAME="GraphicsSave" CONTENT="32">}
%TCIDATA{<META NAME="DocumentShell" CONTENT="Scientific Notebook\Blank with Theorem Tags">}
%TCIDATA{Language=American English}
%TCIDATA{CSTFile=Math.cst}
%TCIDATA{PageSetup=14,14,57,57,0}
%TCIDATA{AllPages=
%H=36
%F=36,\PARA{038<p type="texpara" tag="Body Text" >\hfill \thepage}
%}

%\newtheorem{theorem}{Theorem}[section]
\newtheorem{acknowledgement}[theorem]{Acknowledgement}
\newtheorem{algorithm}[theorem]{Algorithm}
\newtheorem{axiom}{Axiom}[section]
\newtheorem{case}{Case}[section]
\newtheorem{claim}{Claim}[section]
\newtheorem{conclusion}{Conclusion}[section]
\newtheorem{condition}{Condition}[section]
\newtheorem{conjecture}{Conjecture}[section]
%\newtheorem{corollary}{Corallary}[section]
\newtheorem{criterion}{Criterion}[section]
%\newtheorem{definition}{Definition}[section]
%\newtheorem{example}{Example}[section]
\newtheorem{exercise}{Exercise}[section]
%\newtheorem{lemma}{Lemma}[section]
\newtheorem{notation}{Notation}[section]
\newtheorem{problem}{Problem}[section]
%\newtheorem{proposition}{Proposition}[section]
%\newtheorem{remark}{Remark}[section]
\newtheorem{solution}{Solution}[section]
\newtheorem{summary}{Summary}[section]
%\newenvironment{proof}[1][Proof]{\noindent\textbf{#1.} }{\ \rule{0.5em}{0.5em}}

\usepackage{blkarray}
\usepackage{booktabs}

\title{Ex-Ante versus Ex-Post \\ Compromise}
\author{}
\date{}

\newcommand{\commentBN}[1]{\textcolor{magenta}{\small$\big[$BN: #1$\big]$}}
\newcommand{\commentRS}[1]{\textcolor{red}{\small$\big[$RS: #1$\big]$}}
\newcommand{\paretopt}{\mathit{PO}}
\newcommand{\SPPd}{\Sigma^\text{PPd}}
\newcommand{\SAll}{\Sigma^\text{All}}
\newcommand{\SThreshold}{\Sigma_\text{threshold}}
\newcommand{\vpr}{v^\text{pr}}

\begin{document}

\maketitle
\thispagestyle{empty}

\begin{abstract}
	A classical social choice setting is composed of a group of individuals, or voters, that express their preferences over a set of alternatives. The social choice problem consists in defining a procedure able to determine a collective choice for this group of voters, starting from their individual preferences. Such procedure is called social choice rule and it can be defined as a function mapping preference profiles to alternatives. Depending on the properties that this function satisfies, very different outcomes can be produced starting from the same initial profile. The plurality rule is one of the most common social choice rule and it consists in selecting, as a winner, the alternative that is considered the best by the largest number of voters forming the society. Yet, this rule can pick, as a winner, an alternative that is considered the worst by a strict majority of voters. Such outcome may be undesirable. Several procedures, the so-called compromise rules, have been proposed in the literature that aim to find a compromise. Nevertheless, all those rules can be defined as \textit{ex-ante compromises} or \textit{procedural compromises}, i.e., they impose over individuals a willingness to compromise but they do not ensure an outcome where everyone has effectively compromised. In this work, we approach the problem of compromise from an \textit{ex-post} perspective, favoring an outcome where every voter gives up her most preferred positions if this increases equality. We propose a new notion of compromise in the social choice context, considering both cardinal and ordinal utilities. 
\end{abstract}

\pagebreak

\section{Introduction and Related Works}

In a social choice situation where individuals have preferences over several alternatives there is no unique way to define a procedure that select a common agreement between them. One of the first problems the decision makers have to face is whether to prefer the quantity of support for an alternative (i.e. the number of voters who favor it) to the quality of its support (i.e. until which level in voters’ preferences is legitimate to descend). This decision leads to various outcomes satisfying different properties. Preferring the first to the latter may result in a lack of a majority support or, worst, into a strict majority of individuals which strongly dislikes the chosen candidates. Vice versa, favoring the number of voters behind an alternative implies descending levels in individual preferences potentially ending on electing a generally disliked candidate because of the opposition of very few individuals. 

Using the plurality rule, for example, the alternatives which are ranked the best by the largest number of voters are picked as outcome. The problem of this procedure is that, as long as there are more than two alternatives, the outcome may lack the support of any majority. In fact, it may also be opposed by a strict majority of agents each of whom ranks the chosen candidate as their least preferred choice. In contrast to the plurality rule, the Majoritarian Compromise \citep{Sertel1999} unambiguously selects candidates who have the support of a majority in the best degree possible. So the trade-off between the quality and quantity of support behind alternatives is always questionable in social choice settings. Several other procedures have been proposed in the literature that aim to find a compromise and \citet{Merlin2019} gather them in the class of Compromise Rules.

In the case where the candidates are only two we talk about bargaining \citep{Thomson1994}. The bargaining problem was first axiomatized by \citet{Nash1950}, who also proposed a solution for the problem that is obtained by maximizing the product of the differences between the utility from an agreement and the one from a disagreement (namely the status quo). This is the only bargaining solution that satisfies all the following axioms: independence of equivalent utility transformations, independence of irrelevant alternatives, pareto optimality and symmetry.
	
\section{Posing the problem}
Consider a finite set $N$ of individuals with $\#N=n\geq 2$ and a finite set $A$ of alternatives with $\#A=m\geq 2$. We write $L(A)$ for the set of linear orders over $A$ and $P_{i}\in L(A)$ stands for the preference of $i\in N$. For each $k\in \left\{ 1,...,\text{ }m\right\} $, $r_{k}(P_{i})$ is the alternative ranked at level $k$ by $P_{i}$. An (ordinal) social choice rule (SCR) is a mapping $f:L(A)^{N}\rightarrow 2^{A} \setminus \{\emptyset \}$.

We call BK-compromises the class of $q$-approval fallback bargaining rules introduced by \citet{Brams2001}: the $q$-approval fallback bargaining rule picks ranks 1, 2, …, and stops as soon as at least $q$ voters rank some alternative at the chosen rank or better. All such alternatives form the compromise set, and among them the ones which receive the highest support are elected.
We start by two examples based on an observation made by Laslier at Buyukada. 
\begin{example}
	Consider the following preference profile $P\in L(A)^{N}$ with $n=100$:
	\begin{center}
		$
		\begin{array}{cccc}
		\mathbf{51} \quad &a&b&c\\
		\mathbf{49} \quad &c&b&a\\
		\end{array}
		$
	\end{center}
	which represents 51 individuals who prefer $a$ to $b$, $b$ to $c$, hence $a$ to $c$; and 49 individuals who prefer $c$ to $b$, $b$ to $a$, hence $c$ to $a $. At $P,$ when $q\in \left\{ 1,..., \frac{n}{2} +1\right\} $, all BK-compromises pick $a$, which does not appear as a compromise, as 51 voters reach their best alternative while the remaining 49 voters have to be contented with their worst one. Note that for $q\in \left\{ 1,..., \frac{n}{2} -1 \right\} $ the set of possible common agreements determined by the fallback bargaining procedure is $\{a,c\}$. Nevertheless, $a$ receives the highest support thus it is elected.
\end{example}

This point can be extended to greater values of $q$, as we can see in the next example.

\begin{example}
	\label{ex:exk}
	Consider again $n=100$, but now with 4 alternatives, and the preference profile
	\begin{center}
		$
		\begin{array}{ccccc}
		\mathbf{26} \quad &a&b&c&d\\
		\mathbf{25} \quad &c&b&a&d\\
		\mathbf{z-51} \quad &d&b&a&c\\
		\mathbf{100-z} \quad &d&a&c&b\\
		\end{array}
		$
	\end{center}
where $z\in \left\{ 64,..., 99\right\}$. For any $q\in \left\{ \lfloor \frac{n}{2}\rfloor,..., z\right\}$ all the BK-compromises pick $b$ as winner. We can observe that the outcome reflects a compromise among the first three blocks of voters that disagree on their top choices. The fourth block of $100-z$ voters, however, does not seem to be taken into account by any of the fallback bargaining rules. 
Let's consider $z=64$, the profile would then be the following: 

\begin{center}
	$
	\begin{array}{ccccc}
	\mathbf{26} \quad &a&b&c&d\\
	\mathbf{25} \quad &c&b&a&d\\
	\mathbf{13} \quad &d&b&a&c\\
	\mathbf{36} \quad &d&a&c&b\\
	\end{array}
	$
\end{center}

If we consider only the first level there are $26$ individuals voting for $a$, $25$ voting for $c$, $49$ voting for $d$, and none voting for $b$. No alternative reaches the quota, which is $q=64$, therefore we descend one level. Considering now the first two levels $a$ gets $62$ votes, $b$ gets $64$ votes, $c$ gets $25$ votes and $d$ gets $49$ votes. The social choice function would then select $b$, even though there is a big group of $36$ voters for which this outcome corresponds to the least preferred alternative.
Let's consider the other extreme, $z=99$, the profile would then be the following: 

\begin{center}
	$
	\begin{array}{ccccc}
	\mathbf{26} \quad &a&b&c&d\\
	\mathbf{25} \quad &c&b&a&d\\
	\mathbf{48} \quad &d&b&a&c\\
	\mathbf{1} \quad &d&a&c&b\\
	\end{array}
	$
\end{center} 

The reasoning follows the same steps than the previous case, working now for any $q\in \left\{ 64,..., 99 \right\}$. Even though only one person now ranks $b$ at the last position, the example does not lose his strength. Indeed, if it may seem reasonable to prefer the willing of the $99\%$ of the population, the result may nevertheless be considered as not satisfying the concept of a compromise.
\end{example}

These two examples illustrate that BK-compromises (except fallback bargaining (FB) where $q=n$ which will be further discussed in Example \ref{ex:qeqn}) are “ex-ante compromises” or “procedural compromises”, i.e., they impose over individuals a willingness to compromise but they don't ensure an outcome where everyone has effectively compromised. So what is a “true compromise”? To define it, we switch to an utilitarian world.

\section{Cardinal compromises}
Let $U(A) \subseteq \R^A$ be the set of injective real-valued “utility” functions defined over $A$, thus mandating, $\forall u_{i}\in U(A)$: $u_{i}(x)\neq u_{i}(y)$ $\forall x ≠ y\in A$. Note that each $u_{i}\in U$ induces a unique $P(u_{i})\in L(A)$.
For each $k\in \left\{ 1, ..., m\right\} $, we write $r_{k}(u_{i})$ for $r_{k}(P(u_{i}))$, thus $r_k \in A$.

A cardinal SCR is a mapping $f:U(A)^{N}\rightarrow 2^{A} \setminus \{\emptyset \}$. We will not specify whether a SCR is cardinal or ordinal whenever this is clear from its domain.

We adopt the “equal loss principle” which Chun (1988, Economics Letters) \cite{Chun1988}, Chun and Peters (1991, MASS) \cite{Chun1991} introduce and characterize for bargaining problems \commentOC{Which one do we want to cite as introducing and characterizing?}. It seems that the concept is also used in bankruptcy and matching problems. For each $u\in U(A)^{N}$, each $x\in A$ and each $i\in N$, let $\lambda_{i}^u(x) = \max_{a \in A} u_i(a) - u_{i}(x)$ represent the loss of utility for $i$ that $x$ be elected instead of her favorite alternative, and $\lambda ^{u}(x)=(\lambda _{i}^{u}(x))_{i\in N}$, thus $\lambda ^{u}(x) \in \R_+^N$ representing the vector of losses if electing $x$.

Unlike bargaining, our problem does not induce a convex space of achievable utilities. As a result, we cannot ensure the existence of $x\in A$ with $\lambda _{i}^{u}(x)=\lambda _{j}^{u}(x)$ for all $i,j\in N$. Hence, we aim to determine the “most equally distributed” vector within $\left\{ \lambda ^{u}(x)\right\} _{x\in A}$. We thus need to adopt a spread measure which is a mapping $\sigma : \R_{+}^{N}\longrightarrow \R_{+}$. In what follows, we will consider various classes of possible spread measures. One of the conditions we consider is the following Pairwise Pareto dominance condition.

\begin{definition}[Pairwise Pareto dominance]
	For all $r$, $s\in \R_{+}^{N}$: 
	\[\left[\left\vert r_{i}-r_{j}\right\vert \leq \left\vert s_{i}-s_{j}\right\vert \forall i, j\in N\right] ⇒ \sigma (r)\leq \sigma (s).\] 
\end{definition}
We write $\SAll = R_+^{R_+^N}$ for the set of all spread measures. We write $\SPPd \subseteq \SAll$ for the class of spread measures that satisfy PPd. Any $\sigma \in \SPPd$ reaches a minimum value for an input whose components are all equal.

\textbf{TODO: }
\begin{itemize}
	\item Give reference to the relevant literature, also to the Pareto dominance condition.
	\item Give examples of spread measure that do and do not satisfy the condition.
	\item Give some examples using average or standard deviation.
\end{itemize}

\commentOC{BEA: please check from here downwards which of our definitions (and results?) make sense for any $\sigma \in \SAll$ and which ones should be restricted to $\sigma \in \SPPd$ as in the original text.}

Given any $\sigma \in \SPPd$, we define a (cardinal) compromise as the cardinal SCR $C^{\sigma }:U(A)^{N}\rightarrow 2^{A} \setminus \{\emptyset \}$ such that $C^{\sigma }(u) = \argmin_{x \in A} (\sigma \circ \lambda^u)(x) = \left\{ x\in A:\sigma (\lambda ^{u}(x))\leq \sigma (\lambda ^{u}(y)), \forall y\in A\right\} $. Let $\paretopt(u)$ be the set of alternatives which are Pareto optimal at $u\in U(A)^{N}$. As a matter of fact, $C^{\sigma }(u)$ can pick alternatives outside $\paretopt(u)$. In fact, $C^{\sigma }(u)$ $\cap $ $\paretopt(u)$ can even be empty. To see this, with two voters and five alternatives, consider $u$ with $u_{1}(a)=5$, $u_{1}(b)=4$, $u_{1}(c)=3$, $u_{1}(d)=2$, $u_{1}(e)=1$, $u_{2}(d)=5$, $u_{2}(c)=4$, $u_{2}(b)=3$, $u_{2}(a)=2$, $u_{2}(e)=1$. Note that for any $\sigma \in \Sigma $, $C^{\sigma }(u)=\left\{ e\right\} $ while $\paretopt(u)=\left\{ a, b, c, d\right\} $. 

\begin{remark}
	Given the spirit of fairness that underlies $C^{\sigma }$, picking $e$ at $u$ may have a merit. To see why, consider the following situation, where a social planner must choose between a world $x$ where individuals may sell their organs, and a world $y$ where they may not (assuming that they do not if they may not). Assume that the utility of individual $1$ would be slightly higher in world $x$, because she would sell an organ to individual $2$, who would gain a lot from this sell: $u_1(x) = 1, u_1(y) = 0, u_2(x) = 100, u_2(y) = 0$. Even though $y$ is Pareto dominated in this example, the social planner might prefer $y$ to $x$ by considering that in $x$, an individual benefits in an unfair way of the sale. (A more thorough argument could observe that in real life, the set of alternatives is not necessarily determined once and for all, and that by adopting a strategy that tolerates huge inequalities, the social planner creates an incentive for $u_2$ to avoid helping adding new alternatives in the agenda, such as one where $u_1$ does not desire to sell an organ to $u_2$.) \commentOC{TODO refer to literature. Perhaps related to Sen - Rationality and Freedom (2002), chapter 12: Liberty and Social Choice.} This example illustrates the tension between Pareto efficiency, which may come with a price of huge inequalities as in this example, and pure equalitarian view, which as the previous example illustrates may completely disregard Pareto efficiency even when the cost in supplemental inequality is low. Exploring rules that trade-off between these two objectives might be interesting, but is left as a future goal.
\end{remark}
From now on, we will ensure the Pareto optimality of cardinal compromises by defining them as 
\[C^{\sigma }(u) = \argmin_{x \in \paretopt(u)} (\sigma \circ \lambda^u)(x) = \left\{ x\in \paretopt(u):\sigma (\lambda ^{u}(x))\leq \sigma (\lambda ^{u}(y)),  \forall y\in A\right\}.\]

OPEN QUESTION: Which axioms would characterize $C^{\sigma }$?

\section{Ordinal compromises} 
We now address the question of relating $C^{\sigma }$ to ordinal SCRs.

A utility assignment (UA) is a strictly decreasing function $v:\left\{ 1,..., m\right\} \rightarrow \R$ which maps ranks to utilities. For each $P_{i}\in L(A)$, $v$ induces a utility function $v_{P_{i}}$ $\in U(A)$ with $v_{P_{i}}(x)=v(k)$ for each $x=r_{k}(P_{i})\in A$. So for every $P\in L(A)^{N}$, $v_{P}=(v_{P_{1}},...,v_{P_{n}})$ $\in U(A)^{N}$ is the utility profile that $v$ induces from the preference profile $P$. Given $v$, we can define $\vpr: L(A)^N → U(A)^N$ as the function which determines a utility profile by mapping each ordinal preference from a profile to its corresponding utility function. \commentOC{Perhaps write $\vpr$ instead of $v$ where appropriate, to clarify?}

Given any UA $v$, an ordinal compromise is the SCR $C^{\sigma }\circ \vpr:L(A)^{N}\rightarrow 2^{A} \setminus \{\emptyset \}$ such that $(C^{\sigma }\circ \vpr)(\{P_i, i \in N\}) = C^{\sigma}(\{v_{P_i}, i \in N\})$ for every $P\in L(A)^{N}$.

\subsection{Interaction between utility assignment and spread measure}
Once a class of functions $\Sigma \subseteq \SAll$ is defined, an ordinal compromise is defined by two parameters: the UA $v$ and the spread measure $\sigma \in \Sigma$. Here we investigate some links between these two parameters.

First, we observe that in some cases, these two parameters can be treated as the choice of a single parameter, in the following sense.
\begin{definition}
	A class of spread measure $\Sigma \subseteq \SAll$ is UA-independent iff, given any $\sigma \in \Sigma $ and any two UAs $v$ and $v^{\prime }$, there exists a $\sigma^\prime\in \Sigma $ such that $C^{\sigma}\circ \vpr = C^{\sigma ^{\prime}} \circ v^{\prime\text{pr}}$.
\end{definition}

As an example of such $\Sigma$, consider $\SThreshold$, the class of spread measures that consider the number of time the loss is greater than or equal to a certain threshold $k \in \R$:
\[\Sigma_\text{threshold}= \set{ \sigma^k , k \in \R},\]
where $\sigma^k (\lambda)= \# \set{ i \in N \suchthat \lambda_i \geq k}$.

\begin{proposition} 
\label{prop:equivalence_SigmaThrs} 
$\SThreshold$ is UA-independent.
\end{proposition}
\begin{proof}
Given any utility assignments $v$ and $v^{\prime}$, and $\sigma^k$, we need to show that it is possible to choose $k^\prime$ such that $C^{\sigma^k} \circ \vpr = C^{\sigma^{k'}} \circ v^{\prime \text{pr}}$. The threshold $k$ is a real positive number \commentOC{If you want it to be positive, you need to state it in the definition above. But I see no need for this. I’d remove the word “positive” from this sentence.}, without loss of generality we can say it is greater than the $i_{th}$ biggest possible loss in $v$ and smaller than or equal to the $(i+1)_{th}$ \commentOC{What about $k$ smaller than all losses? I agree it is easily treated as a separate case, but this is not very elegant, and it is not very nice to let the reader realize this for herself (and the phrase “wlog” does not apply). Your proof can (I think) easily cover that case as well, using the same construction. Say that you define $i \in \intvl{1, m}$ ($\intvl{1, m}$ denoting an interval of integers) as the position such that, etc., and pick $m$ if $k$ is “small enough” (to be defined).}. Consider now $\sigma^{k^\prime}$ such that $k^\prime$ is a number greater than the $i_{th}$ biggest possible loss considering now $v^\prime$ and smaller than or equal to the $(i+1)_{th}$. Please note that $k$ denotes a position $p$ in the rank \commentOC{Are you using the letters $p$ and $k$ for exactly the same thing?}, the idea is \commentOC{“the idea is” is informal} to select the alternative that minimize the number of times it is ranked after the position $p$. Since $v$ and $v^{\prime}$ are strictly monotonic, choosing a $k^\prime$ in $v^\prime$ that corresponds to the same loss of $k$ in $v$ allows to consider the same position $p$ in the rank. For each alternative $a$, the number of times $a$ is ranked after the position $p$ does not change, thus the values of $\sigma^k (\lambda(a))$ are the same \commentOC{than what?}. This concludes the proof.  
\begin{example}
	Let's take the following profile $P$ as an example:
	\begin{center}
		$
		\begin{array}{cccc}
		\mathbf{i_1} \quad &a&b&c\\
		\mathbf{i_2} \quad &b&c&a\\
		\end{array}
		$
	\end{center}
	and consider the utility assignments $v$ such that $v_{P_1}=(3,1,0) , v_{P_2}=(0,3,1)$ and $v^\prime$ such that $v^\prime_{P_1}=(10,9,0) , v^\prime_{P_2}=(0,10,9)$. Define by $\lambda^P(x)$ the loss of utility of the alternative $x$ in the profile P under the utility assignment $v$, and $\lambda^{\prime P}(x)$ the one under the UA $v^\prime$. The possible values of $\lambda^P(x)$ are $(0,2,3)$ and the ones of $\lambda^{\prime P}(x)$ are $(0,1,10)$. Let's consider $\sigma^2$; $k=2$ is greater than the $1_{st}$ biggest lost and smaller than or equal to the $2_{nd}$. We can construct $\sigma^{k^\prime}$ by choosing a $k^\prime$ greater than the $1_{st}$ biggest lost considering $v^\prime$ and smaller than or equal to the $2_{nd}$ one. Let's \commentOC{Let’s is informal} take $k^\prime=1$. For $v$:
	\begin{center}
		$
		\begin{array}{ccccc}
		\lambda(a) \ = \ &0&3 & \rightarrow & \sigma^k(\lambda(a)) = 1\\
		\lambda(b) \ = \ &2&0 & \rightarrow & \sigma^k(\lambda(b)) = 1\\
		\lambda(c) \ = \ &3&2 & \rightarrow & \sigma^k(\lambda(c)) = 2\\
		\end{array}
		$
	\end{center}
	Thus $C^{\sigma^k}(v)\in \{a,b\}$. Consider now $v^\prime$:
	\begin{center}
		$
		\begin{array}{ccccc}
		\lambda(a) \ = \ &0&10 & \rightarrow & \sigma^{k^\prime}(\lambda(a)) = 1\\
		\lambda(b) \ = \ &1&0 & \rightarrow & \sigma^{k^\prime}(\lambda(b)) = 1\\
		\lambda(c) \ = \ &10&1 & \rightarrow & \sigma^{k^\prime}(\lambda(c)) = 2\\
		\end{array}
		$
	\end{center}
	Thus $C^{\sigma^{k^\prime}}(v^\prime)\in \{a,b\}$.
\end{example}
	
(Note: observe incompatibility between the $\Sigma_\text{threshold}$ and $\Sigma_\text{PPD}$)\commentOC{Here we should simply observe that some $\sigma \in \SThreshold$ fail to satisfy PPd (give an example to prove it), and conclude that therefore, the Threshold condition is not enough to ensure a satisfactory notion of spread measure.}

\begin{example}[Problem: does $\Sigma$ really compromise?]
	\commentOC{Try to capture your intuition formally and in a shorter way by defining a property that you think a “compromise” sigma should satisfy and that some sigma in $\SThreshold$ do not satisfy. Unless this is already captured by PPd. (Not urgent, can be left for later.)}
	Note that $\mathbf{\Sigma_\text{threshold}}$ does not consider the notion of compromise as we promote. In the case of the Example \ref{ex:exk}, $b$ is still the outcome considering $\sigma^k \in \Sigma_\text{threshold}$ with $k = 2$. With $k=1$ the winner is $d$ which is even worse fairness-wise. By Proposition \ref{prop:equivalence_SigmaThrs} the specific utility assignment is not relevant to determine the outcome, so consider the Borda one and let's have a look at the utility vectors of the alternatives:
	
	\begin{center}
		$
		\begin{array}{ccccccc}
		\mathbf{a} \ = \ &(& \overbrace{3, \dots,3,}^{v_{1-26}} & \overbrace{1, \dots, 1,}^{v_{27-51}} & \overbrace{1, \dots, 1,}^{v_{52-z}} & \overbrace{2, \dots, 2}^{v_{(z+1)-100}} &) \\
		\mathbf{b}\ = \ &(& 2, \dots,2, & 2, \dots, 2, & 2, \dots, 2, & 0, \dots, 0 &) \\
		\mathbf{c}\ = \ &(& 1, \dots,1, & 3, \dots, 3, & 0, \dots, 0, & 1, \dots, 1 &) \\
		\mathbf{d}\ = \ &(& 0, \dots,0, & 0, \dots, 0, & 3, \dots, 3, & 3, \dots, 3 &) \\
		\end{array}
		$
	\end{center}
	
	Where $v_{1-26}$ indicate the group of voters from $v_1$ to $v_{26}$. Consider now the loss vectors: 

	\begin{center}
		$
		\begin{array}{ccccccc}
		\lambda(a)\ = \ &(& \overbrace{0, \dots,0,}^{|v_{1-26}|=26} & \overbrace{2, \dots, 2,}^{|v_{27-51}|=25} & \overbrace{2, \dots, 2,}^{|v_{\dots}|=z-51} & \overbrace{1, \dots, 1}^{|v_{\dots}|=100-z} &) \\
		\lambda(b)\ = \ &(& 1, \dots,1, & 1, \dots, 1, & 1, \dots, 1, & 3, \dots, 3 &) \\
		\lambda(c)\ = \ &(& 2, \dots,2, & 0, \dots, 0, & 3, \dots, 3, & 2, \dots, 2 &) \\
		\lambda(d)\ = \ &(& 3, \dots,3, & 3, \dots, 3, & 0, \dots, 0, & 0, \dots, 0 &) \\
		\end{array}
		$
	\end{center}
	
	\begin{center}
		$
		\begin{array}{cc|c|c}
		& k=1 & k=2 & k=3 \\ \cmidrule{2-4}
		\sigma^k (\lambda (a))\ = \ & 74 & z-26 & {\color{red}0} \\
		\sigma^k (\lambda (b))\ = \ & 100 & {\color{red}100-z} & 100-z \\
		\sigma^k (\lambda (c))\ = \ & 75 & 75 & z-51 \\
		\sigma^k (\lambda (d))\ = \ & {\color{red}51} & 51 & 51 \\
		\end{array}
		$
	\end{center}	
	
	Then for $k=1$ the winner would be $\color{red}d$; for $k=2$ would be $\color{red}b$ because $100-z$ is always less than $z-26$, $75$ and $51$ (remember that $z \in {64, \dots 99}$); for $k=3$ the winner would be $\color{red}a$.
	\\\\
	It is worth noting that if we consider $\sigma \in \mathbf{\Sigma_\text{PPD}}$ the outcome changes according to the utility assignment. For example, if we consider the UA such that the alternative ranked first has utility $1000$ and the rest $2,1,0$ then the loss vectors are: 
	\begin{center}
		$
		\begin{array}{ccccccc}
		\lambda(a)\ = \ &(& \overbrace{0, \dots,0,}^{|v_{1-26}|=26} & \overbrace{999, \dots, 999,}^{|v_{27-51}|=25} & \overbrace{999, \dots, 999,}^{|v_{\dots}|=z-51} & \overbrace{998, \dots, 998}^{|v_{\dots}|=100-z} &) \\
		\lambda(b)\ = \ &(& 998, \dots,998, & 998, \dots, 998, & 998, \dots, 998, & 1000, \dots, 1000 &) \\
		\lambda(c)\ = \ &(& 998, \dots,998, & 0, \dots, 0, & 1000, \dots, 1000, & 998, \dots, 998 &) \\
		\lambda(d)\ = \ &(& 1000, \dots,1000, & 1000, \dots, 1000, & 0, \dots, 0, & 0, \dots, 0 &) \\
		\end{array}
		$
	\end{center}
	The winner is again $b$ but in this case we want it to be the outcome since it corresponds to our idea of compromise.
\end{example}

\end{proof}

\begin{definition}
\label{D1} \bigskip We say that a SCR $f:L(A)^{N}\rightarrow 2^{A} \setminus \{\emptyset \}$ is a $\Sigma$-ordinal compromise iff there exists $\sigma \in \Sigma $ and an UA $v$ such that $C^{\sigma }\circ \vpr=f$.
\end{definition}

Note that for $\SThreshold$, by \cref{prop:equivalence_SigmaThrs}, \cref{D1} is equivalent to the apparently following stronger version:

\begin{definition}
\label{D2} We say that an SCR $f:L(A)^{N}\rightarrow 2^{A} \setminus \{\emptyset \}$ is a $\SThreshold$-ordinal compromise iff for every UA $v$ there exists $\sigma \in \SThreshold$ such that $C^{\sigma }\circ \vpr=f$.
\end{definition}

\subsection{Comparison of compromises}
Many ordinal SCRs fail to be ordinal compromises. 

\begin{example}
	\label{ex:qeqn}
To see this, consider the following preference profile

\begin{center}
	$
	\begin{array}{cccc}
	\mathbf{n-1} \quad &a&b&c\\
	\mathbf{1} \quad &c&b&a\\
	\end{array}
	$
\end{center}

where $C^{\sigma }\circ v$ picks $b$ for any UA $v$ and any $\sigma \in \Sigma $ while all BK-compromises (except when $q=n$) pick $a$. Moreover, for any scoring rule (except anti plurality), there is a choice of $n$ where the scoring rule picks $a$. Finally, all Condorcet consistent rules pick $a$ as well.
\end{example}

\commentOC{The proof does not hold here with $\SPPd$. To be checked.}
Regarding FB, consider the following preference profile with $n=2$ and $m=2k+2$

\bigskip

$x$ $a_{1}...a_{k}$ $y$ $b_{1}....b_{k}$

$b_{1}....b_{k-1}$ $y$ $x$ $b_{k}$ $a_{1}...a_{k}$

\bigskip

where under any $\sigma \in \Sigma $, $C^{\sigma }\circ v$ picks $y$ for the UA $v(k)=n-k+1$ while FB picks $x$. So FB fails Definition \ref{D2}, hence Definition \ref{D1}.

Note that this example can be generalized to any $n$ by replicating the first voter as much as one wishes. This illustrates in a clear and simple
way that $C^{\sigma }$ is ready to pay arbitrary amounts of utility (computed as a total loss of utility over all voters) to satisfy someone's envy.

To see that anti plurality also fails Definition 2, consider the following preference profile with $n=12$ and $m=5$

$3$ $.....d$ $a$

$3$ $.....d$ $b$

$3$ $.....d$ $c$

$2$ $......$ $d$

$1$ $.....d$ $e$

\bigskip

where under any $\sigma \in \Sigma $, $C^{\sigma }\circ v$ picks $d$ for the UA $v(k)=n-k+1$ while anti plurality picks $e$.

We thus observe that all BK-compromises, all scoring rules and all Condorcet consistent rules fail to be ordinal compromises.

TODO: See whether the analysis above prevails when the UA $v$ is allowed to vary among individuals.

The two person example that shows that FB is not an ordinal compromise of special interest because $C^{\sigma }$ seems to particularly appealing in two person discrete bargaining environments. In that example, $y$ seems to be a more natural bargaining solution than $x$. So we can try to present $C^{\sigma }$ as a discrete bargaining solution. In this literature there is FB of Brams and Kilgour which is later rectified by Kibris and Sertel \cite{Sertel1999} by adding the disagreement outcome that was originally missing. It will be certainly interesting to introduce and analyze $C^{\sigma }$ in this ordinal context where ranks are utilities. Clearly, minimized envy will be a key condition which will differentiate $C^{\sigma }$ from FB.

Another direction of research to which Ali had pointed: As a spirit of compromise, we pick the alternative whose lambda is “most equally distributed”. But once we have lambdas, it makes sense to ask what else we can do with these. Ali suggested to minimize the median value of lambda. Sounds plausible but what does this mean? What else can we do with our lambdas? I feel that there are plenty of interesting questions here.

\bibliography{biblio}
\bibliographystyle{abbrv}

\appendix
\section{Transf}
Given sets $A, B, C, G \subseteq C^B$, given $\emptyset ≠ V \subseteq B^A$ a subset of the injective functions from $A$ to $B$. Define $F_1 \subseteq C^A$ as the set of functions $f$ such that $\exists v \in V, g \in G \suchthat f = g \circ v$. Define $F_2 \subseteq C^A$ as the set of functions $f$ such that $\forall v \in V, \exists g \in G \suchthat f = g \circ v$. It follows that $F_2 \subseteq F_1$. We are interested in defining a condition on $G$ which permit to obtain $F_1 \subseteq F_2$. 

We say that \emph{$G$ can compensate for $V$-transformations} iff $\forall v, v' \in V, g \in G, \exists g' \in G \suchthat g' \circ v' = g \circ v$.%; equivalently, such that $g^{v'} = g \circ v \circ v'^{-1}$ on the restricted domain $v'(A) \subseteq B$.

\begin{lemma}
	Given $A, B, C, G \subseteq C^B, \emptyset ≠ V \subseteq B^A$ as constrained above, and $F_1, F_2$ defined as above, if $G$ can compensate for $V$-transformations, then $F_1 \subseteq F_2$.
\end{lemma}
\begin{proof}
	Given $f \in F_1$, take $v, g$ such that $f = g \circ v$. Given any $v' \in V$, we want to find $g'$ such that $f = g' \circ v$. Such a function exists thanks to the hypothesized condition.
\end{proof}

\begin{lemma}
	Given $A, B, C, G \subseteq C^B, \emptyset ≠ V \subseteq B^A$ as constrained above, and $F_1, F_2$ defined as above, if $F_1 \subseteq F_2$, then $G$ can compensate for $V$-transformations.
\end{lemma}
\begin{proof}
	We know that $\forall v, g: [\forall v', \exists g' \suchthat g' \circ v' = g \circ v]$. We have to show that $\forall v, v', g: \exists g' \suchthat g' \circ v' = g \circ v$, which is the same thing.
\end{proof}
\end{document}
