\RequirePackage[l2tabu, orthodox]{nag}
\documentclass[version=3.21, pagesize, twoside=off, bibliography=totoc, DIV=calc, fontsize=12pt, a4paper]{scrartcl}
%Permits to copy eg x ⪰ y ⇔ v(x) ≥ v(y) from PDF to unicode data, and to search. From pdfTeX users manual. See https://tex.stackexchange.com/posts/comments/1203887.
	\input glyphtounicode
	\pdfgentounicode=1
%Latin Modern has more glyphs than Computer Modern, such as diacritical characters. fntguide commands to load the font before fontenc, to prevent default loading of cmr.
	\usepackage{lmodern}
%Encode resulting accented characters correctly in resulting PDF, permits copy from PDF.
	\usepackage[T1]{fontenc}
%UTF8 seems to be the default in recent TeX installations, but not all, see https://tex.stackexchange.com/a/370280.
	\usepackage[utf8]{inputenc}
%Provides \newunicodechar for easy definition of supplementary UTF8 characters such as → or ≤ for use in source code.
	\usepackage{newunicodechar}
%Text Companion fonts, much used together with CM-like fonts. Provides \texteuro and commands for text mode characters such as \textminus, \textrightarrow, \textlbrackdbl.
	\usepackage{textcomp}
%St Mary’s Road symbol font, used for ⟦ = \llbracket.
	\usepackage{stmaryrd}
\usepackage{centernot}
%Solves bug in lmodern, https://tex.stackexchange.com/a/261188; probably useful only for unusually big font sizes; and probably better to use exscale instead. Note that the authors of exscale write against this trick.
	%\DeclareFontShape{OMX}{cmex}{m}{n}{
		%<-7.5> cmex7
		%<7.5-8.5> cmex8
		%<8.5-9.5> cmex9
		%<9.5-> cmex10
	%}{}
	%\SetSymbolFont{largesymbols}{normal}{OMX}{cmex}{m}{n}
%More symbols (such as \sum) available in bold version, see https://github.com/latex3/latex2e/issues/71.
	\DeclareFontShape{OMX}{cmex}{bx}{n}{%
	   <->sfixed*cmexb10%
	   }{}
	\SetSymbolFont{largesymbols}{bold}{OMX}{cmex}{bx}{n}
%For small caps also in italics, see https://tex.stackexchange.com/questions/32942/italic-shape-needed-in-small-caps-fonts, https://tex.stackexchange.com/questions/284338/italic-small-caps-not-working.
	\usepackage{slantsc}
	\AtBeginDocument{%
		%“Since nearly no font family will contain real italic small caps variants, the best approach is to substitute them by slanted variants.” -- slantsc doc
		%\DeclareFontShape{T1}{lmr}{m}{scit}{<->ssub*lmr/m/scsl}{}%
		%There’s no bold small caps in Latin Modern, we switch to Computer Modern for bold small caps, see https://tex.stackexchange.com/a/22241
		%\DeclareFontShape{T1}{lmr}{bx}{sc}{<->ssub*cmr/bx/sc}{}%
		%\DeclareFontShape{T1}{lmr}{bx}{scit}{<->ssub*cmr/bx/scsl}{}%
	}
%Warn about missing characters.
	\tracinglostchars=2
%Nicer tables: provides \toprule, \midrule, \bottomrule.
	%\usepackage{booktabs}
%For new column type X which stretches; can be used together with booktabs, see https://tex.stackexchange.com/a/97137. “tabularx modifies the widths of the columns, whereas tabular* modifies the widths of the inter-column spaces.” Loads array.
	%\usepackage{tabularx}
%math-mode version of "l" column type. Requires \usepackage{array}.
	%\usepackage{array}
	%\newcolumntype{L}{>{$}l<{$}}
%Provides \xpretocmd and loads etoolbox which provides \apptocmd, \patchcmd, \newtoggle… Also loads xparse, which provides \NewDocumentCommand and similar commands intended as replacement of \newcommand in LaTeX3 for defining commands (see https://tex.stackexchange.com/q/98152 and https://github.com/latex3/latex2e/issues/89).
	\usepackage{xpatch}
%ntheorem doc says: “empheq provides an enhanced vertical placement of the endmarks”; must be loaded before ntheorem. Loads the mathtools package, which loads and fixes some bugs in amsmath and provides \DeclarePairedDelimiter. amsmath is considered a basic, mandatory package nowadays (Grätzer, More Math Into LaTeX).
	\usepackage[ntheorem]{empheq}
%Package frenchb asks to load natbib before babel-french. Package hyperref asks to load natbib before hyperref.
	\usepackage{natbib}

\newtoggle{LCpres}
	\newtoggle{LCart}
	\newtoggle{LCposter}
	\makeatletter
	\@ifclassloaded{beamer}{
		\toggletrue{LCpres}
		\togglefalse{LCart}
		\togglefalse{LCposter}
		\wlog{Presentation mode}
	}{
		\@ifclassloaded{tikzposter}{
			\toggletrue{LCposter}
			\togglefalse{LCpres}
			\togglefalse{LCart}
			\wlog{Poster mode}
		}{
			\toggletrue{LCart}
			\togglefalse{LCpres}
			\togglefalse{LCposter}
			\wlog{Article mode}
		}
	}
	\makeatother%

%Language options ([french, english]) should be on the document level (last is main); except with tikzposter: put [french, english] options next to \usepackage{babel} to avoid warning. beamer uses the \translate command for the appendix: omitting babel results in a warning, see https://github.com/josephwright/beamer/issues/449. Babel also seems required for \refname.
	\iftoggle{LCpres}{
		\usepackage{babel}
	}{
	}
	%\frenchbsetup{AutoSpacePunctuation=false}
%listings (1.7) does not allow multi-byte encodings. listingsutf8 works around this only for characters that can be represented in a known one-byte encoding and only for \lstinputlisting. Other workarounds: use literate mechanism; or escape to LaTeX (but breaks alignment).
	%\usepackage{listings}
	%\lstset{tabsize=2, basicstyle=\ttfamily, escapechar=§, literate={é}{{\'e}}1}
%I favor acro over acronym because the former is more recently updated (2018 VS 2015 at time of writing); has a longer user manual (about 40 pages VS 6 pages if not counting the example and implementation parts); has a command for capitalization; and acronym suffers a nasty bug when ac used in section, see https://tex.stackexchange.com/q/103483 (though this might be the fault of the silence package and might be solved in more recent versions, I do not know) and from a bug when used with cleveref, see https://tex.stackexchange.com/q/71364. However, loading it makes compilation time (one pass on this template) go from 0.6 to 1.4 seconds, see https://bitbucket.org/cgnieder/acro/issues/115. Option short-format not usable in the package options as it is fragile, see https://tex.stackexchange.com/q/466882.
	\usepackage[single]{acro}
	%\acsetup{short-format = {\MakeUppercase}}
	\DeclareAcronym{AMCD}{short=amcd, long={Aide Multicritère à la Décision}}
\DeclareAcronym{AR}{short=ar, long={Argumentative Recommender}}
\DeclareAcronym{DA}{short=da, long={Decision Analysis}}
\DeclareAcronym{DJ}{short=dj, long={Deliberated Judgment}}
\DeclareAcronym{DM}{short=dm, long={Decision Maker}}
\DeclareAcronym{DP}{short=dp, long={Deliberated Preference}}
\DeclareAcronym{MAVT}{short=mavt, long={Multiple Attribute Value Theory}}
\DeclareAcronym{MCDA}{short=mcda, long={Multicriteria Decision Aid}}
\DeclareAcronym{MIP}{short=mip, long={Mixed Integer Program}}


\iftoggle{LCpres}{
	%I favor fmtcount over nth because it is loaded by datetime anyway; and fmtcount warns about possible conflicts when loaded after nth.
	\usepackage{fmtcount}
	%For nice input of date of presentation. Must be loaded after the babel package. Has possible problems with srcletter: https://golatex.de/verwendung-von-babel-und-datetime-in-scrlttr2-schlaegt-fehlt-t14779.html.
	\usepackage[nodayofweek]{datetime}
}{
}
%For presentations, Beamer implicitely uses the pdfusetitle option. ntheorem doc says to load hyperref “before the first use of \newtheorem”. autonum doc mandates option hypertexnames=false. I want to highlight links only if necessary for the reader to recognize it as a link, to reduce distraction. In presentations, this is already taken care of by beamer (https://tex.stackexchange.com/a/262014). If using colorlinks=true in a presentation, see https://tex.stackexchange.com/q/203056. Crashes the first compilation with tikzposter, just compile again and the problem disappears, see https://tex.stackexchange.com/q/254257.
\makeatletter
\iftoggle{LCpres}{
	\usepackage{hyperref}
}{
	\usepackage[hypertexnames=false, pdfusetitle, linkbordercolor={1 1 1}, citebordercolor={1 1 1}, urlbordercolor={1 1 1}]{hyperref}
	%https://tex.stackexchange.com/a/466235
	\pdfstringdefDisableCommands{%
		\let\thanks\@gobble
	}
}
\makeatother
%urlbordercolor is used both for \url and \doi, which I think shouldn’t be colored, and for \href, thus might want to color manually when required. Requires xcolor.
	\NewDocumentCommand{\hrefblue}{mm}{\textcolor{blue}{\href{#1}{#2}}}
%hyperref doc says: “Package bookmark replaces hyperref’s bookmark organization by a new algorithm (...) Therefore I recommend using this package”.
	\usepackage{bookmark}
%Need to invoke hyperref explicitly to link to line numbers: \hyperlink{lintarget:mylinelabel}{\ref*{lin:mylinelabel}}, with \ref* to disable automatic link. Also see https://tex.stackexchange.com/q/428656 for referencing lines from another document.
	%\usepackage{lineno}
	%\NewDocumentCommand{\llabel}{m}{\hypertarget{lintarget:#1}{}\linelabel{lin:#1}}
	%\setlength\linenumbersep{9mm}
%For complex authors blocks. Seems like authblk wants to be later than hyperref, but sooner than silence. See https://tex.stackexchange.com/q/475513 for the patch to hyperref pdfauthor.
	\ExplSyntaxOn
	\seq_new:N \g_oc_hrauthor_seq
	\NewDocumentCommand{\addhrauthor}{m}{
		\seq_gput_right:Nn \g_oc_hrauthor_seq { #1 }
	}
	%Should be \NewExpandableDocumentCommand, but this is not yet provided by my version of xparse
	\DeclareExpandableDocumentCommand{\hrauthor}{}{
		\seq_use:Nn \g_oc_hrauthor_seq {,~}
	}
	\ExplSyntaxOff
	{
		\catcode`#=11\relax
		\gdef\fixauthor{\xpretocmd{\author}{\addhrauthor{#2}}{}{}}%
	}
	\iftoggle{LCart}{
		\usepackage{authblk}
		\renewcommand\Affilfont{\small}
		\fixauthor
		\AtBeginDocument{
		    \hypersetup{pdfauthor={\hrauthor}}
		}
	}{
	}
%I do not use floatrow, because it requires an ugly hack for proper functioning with KOMA script (see scrhack doc). Instead, the following command centers all floats (using \centering, as the center environment adds space, http://texblog.net/latex-archive/layout/center-centering/), and I manually place my table captions above and figure captions below their contents (https://tex.stackexchange.com/a/3253).
	\makeatletter
	\g@addto@macro\@floatboxreset\centering
	\makeatother
%Permits to customize enumeration display and references
	%\nottoggle{LCpres}{
		\usepackage{enumitem} %follow list environments by a string to customize enumeration, example: \begin{description}[itemindent=8em, labelwidth=!] or \begin{enumerate}[label=({\roman*}), ref={\roman*}].
	%}{
	%}
%Provides \Cen­ter­ing, \RaggedLeft, and \RaggedRight and en­vi­ron­ments Cen­ter, FlushLeft, and FlushRight, which al­low hy­phen­ation. With tikzposter, seems to cause 1=1 to be printed in the middle of the poster.
	%\usepackage{ragged2e}
%To typeset units by closely following the “official” rules.
	%\usepackage[strict]{siunitx}
%Turns the doi provided by some bibliography styles into URLs. However, uses old-style dx.doi url (see 3.8 DOI system Proxy Server technical details, “Users may resolve DOI names that are structured to use the DOI system Proxy Server (https://doi.org (current, preferred) or earlier syntax http://dx.doi.org).”, https://www.doi.org/doi_handbook/3_Resolution.html). The patch solves this.
	\usepackage{doi}
	\makeatletter
	\patchcmd{\@doi}{http://dx.doi.org}{https://doi.org}{}{}
	\makeatother
%Makes sure upper case greek letters are italic as well.
	\usepackage{fixmath}
%Provides \mathbb; obsoletes latexsym (see http://tug.ctan.org/macros/latex/base/latexsym.dtx). Relatedly, \usepackage{eucal} to change the mathcal font and \usepackage[mathscr]{eucal} (apparently equivalent to \usepackage[mathscr]{euscript}) to supplement \mathcal with \mathscr. This last option is not very useful as both fonts are similar, and the intent of the authors of eucal was to provide a replacement to mathcal (see doc euscript). Also provides \mathfrak for supplementary letters.
	\usepackage{amsfonts}
%Provides a beautiful (IMHO) \mathscr and really different than \mathcal, for supplementary uppercase letters. But there is no bold version. Alternative: mathrsfs (more slanted), but when used with tikzposter, it warns about size substitution, see https://tex.stackexchange.com/q/495167.
	\usepackage[scr]{rsfso}
%Multiple means to produce bold math: \mathbf, \boldmath (defined to be \mathversion{bold}, see fntguide), \pmb, \boldsymbol (all legacy, from LaTeX base and AMS), \bm (the most recommended one), \mathbold from package fixmath (I don’t see its advantage over \boldsymbol).
%“The \boldsymbol command is obtained preferably by using the bm package, which provides a newer, more powerful version than the one provided by the amsmath package. Generally speaking, it is ill-advised to apply \boldsymbol to more than one symbol at a time.” — AMS Short math guide. “If no bold font appears to be available for a particular symbol, \bm will use ‘poor man’s bold’” — bm. It is “best to load the package after any packages that define new symbol fonts” – bm. bm defines \boldsymbol as synonym to \bm. \boldmath accesses the correct font if it exists; it is used by \bm when appropriate. See https://tex.stackexchange.com/a/10643 and https://github.com/latex3/latex2e/issues/71 for some difficulties with \bm.
	\usepackage{bm}
	\nottoggle{LCpres}{
	%https://ctan.org/pkg/amsmath recommends ntheorem, which supersedes amsthm, which corrects the spacing of proclamations and allows for theoremstyle. Option standard loads amssymb and latexsym. Must be loaded after amsmath (from ntheorem doc). From cleveref doc, “ntheorem is fully supported and even recommended”; says to load cleveref after ntheorem. When used with tikzposter, warns about size substitution for the lasy (latexsym) font when using \url, because ntheorem loads latexsym; relatedly (but not directly related to ntheorem), size substitution warning with the cmex font happens when loading amsmath and using \url.
		\usepackage[thmmarks, amsmath, standard, hyperref]{ntheorem}
		%empheq doc says to do this after loading ntheorem
		\usetagform{default}
	%Provides \cref. Unfortunately, cref fails when the language is French and referring to a label whose name contains a colon (https://tex.stackexchange.com/q/83798). Use \cref{sec\string:intro} to work around this. cleveref should go “laster” than hyperref.
		\usepackage[capitalise]{cleveref}
	}{
	}
	\nottoggle{LCposter}{
	%Equations get numbers iff they are referenced. Loading order should be “amsmath → hyperref → cleveref → autonum”, according to autonum doc. Use this in preference to the showonlyrefs option from mathtools, see https://tex.stackexchange.com/q/459918 and autonum doc. See https://tex.stackexchange.com/a/285953 for the etex line. Incompatible with my version of tikzposter (produces “! Improper \prevdepth”).
		\expandafter\def\csname ver@etex.sty\endcsname{3000/12/31}\let\globcount\newcount
		\usepackage{autonum}
	}{
	}
%Also loaded by tikz.
	\usepackage{xcolor}
\iftoggle{LCpres}{
	\usepackage{tikz}
	%\usetikzlibrary{babel, matrix, fit, plotmarks, calc, trees, shapes.geometric, positioning, plothandlers, arrows, shapes.multipart}
}{
}
%Vizualization, on top of TikZ
	%\usepackage{pgfplots}
	%\pgfplotsset{compat=1.14}
\usepackage{graphicx}
	\graphicspath{{graphics/}}

%Provides \print­length{length}, useful for debugging.
	%\usepackage{printlen}
	%\uselengthunit{mm}

\iftoggle{LCpres}{
	\usepackage{appendixnumberbeamer}
	%I have yet to see anyone actually use these navigation symbols; let’s disable them
	\setbeamertemplate{navigation symbols}{} 
	\usepackage{preamble/beamerthemeParisFrance}
	\setcounter{tocdepth}{10}
}{
}

%Do not use the displaymath environment: use equation. Do not use the eqnarray or eqnarray* environments: use align(*). This improves spacing. (See l2tabu or amsldoc.)


%Requires package xcolor.
\newcommand{\commentOC}[1]{\textcolor{blue}{\small$\big[$OC: #1$\big]$}}
%Requires package babel and option [french]. According to babel doc, need two braces around \selectlanguage to make the changes really local.
\newcommand{\commentOCf}[1]{\textcolor{blue}{{\small\selectlanguage{french}$\big[$OC : #1$\big]$}}}
\newcommand{\commentYM}[1]{\textcolor{red}{\small$\big[$YM: #1$\big]$}}
\newcommand{\commentYMf}[1]{\textcolor{red}{{\small\selectlanguage{french}$\big[$YM : #1$\big]$}}}

\bibliographystyle{abbrvnat}

%https://tex.stackexchange.com/a/467188 - uncomment if one of those symbols is used.
%\DeclareFontFamily{U} {MnSymbolD}{}
%\DeclareFontShape{U}{MnSymbolD}{m}{n}{
%  <-6> MnSymbolD5
%  <6-7> MnSymbolD6
%  <7-8> MnSymbolD7
%  <8-9> MnSymbolD8
%  <9-10> MnSymbolD9
%  <10-12> MnSymbolD10
%  <12-> MnSymbolD12}{}
%\DeclareFontShape{U}{MnSymbolD}{b}{n}{
%  <-6> MnSymbolD-Bold5
%  <6-7> MnSymbolD-Bold6
%  <7-8> MnSymbolD-Bold7
%  <8-9> MnSymbolD-Bold8
%  <9-10> MnSymbolD-Bold9
%  <10-12> MnSymbolD-Bold10
%  <12-> MnSymbolD-Bold12}{}
%\DeclareSymbolFont{MnSyD} {U} {MnSymbolD}{m}{n}
%\DeclareMathSymbol{\ntriplesim}{\mathrel}{MnSyD}{126}
%\DeclareMathSymbol{\nlessgtr}{\mathrel}{MnSyD}{192}
%\DeclareMathSymbol{\ngtrless}{\mathrel}{MnSyD}{193}
%\DeclareMathSymbol{\nlesseqgtr}{\mathrel}{MnSyD}{194}
%\DeclareMathSymbol{\ngtreqless}{\mathrel}{MnSyD}{195}
%\DeclareMathSymbol{\nlesseqgtrslant}{\mathrel}{MnSyD}{198}
%\DeclareMathSymbol{\ngtreqlessslant}{\mathrel}{MnSyD}{199}
%\DeclareMathSymbol{\npreccurlyeq}{\mathrel}{MnSyD}{228}
%\DeclareMathSymbol{\nsucccurlyeq}{\mathrel}{MnSyD}{229}

%03B3 Greek Small Letter Gamma
\newunicodechar{γ}{\gamma}
%03B4 Greek Small Letter Delta
\newunicodechar{δ}{\delta}
%2115 Double-Struck Capital N
\newunicodechar{ℕ}{\mathbb{N}}
%211D Double-Struck Capital R
\newunicodechar{ℝ}{\mathbb{R}}
%21CF Rightwards Double Arrow with Stroke
\newunicodechar{⇏}{\nRightarrow}
%21D2 Rightwards Double Arrow
\newunicodechar{⇒}{\ensuremath{\Rightarrow}}
%21D4 Left Right Double Arrow
\newunicodechar{⇔}{\Leftrightarrow}
%21DD Rightwards Squiggle Arrow
\newunicodechar{⇝}{\rightsquigarrow}
%2212 Minus Sign
\newunicodechar{−}{\ifmmode{-}\else\textminus\fi}
%2227 Logical And
\newunicodechar{∧}{\land}
%2228 Logical Or
\newunicodechar{∨}{\lor}
%2229 Intersection
\newunicodechar{∩}{\cap}
%222A Union
\newunicodechar{∪}{\cup}
%2260 Not Equal To (handy also as text in informal writing)
\newunicodechar{≠}{\ensuremath{\neq}}
%2264 Less-Than or Equal To
\newunicodechar{≤}{\leq}
%2265 Greater-Than or Equal To
\newunicodechar{≥}{\geq}
%2270 Neither Less-Than nor Equal To
\newunicodechar{≰}{\nleq}
%2271 Neither Greater-Than nor Equal To
\newunicodechar{≱}{\ngeq}
%2272 Less-Than or Equivalent To
\newunicodechar{≲}{\lesssim}
%2273 Greater-Than or Equivalent To
\newunicodechar{≳}{\gtrsim}
%2274 Neither Less-Than nor Equivalent To – also, from MnSymbol: \nprecsim, a more exact match to the Unicode symbol; and \npreccurlyeq, too small
\newunicodechar{≴}{\not\preccurlyeq}
%2275 Neither Greater-Than nor Equivalent To
\newunicodechar{≵}{\not\succcurlyeq}
%2279 Neither Greater-Than nor Less-Than – requires MnSymbol; also \nlessgtr from txfonts/pxfonts, \ngtreqless from MnSymbol (but much higher), \ngtrless from MnSymbol (a more exact match to the Unicode symbol); for incomparability (not matching this Unicode symbol), may also consider \ntriplesim from MnSymbol,\nparallelslant from fourier, \between from mathabx, or ⋈
\newunicodechar{≹}{\ngtreqlessslant}
%227A Precedes
\newunicodechar{≺}{\prec}
%227B Succeeds
\newunicodechar{≻}{\succ}
%227C Precedes or Equal To
\newunicodechar{≼}{\preccurlyeq}
%227D Succeeds or Equal To
\newunicodechar{≽}{\succcurlyeq}
%227E Precedes or Equivalent To
\newunicodechar{≾}{\precsim}
%227F Succeeds or Equivalent To
\newunicodechar{≿}{\succsim}
%2280 Does Not Precede
\newunicodechar{⊀}{\nprec}
%2281 Does Not Succeed
\newunicodechar{⊁}{\nsucc}
%22B2 Normal Subgroup Of – \triangleleft is too small compared to \trianglelefteq and the like
\newunicodechar{⊲}{\lhd}
%22B3 Contains as Normal Subgroup
\newunicodechar{⊳}{\rhd}
%22B4 Normal Subgroup of or Equal To
\newunicodechar{⊴}{\trianglelefteq}
%22B5 Contains as Normal Subgroup or Equal To
\newunicodechar{⊵}{\trianglerighteq}
%22C8 Bowtie
\newunicodechar{⋈}{\bowtie}
%22EA Not Normal Subgroup Of
\newunicodechar{⋪}{\ntriangleleft}
%22EB Does Not Contain As Normal Subgroup
\newunicodechar{⋫}{\ntriangleright}
%22EC Not Normal Subgroup of or Equal To
\newunicodechar{⋬}{\ntrianglelefteq}
%22ED Does Not Contain as Normal Subgroup or Equal
\newunicodechar{⋭}{\ntrianglerighteq}
%25A1 White Square
\newunicodechar{□}{\Box}
%27E6 Mathematical Left White Square Bracket – there’s also \llbracket from stmaryrd
\newunicodechar{⟦}{\text{\textlbrackdbl}}
%27E7 Mathematical Right White Square Bracket – there’s also \rrbracket from stmaryrd
\newunicodechar{⟧}{\text{\textrbrackdbl}}
%27FC Long Rightwards Arrow from Bar
\newunicodechar{⟼}{\longmapsto}
%2AB0 Succeeds Above Single-Line Equals Sign
\newunicodechar{⪰}{\succeq}
%301A Left White Square Bracket
\newunicodechar{〚}{\textlbrackdbl}
%301B Right White Square Bracket
\newunicodechar{〛}{\textrbrackdbl}
%→ is defined by default as \textrightarrow, which is invalid in math mode. Same thing for the three other commands. I redefine those four using \DeclareUnicodeCharacter instead of \newunicodechar because the latter warns about the previous definition.
%→ Rightwards Arrow
\DeclareUnicodeCharacter{2192}{\ifmmode\rightarrow\else\textrightarrow\fi}
%¬ Not Sign
\DeclareUnicodeCharacter{00AC}{\ifmmode\lnot\else\textlnot\fi}
%… Horizontal Ellipsis
\DeclareUnicodeCharacter{2026}{\ifmmode\dots\else\textellipsis\fi}
%× Multiplication Sign
\DeclareUnicodeCharacter{00D7}{\ifmmode\times\else\texttimes\fi}


\NewDocumentCommand{\R}{}{ℝ}
\NewDocumentCommand{\N}{}{ℕ}
%\mathscr is rounder than \mathcal.
\NewDocumentCommand{\powerset}{m}{\mathscr{P}(#1)}
%Powerset without zero.
\NewDocumentCommand{\powersetz}{m}{\mathscr{P}^*(#1)}
%https://tex.stackexchange.com/a/45732, works within both \set and \set*, same spacing than \mid (https://tex.stackexchange.com/a/52905).
\NewDocumentCommand{\suchthat}{}{\;\ifnum\currentgrouptype=16 \middle\fi|\;}
%Integer interval.
\NewDocumentCommand{\intvl}{m}{⟦#1⟧}
%Allows for \abs and \abs*, which resizes the delimiters.
\DeclarePairedDelimiter\abs{\lvert}{\rvert}
\DeclarePairedDelimiter\card{\lvert}{\rvert}
%Perhaps should use U+2016 ‖ DOUBLE VERTICAL LINE here?
\DeclarePairedDelimiter\norm{\lVert}{\rVert}
%Better than using the package braket because braket introduces possibly undesirable space. Then: \begin{equation}\set*{x \in \R^2 \suchthat \norm{x}<5}\end{equation}.
\DeclarePairedDelimiter\set{\{}{\}}
\DeclarePairedDelimiter\ceil{\lceil}{\rceil}
\DeclarePairedDelimiter\floor{\lfloor}{\rfloor}
\DeclareMathOperator*{\argmax}{arg\,max}
\DeclareMathOperator*{\argmin}{arg\,min}

%We want the straight form of \phi for mathematics, as recommended in UTR #25: Unicode support for mathematics, and thus use \phi for the mathematical symbol and not \varphi; and similarly \epsilon is preferred to \varepsilon for the mathematical symbol.

%The amssymb solution.
%\NewDocumentCommand{\restr}{mm}{{#1}_{\restriction #2}}
%Another acceptable solution.
%\NewDocumentCommand{\restr}{mm}{{#1|}_{#2}}
%https://tex.stackexchange.com/a/278631; drawback being that sometimes the text collides with the line below.
\NewDocumentCommand\restr{mm}{#1\raisebox{-.5ex}{$|$}_{#2}}


%Decision Theory (MCDA and SC)
\NewDocumentCommand{\allalts}{}{A}
\NewDocumentCommand{\allcrits}{}{\mathscr{C}}
\NewDocumentCommand{\alts}{}{A}
\NewDocumentCommand{\dm}{}{i}
\NewDocumentCommand{\allF}{}{\mathscr{F}}
\NewDocumentCommand{\allvoters}{}{\mathscr{N}}
\NewDocumentCommand{\voters}{}{N}
\NewDocumentCommand{\allprofs}{}{\boldsymbol{\mathcal{R}}}
\NewDocumentCommand{\prof}{}{P}
\NewDocumentCommand{\ibar}{}{\overline{i}}
\NewDocumentCommand{\lprof}{}{\lambda_P}
\NewDocumentCommand{\lprofi}{O{x}}{\lambda_P(#1)_i}
\NewDocumentCommand{\lprofibar}{O{x}}{\lambda_P(#1)_{\overline{i}}}
\NewDocumentCommand{\ineq}{}{(\sigma \circ \lambda_P)}

\NewDocumentCommand{\linors}{}{\mathcal{L}(\allalts)}
%Thanks to https://tex.stackexchange.com/q/154549
	%\makeatletter
	%\def\@myRgood@#1#2{\mathrel{R^X_{#2}}}
	%\def\myRgood{\@ifnextchar_{\@myRgood@}{\mathrel{R^X}}}
	%\makeatother
\NewDocumentCommand{\pref}{}{\succ}
\NewDocumentCommand{\prefi}{O{i}}{\succ_{#1}}
\NewDocumentCommand{\paretopt}{}{\text{PO}}
\NewDocumentCommand{\SPPd}{}{\Sigma^\text{PPd}}
\NewDocumentCommand{\SAll}{}{\Sigma^\text{All}}
\NewDocumentCommand{\SThreshold}{}{\Sigma_\text{threshold}}
\NewDocumentCommand{\vpr}{}{\boldsymbol{v}}

\NewDocumentCommand{\musigma}{O{\sigma}O{P}}{\argmin_{A}({#1}\circ\lambda_{{#2}})}
\NewDocumentCommand{\mustar}{O{\sigma}O{P}}{\argmin_{\paretopt({#2})} ({#1} \circ \lambda_{#2})}
\NewDocumentCommand{\minineq}{O{\allalts}}{\argmin_{#1}(\sigma \circ \lambda)}
\NewDocumentCommand{\FBP}{}{\text{FB}(P)}
\NewDocumentCommand{\POP}{}{\text{PO}(P)}

\NewDocumentCommand{\alllosses}{}{\intvl{0, m-1}^N}

\NewDocumentCommand{\Ptop}{}{\bar{P}}
\NewDocumentCommand{\sigmatop}{}{\bar{\sigma}}

\NewDocumentCommand{\fltwo}{}{\floor{\bar{l_2}}}
\NewDocumentCommand{\bltwo}{}{\bar{l_2}}

\newtheorem{conjecture}{Conjecture}


%I find these settings useful in draft mode. Should be removed for final versions.
	%Which line breaks are chosen: accept worse lines, therefore reducing risk of overfull lines. Default = 200.
		\tolerance=2000
	%Accept overfull hbox up to...
		\hfuzz=2cm
	%Reduces verbosity about the bad line breaks.
		\hbadness 5000
	%Reduces verbosity about the underful vboxes.
		\vbadness=1300

\title{Ex-Ante versus Ex-Post Compromise}
\author{Submission ID}
\hypersetup{
	pdfsubject={Social choice},
	pdfkeywords={axiomatic analysis},
}

\begin{document}
\maketitle

\begin{abstract}
	A classical social choice setting is composed of a group of individuals, or voters, that express their preferences over a set of alternatives. The social choice problem consists in defining a procedure able to determine a collective choice for this group of voters, starting from their individual preferences. Such procedure is called social choice rule and it can be defined as a function mapping preference profiles to alternatives. Depending on the properties that this function satisfies, very different outcomes can be produced starting from the same initial profile. The plurality rule is one of the most common social choice rule and it consists in selecting, as a winner, the alternative that is considered the best by the largest number of voters forming the society. Yet, this rule can pick, as a winner, an alternative that is considered the worst by a strict majority of voters. Such outcome may be undesirable. Several procedures, the so-called compromise rules, have been proposed in the literature that aim to find a compromise. Nevertheless, all those rules can be defined as \emph{ex-ante compromises} or \emph{procedural compromises}, i.e., they impose over individuals a willingness to compromise but they do not ensure an outcome where everyone has effectively compromised. In this work, we approach the problem of compromise from an \emph{ex-post} perspective, favoring an outcome where every voter gives up her most preferred positions if this increases equality. We propose a new notion of compromise in the social choice context, considering ordinal utilities.
\end{abstract}

\section{Introduction}
\label{sec:introduction}
In a classical social choice scenario, several individuals express their preferences over a set of alternatives and there is no unique procedure for selecting a common agreement between them. Nevertheless, there is an accepted understanding that collective choices must reflect compromises. One of the first to explicitly refer to a social choice rule (SCR) as a compromise is \cite{Sertel1986} introducing the \textit{majoritarian compromise}. This SCR, further analyzed by \cite{Sertel1999}, is a rediscovery of a method suggested by James W. Bucklin \commentOC{Ref?} 
\commentRS{ I remember an article by Bucklin which I cannot find. I think we cancite it as done in Section 2.1 of this paper: https://www.sciencedirect.com/science/article/pii/S0022000014001433\#br0490}
at the beginning of the 20th century. It falls back from considering everyone’s ideal alternative to considering the voters’ second, third and more generally k-\emph{th} best until one of the alternatives is among the first k best for a majority. \cite{Brams2001} generalize this concept and introduce a class of SCRs called $q-$\textit{approval fall-back bargaining }where $q$ is the level of required support which can vary from a single voter up to unanimity. Naturally, different choices of $q$ lead to different SCRs, such as $q=1$ giving the plurality rule; $q$ being majority giving the majoritarian compromise and $q$ being unanimity giving a bargaining procedure called \textit{fall-back bargaining} which has been further analyzed by \cite{Kibris2007} and \cite{Congar2012}. 

As \cite{OezkalSanver2004} discuss, the concept of compromising is mostly understood as the trade off between the number of voters supporting an alternative (i.e., the quantity of support) and the distance of that alternative from the supporters' ideal alternative (i.e., the quality of support). This trade off, which is explicit for $q-$approval fall-back bargaining, is also the basis for several other SCRs such as the \textit{median voting rule} proposed by \cite{Bassett1999} and further analyzed by \cite{Gehrlein2003} or the \textit{Condorcet practical method }described by \cite{Nurmi1999}.

\cite{Merlin2019} identify and analyze a large class of \textit{compromise rules} which are based on balancing the trade off between the quality and the quantity of support. On the other hand, as conflicting
individual preferences over the available alternatives makes impossible to ensure the best outcome for every member of the group, one can argue that making a collective choice \textit{per se} implies compromising. In that sense, every SCR incorporates some understanding of what a compromise means. To be sure, there are instances where this understanding may contradict common sense, such as dictatorships where one voter always ensures his best outcome whatever the others prefer. Nevertheless, interesting SCRs base the collective choice on the principle that all voters may have to fall back from their ideal position. Whether, at the end of the day, all voters do effectively fall back or not is another issue which is the subject matter of this paper.

Sometimes, indeed, they do not. This observation was the basis for an objection made by Jean-François Laslier to the nomenclature on compromising.\footnote{This happened at a CNRS workshop on compromising hosted by Istanbul Bilgi University at Buyukada, Istanbul on fall 2018.} Consider the following example.
\begin{example}
	\label{ex:ex1}
	Let $N$ be a set of $n ≥ 3$ voters and $A$ a set of alternatives. $\linors$ represents the set of linear orders over $A$. Consider the following preference profile $P\in \linors^{N}$:
	\begin{center}
		$
		\begin{array}{cccc}
		\mathbf{1} \quad &c&b&a\\
		\mathbf{n-1} \quad &a&b&c\\
		\end{array},
		$
	\end{center}
	which represents one individual who prefers $c$ to $b$, $b$ to $a$, hence $c$ to $a$; and $n-1$ individuals who prefer $a$ to $b$, $b$ to $c$, hence $a$ to $c$. At $P$, all BK-compromises, except fall-back bargaining i.e. when $q=n$, will ignore the single voter and will pick $a$ as the collective outcome.
\end{example}

As a matter of fact, almost every interesting SCR will ignore this “marginal minority” and choose $a$ in this situation. While this choice is defensible on the grounds of qualified majoritarianism, the presence of $b$, which receives unanimous support when each voter falls back one step from his ideal point, renders questionable whether $a$ can be qualified as a compromise. The question becomes even more acute for majoritarian SCRs, including the majoritarian compromise, where $a$ would
remain the collective choice even when the ignored group is much larger.

\begin{example}
	\label{ex:ex2}
	Consider the following preference profile with $n=100$:
	\begin{center}
		$
		\begin{array}{cccc}
		\mathbf{49} \quad &c&b&a\\
		\mathbf{51} \quad &a&b&c\\
		\end{array}.
		$
	\end{center}
	When $q\in \intvl{1,\frac{n}{2}+1} $, all BK-compromises pick $a$, and, again, it does not appear as a compromise as 51 voters reach their best alternative while the remaining 49 voters have to be contented with their worst one. Note that for $q\in \intvl{1,\frac{n}{2}-1} $ the set of possible common agreements determined by the fall-back bargaining procedure is $\{a,c\}$. Nevertheless, $a$ receives the highest support, thus it is elected.
\end{example}

It is important to observe that all these SCRs impose to voters a willingness to compromise which does not mean that under the collective choice that will be made, all voters will be effectively compromising. In other words, the term “compromise” in this literature refers to procedural or \textit{ex-ante} compromises, which is different than outcome oriented or \textit{ex-post} compromises, a conceptual distinction that seems to be overlooked in the literature.

To define an ex-post compromise, we adopt to our framework a concept of equal losses that is prevalent in the literature that considers the allocation of continuous utilities. This principle is used for bargaining problems (\cite{Chun1988}, \cite{Chun1991}) \commentOC{TODO use citet and citep} as well as for bankruptcy problems (\cite{Herrero2001}).
We introduce two definitions for being a compromise. In both of them, we pick a spread measure that determines how equally a given vector of real numbers is distributed and make a collective choice where voters give up from their ideal points “as equal as possible". \commentRS{Here, put a footnote where we verbally explain what a spread measure is, tell that we accept any spread measure that satisfies the very basic pure equality recognition condition and cite examples from the literature satisfying this condition.} However, one of them, called \textit{egalitarian compromise}, insists on equality at the expense of Pareto efficiency while the other, called \textit{Paretian compromise}, is constrained to pick among the Pareto efficient alternatives. 
The two concepts are logically incompatible. As a result, Pareto efficient SCRs cannot ensure egalitarian compromises and this is valid under any spread measure. Moreover, several well-knowm SCRs of the literature such as Condorcet extensions, scoring rules, $q-$approval fall-back bargaining, all fail to be Paretian compromises under any spread measure. In fact, we are able to observe the existence of instances where being a Paretian compromise necessitates to pick an alternative that is, although Pareto optimal, ranked so low by all voters that this alternative wouldn't be picked by any of the popular SCRs of the literature. All these observations make the equal-loss principle appear quite inadequate for collective choice problems, unless envy-freeness is a major concern.
Collective choice models with two individuals present instances where envy-freeness matters. Here, the model is interpreted as a bargaining problem and bargaining procedures replace voting rules. As prominent examples, we have fallback bargaining proposed by Brams and Kilgour (2001) and characterized by Congar and Merlin (2012); the unanimity compromise and the rational compromise introduced by Kibris and Sertel (2007); the veto-rank and short listing procedures analyzed by de Clippel, Eliaz, and Knight (2014) and the Pareto-and-veto rules analyzed by Laslier, Nunez and Sanver (2020). As is ours, these are all models with discrete alternatives which are not contained by the classical Nash (1950) bargaining environment with convex utilities. However, as Mariotti (1998) and Nagahisa and Tanaka (2002) illustrate, the two worlds can be interconnected, as we do for the equal-loss princple of Chun (1988) and Chun and Peters (1991).

examples of envy-freeness 2-person
\begin{example}
	\begin{center}
		$
		\begin{array}{cccccccccc}
		\mathbf{i_1} \quad &x&a_1&\dots&&a_k&y&b_1&\dots&b_k\\
		\mathbf{i_2} \quad &b_1&\dots&b_{k-1}&y&x&b_k&a_1&\dots&a_k\\
		\end{array}
		$
	\end{center}
\end{example}

\commentRS{ I think the "removed material" at the end of the document can all go to trash, except section D on spread measures which we can exploit for the footnote I have suggested above as well as for section 4.4 of the paper.}

computational?

\section{Basic notions and notation}
\label{sec:notation}
Consider a finite set $N$ of individuals with $\#N=n\geq 2$ and a finite set $A$ of alternatives with $\#A=m\geq 3$. We write $\linors$ for the set of linear orders over $A$.
A generic element $\prefi$ of $\linors$ stands for a preference of $i\in N$.%
\footnote{So given any $x ≠ y\in A$, precisely one of $x \prefi y$ and $y\prefi x$ holds while $x \prefi x$ holds for no $x\in A.$ Moreover, $x\prefi y$ and $y\prefi z$ implies $x\prefi z$ $\forall x,y,z\in A$.}
A \emph{profile} $P: N → \linors$ associates with each individual $i \in N$ a preference order  $P(i) = {\prefi}$. A \emph{social choice rule} (SCR) is a mapping $f:\linors^{N}\rightarrow 2^{A} \setminus \{\emptyset \}$. 

We write $r_{\prefi}(x)=\#\{y\in A \suchthat y \prefi x\}+1$ for the \emph{rank} of $x\in A$ at ${\prefi} \in \linors$. We denote by $\lambda_{\prefi}(x)=r_{\prefi}(x)-1$ the loss in terms of ranks for $i\in N$ with preference $\prefi$, when $x$ is elected instead of the best alternative
for $i$. The mapping $\lambda_P: A → \alllosses$ assigns to each $x\in A$ the loss vector $\lambda_{P}(x)=(\lambda_{\prefi}(x))_{i\in N}$ induced by the election of $x$.\footnote{We use double brackets to denote intervals in the integers.}

We are interested in measuring the spread of loss vectors. To this end, we adopt a \emph{spread measure} $\sigma: \alllosses → \R_{+}$ that associates a spread value to every possible loss
vector. We write $\Sigma$ for the set of spread measures $\sigma$ that satisfy for every $l\in\alllosses$, $\sigma(l)=0\iff l_{i}=l_{j}$ $\forall i,j\in N$. Thus, the spread of $l$ gets its lowest value $0$ in case of perfect equality and only in this case. 

Given any distinct $x,y\in A$, we say that $x$ \emph{Pareto dominates} $y$ at $P \in\linors^{N}$ (or equivalenty $y$ is \emph{Pareto dominated} by $x $ at $P$) iff $x\prefi y,\forall i\in N$. We denote
$\paretopt(P)= \set{x \in A \suchthat \forall y ≠ x \in A, \exists i \in N \suchthat x \pref_i y}$ the set of \emph{Pareto optimal} alternatives at $P$.
A SCR $f$ is \emph{Paretian} iff $f(P)\subseteq\paretopt(P)$ $\forall P\in\linors^{N}$.

\section{Egalitarian versus Paretian compromises}
\subsection{Egalitarian compromises}
\label{sec:EgCompromise}
We denote the minimal elements of $X\in2^{A}\setminus \{\emptyset\}$ according to $(\sigma\circ\lambda_{P})$ with $\min_{\sigma \circ \lambda_P} X = \set{x \in X \suchthat \forall y \in A: \sigma(\lambda_P(x)) ≤ \sigma(\lambda_P(y))}$. Thus, $\min_{\sigma\circ\lambda_{P}}(A)$ denotes the alternatives whose loss vectors are the most equally distributed according to the spread measure $\sigma$.

In what follows, we define some classes of SCRs that we are interested in analyzing. 


\begin{definition} A SCR $f$ is an \emph{Egalitarian Compromise} (EC) iff \[\exists \sigma \in \Sigma \suchthat \forall P \in \linors^N \text{ we have }f(P) \subseteq \musigma.\]
\end{definition}

\begin{definition} A SCR $f$ is \emph{Egalitarian Compromise Compatible} (ECC) iff \[\exists \sigma \in \Sigma \suchthat \forall P \in \linors^N \text{ we have } f(P) \cap \musigma \neq \emptyset.\]
\end{definition}

Under a SCR that is EC (resp., ECC), \emph{all} (resp., \emph{some}) winners are among the alternatives with most equally distributed losses. Clearly, EC is a subclass of ECC. Perhaps less obviously, being ECC (or EC) is incompatible with being Paretian. This will be deduced from the following proposition, which will also be useful to prove other theorems.% \cref{th:incompatibility}.


\begin{proposition} \label{prop:muSigmaLast}
	For $n ≥ 2, m ≥ 3$, there exists a profile $P \in \linors^N$ and an alternative $a_m$ such that $\forall i \in N$: $r_{\prefi}(a_m)=m$, and such that $\forall \sigma \in \Sigma: \musigma = \set{a_m}$; hence, $\musigma \cap \paretopt(P) = \emptyset$.
\end{proposition}
\begin{proof}
	Consider the following profile $P$:
	\begin{center}
		$
		\begin{array}{cccccc}
		\mathbf{1} \quad &a_1&a_2&\dots&a_{m-1}&a_m\\
		\mathbf{n-1} \quad &a_{\pi_(1)}&a_{\pi_(2)}&\dots&a_{\pi_(m-1)}&a_m\\
		\end{array}
		$,
	\end{center}
	where $\pi$ is the following permutation over $\intvl{1, m-1}$:
	\[
	\pi(i) = 
	\begin{cases}
	i+1 & \text{if } i \in \intvl{1, m-2} \\
	1 & \text{if } i = m-1
	\end{cases}.
	\]
	In $P$, $a_m$ is the only alternative such that $r_{\prefi}(a_m)=m$, $\forall i \in N$; hence, $\sigma(\lambda_P(a)\geq 0$, $\forall a \in A\setminus \{a_m\}$, $\forall \sigma \in \Sigma$. Thus, the set $\musigma$ consists of the sole element $a_m$, and, because $a_m$ is Pareto dominated, $\musigma \cap \paretopt(P) = \emptyset$.
\end{proof}

Our main result for \cref{sec:EgCompromise} follows easily.
\begin{theorem} \label{th:nonParetian}
	For $n\geq 2, \ m\geq3$, no Paretian SCR is ECC.
\end{theorem}
\begin{proof}
	Proving this amounts to show that $\forall \sigma \in \Sigma, \exists P \in \linors^N \suchthat \paretopt(P) \cap \musigma = \emptyset$. Suffices to use \cref{prop:muSigmaLast}, which asserts that there exists a profile $P$ such that $\forall \sigma \in \Sigma: \musigma \cap \paretopt(P) = \emptyset$.
\end{proof}

\subsection{Paretian compromises}
Having seen the tension for a SCR being Paretian and ECC, we investigate the consequences of inverting the order of priorities by insisting that at least some of the winning alternatives are Pareto optimal, and considering the most equally distributed loss vectors among those.

We consider two classes of SCRs. 
Observe that $\mustar$ denotes the set of Pareto optimal alternatives whose loss vectors are the most equally distributed according to the spread measure $\sigma$.

\begin{definition} A SCR $f$ is a \emph{Paretian Compromise} PC iff \[\exists \sigma \in \Sigma \suchthat \forall P \in \linors^N \text{ we have } f(P) \subseteq \mustar.\]
\end{definition}

\begin{definition} A SCR $f$ is \emph{Paretian Compromise Compatible} PCC iff \[\exists \sigma \in \Sigma \suchthat \forall P \in \linors^N \text{ we have } f(P) \cap \mustar \neq \emptyset.\]
\end{definition}

Again, it is clear that PC is a subclass of PCC. It will also probably come with no surprise that for a SCR, being PC is incompatible with being ECC, as being PC requires to be Paretian, which permits to use \cref{th:nonParetian}. On the other hand, it is less immediate that being EC is incompatible with
being PCC, because being PCC does not require to be Paretian. This is however true.

\begin{theorem} \label{th:incompatibility} 
	For $n ≥ 2, m ≥ 3$, no SCR is both EC and PCC.
\end{theorem}
\begin{proof}	
	Letting $\Ptop$ denote the profile of \cref{prop:muSigmaLast}, with $a_m$ the alternative mentioned there, and considering any EC $f$ and any $\sigma \in \Sigma$, suffices to prove that $f(\Ptop) \cap {\mustar[\sigma][\Ptop]} = \emptyset$.
	
	First, from \cref{prop:muSigmaLast}, $\set{a_m} \cap \paretopt(\Ptop) = \emptyset$, hence $\set{a_m} \cap {\mustar[\sigma][\Ptop]} = \emptyset$. 
	
	Second, because $f$ is an EC, for some $\sigmatop$, $f(\Ptop) \subseteq {\musigma[\sigmatop][\Ptop]}$. Using \cref{prop:muSigmaLast} again, we see that ${\musigma[\sigmatop][\Ptop]} = \set{a_m}$, hence $f(\Ptop) = \set{a_m}$.
	
	That $f(\Ptop) \cap {\mustar[\sigma][\Ptop]} = \emptyset$ follow from these two facts.
\end{proof}

It is interesting to note that the incompatibility is not complete, however.

\begin{remark}
	For $n ≥ 2$, $m ≥ 3$, there exist SCRs that are both ECC and PCC, such as the SCR that selects the whole set of alternatives at every profile. However, this SCR fails to be Paretian, as it must be for every SCR that is ECC.
\end{remark}


\section{Which SCRs are compromises?}
\label{sec:more2voters}
In this section we assume $n\geq 3$ and leave the analysis of $n=2$ to the
next section.

\subsection{Condorcet consistent rules}

An alternative $x\in A$ is a \textit{Condorcet winner} at $P\in L(A)^{N}$ iff for all $y\in A \setminus \set{x} $, $\#\set{i \in N \suchthat x \prefi y} >\#\set{i \in N \suchthat y \prefi x}$. So each profile admits
either no or a unique Condorcet winner. An SCR $f$ is \textit{Condorcet
consistent} iff $f(P)=$ $\left\{ x\right\} $ at each $P\in L(A)^{N}$ that
admits $x$ as the unique Condorcet winner.

\begin{theorem} \label{th:condorcet}
Let $n\geq 3$ and $m\geq 3$. A Condorcet consistent SCR $f$ is neither ECC nor PCC.
\end{theorem}
\begin{proof}
Consider the following profile $P$, where the dots represent the sequence $a_4$ to $a_m$:
	\begin{center}
		$
		\begin{array}{cccccc}
		\mathbf{n-1} \quad &a_1&a_2&a_3&\dots\\
		\mathbf{1} \quad &a_3&a_2&\dots&a_1\\
		\end{array}
		$.
	\end{center}

Consider any Condorcet consistent SCR $f$, then $f(P)=\{a_1\}$. However, $\musigma=\mustar=\{a_2\}$ $\forall \sigma \in \Sigma$, so there exists a profile $P$ such that both $f(P)\cap \musigma$ and $f(P)\cap \mustar$ are empty.
\end{proof}

Note that Condorcet consistent rules need not be Paretian so the fact that they all fail ECC does not follow from \cref{th:nonParetian}. 

\subsection{Scoring rules}
A \emph{score vector} is an $m-$tuple $w=(w_{1},\dots,$ $w_{m})\in \intvl{0, 1}^{m}$ with $w_{1}=1$, $w_{m}=0$ and $w_{i}\geq w_{i+1}$ $\forall
i\in \intvl{1, m-1}$. A score vector $w$ is \emph{strict} iff $w_{i}>w_{i+1}$ $\forall i\in \intvl{1, m-1}$. \commentBN{Do we still need this? We only use it in the proof of \cref{th:srECC}} \commentRS{ I agree. We don't need to define it here. We can refer to the concept within the proof of Theorem 4.}
Given a score vector $w$, we write $s^{w}(x,P)=\sum_{i\in N}w_{r_{\prefi}(x)}$ for the score of $x\in A$ at $P\in L(A)^{N}$. Every score vector $w$ identifies a \emph{scoring rule} $f^w_n$ defined as $f^w_n(P)=\left\{ x\in A:s^{w}(x,P)\geq s^{w}(y,P) \ \forall y\in A\right\}$ for every $P\in L(A)^{N}$. The score vector $w'$ formed such that $w'_{i}=1 \forall i\in \intvl{0, m-1}$ and $w'_{m}=0$ is called antiplurality score vector.\commentRS{I think we can carry the definition just before Theorem 5. I am not sure that we should put a special notation (prime) for it}

We first show that no scoring rule is ECC, for any value of $n$ and $m$ at least 3.

\begin{theorem}\label{th:srECC}
Let $n\geq 3$ and $m\geq 3.$ No score vector $w$ induces a scoring rule $f^w_n$ that is ECC.
\end{theorem}
\begin{proof}
Take any score vector $w$. If $w$ is strict, then $f^w$ is Paretian and \cref{th:nonParetian} implies that $f^w$ is not ECC. Now, let $w$ be non-strict. Consider the profile $P$ of \cref{prop:muSigmaLast}. Observe that $\musigma=\{a_m\} \ \forall \sigma \in \Sigma $. However, as $w_{1}>w_{m}$, we have $s^{w}(a_{1},P)>s^{w}(a_{m},P)$ which implies $a_{m}\notin f^{w}(P)$.
\end{proof}


\begin{theorem}
	Let $m\geq 3$ and let $w$ be the antiplurality score vector. The SCR $f_{n}^{w}$ satisfies PCC for all $n\geq 3$.
\end{theorem}

\begin{proof}
	Take any $n\geq 3$ and $m\geq 3$. Let $A=\left\{a_{1},a_{2}, \dots,a_{m}\right\}$. 
	To see that the antiplurality rule $f^w_n$ satisfies PCC, pick $\bar{\sigma }\in \Sigma $ defined for each $(l_{1}, \dots ,l_{n})$ $\in \intvl{0,m-1}^{N}$ as $\bar{\sigma }(l_{1},\dots,l_{n})=1$ if $l_{i}\neq l_{j}$ for some $i,j\in N$ and $\bar{\sigma }(l_{1},\dots,l_{n})=0$ otherwise. Suppose, for a contradiction, $f^w_n$ fails PCC. So, $\mustar[\bar{\sigma }]\cap f^{w}_n(P)= \emptyset$ for some $P\in \linors^{N}$. Note that $\bar{\sigma }(\lambda _{P}(x))=\bar{\sigma }(\lambda _{P}(y)) \forall x, y\in \mustar[\bar{\sigma }]$ holds by definition of $\mustar[\bar{\sigma }]$ and by the choice of $\bar{\sigma }$, for $x\in \mustar[\bar{\sigma }]$, either $\bar{\sigma } (\lambda _{P}(x))=1$ or $\bar{\sigma }(\lambda_{P}(x))=0$. In the former case, $\mustar[\bar{\sigma }]=\paretopt(P)$, and $\mustar[\bar{\sigma }]\cap $ $f^w(P)=\emptyset $ cannot hold, as the antiplurality rule, although not Paretian, never picks only non-Pareto optimal alternatives. So we consider the case $\bar{\sigma } (\lambda _{P}(x))=0$ $\forall x\in \mustar[\bar{\sigma }]$.
	Take any $x\in \mustar[\bar{\sigma }]$. As $\bar{\sigma } (\lambda _{P}(x))=0,$ we have $\lambda _{i}^{P}(x)=\lambda _{i}^{P}(x) \forall i,j\in N$, hence $r_{\succ _{i}}(x)=r_{\succ _{j}}(x)$ $\forall
	i,j\in N$. However, in case $r_{\succ _{i}}(x)\in \intvl{1,m-1}$, we have $x\in f^w(P)$ and in case $r_{\succ _{i}}(x)=m$,we have $x\in f^w(P)$, both cases giving a contradiction.
\end{proof}

It is worth noting that the antiplurality rule $f_{n}^{w}$ is not Paretian, hence fails PC  for all $n\geq 3$. This, can bee seen by picking a unanimous profile $P\in \linors^{N}$ with $a_{1}\prefi a_{2}\prefi \dots \prefi a_{m}$ $\forall i\in N$, where $\mustar=\left\{ a_{1}\right\} \forall \sigma \in \Sigma $ while $f_{n}^{w}(P)=A \setminus \left\{ a_{m}\right\}$.


\begin{theorem}
	Let $m\geq 3.$ Take any score vector $w$ which is not the antiplurality score vector. The SCR $f_{n}^{w}$ fails PCC for some $n\geq 3$.
\end{theorem}

\begin{proof}
	Take any $n\geq 3$ and $m\geq 3$ and a score vector $w$ such that it is not the antiplurality score vector. Therefore, $w_{m-1}<1$. Observe the existence of two natural numbers $n_{1}$, $n_{2}\geq 3$ with $n_{1}\geq m-1$ and $\frac{n_{2}-1}{n_{2}}>w_{m-1}$.
	Let $n=\max \left\{ n_{1,}n_{2}\right\} $ and let $A=\left\{ a_{1}, \dots, a_{m}\right\} $. Take some $P\in L(A)^{N}$ with
	
	\begin{center}
		$
		\begin{array}{ccccccc}
		i = 1 \quad & a_2 & … & a_{m-1} & a_m & a_1\\
		2 ≤ i ≤ m - 2 \quad & a_1 & … & a_{i+1} & a_m & a_i\\
		m - 1 ≤ i ≤ n \quad & a_1 & … & a_{m-2} & a_m & a_{m-1}\\
		\end{array}
		$.
	\end{center}
	
	Note that for every $\sigma \in \Sigma $. we have $\sigma (\lambda_{P}(a_{m}))=0$ while $\sigma (\lambda _{P}(x))>0$ $\forall x\in A \setminus
	\left\{ a_{m}\right\} $. Moreover, $a_{m}\in \paretopt(P)$. Thus, $\mustar=\left\{ a_{m}\right\} $ $\forall \sigma \in \Sigma $. On the
	other hand, $s^{w}(a_{1}; P)=n-1$, $s^{w}(a_{m}; P)=n\cdot w_{m-1}$ and
	as $\frac{n-1}{n}>w_{m-k}$, we have $s^{w}(a_{1}; P)>s^{w}(a_{m};$ $P)$,
	establishing $a_{m}\notin f^{w}(P)$, thus $f^{w}(P)\cap \mustar=\emptyset $ $\forall \sigma \in \Sigma $.
\end{proof}



\subsection{BK-compromises}
\label{sec:BKn3}
Given any $k\in \intvl{1, m}$, we write $n_{k}(x,P)=\#\{i\in
N\mid r_{\prefi}(x)\leq k\}$ for the \emph{$k$-support} that $x$ gets at $P$, that is, the number of individuals for whom the rank of alternative $x\in A$ is lower than or equal to $k$ in the profile $P\in $ $L(A)^{N}$.
Note that $n_{k}(x,P)\in \intvl{1, n}$ is non-decreasing on $k$ and $n_{m}(x,P)=n.$ For each $q\in \intvl{1,n}$, we define $\rho_{q}(x,P)=\min \{k\in \intvl{1,m} \suchthat n_{k}(x,P)\geq q\}$ as the minimal rank $k$ at which the $k$-support that $x$ gets at $P$ is at least $q$. We
write $\rho _{q}(P)=\min_{x\in A} \{\rho _{q}(x,P)\}$ for the minimal rank $k$ at which the $k$-support that some alternative gets at $P$ is at least $q$. \textit{A Brams and Kilgour (BK) compromise with threshold }$q$ is the
SCR $f_{q}$ defined for each $P\in \linors^N$ as $f_{q}(P)=\{x\in A | n_{\rho _{q}(P)}(x,P)\geq n_{\rho _{q}(P)}(y,P)$ $\forall y\in A\}.$

\commentRS{ I think we can split Theorem 7 in two theorems, one for the failure of ECC, other for the satisfaction of PC}

\begin{theorem}
	\label{th:FBn3}
Let $n\geq 3$ and $m\geq 3.$ The BK compromise $f_{n}$ fails ECC but satisfies PC.
\end{theorem}

\begin{proof}
Take any $n\geq 3$ and $m\geq 3.$ As $f_{n}$ is Paretian, it fails ECC by
Proposition 3.1. To see that $f_{n}$ satisfies PC, pick $\bar{\sigma }\in \Sigma $ defined for each $(l_{1}, \dots ,l_{n}) \in \intvl{0,m-1}^N$ as $\bar{\sigma}(l_{1},...,l_{n})=1$ if 
$l_{i}\neq l_{j}$ for some $i,j\in N$ and $\bar{\sigma }(l_{1},\dots,l_{n})=0$ otherwise. 

Suppose, for a contradiction, $f_{n}$ fails PC. So,
there exist some $P\in \linors^{N}$ for which $f_{n}(P) \setminus \mustar[\bar{\sigma }]\neq \emptyset$ . Pick some $x\in f_{n}(P) \setminus \mustar[\bar{\sigma }]$. As $x\in f_{n}(P)$, $x\in \paretopt(P)$ as well. Thus, $x\notin \mustar[\bar{\sigma }]$
because there exists $y\in \paretopt(P)\setminus \left\{ x\right\} $ with  $\bar{\sigma}(\lambda_{P}(y)) < \bar{\sigma }(\lambda_{P}(x))$. Given the choice of $\bar{\sigma}$, we have  $\bar{\sigma}(\lambda_{P}(y))=0$ and $\bar{\sigma}(\lambda_{P}(x))=1$. Thus, $r_{\prefi}(y)=r_{\prefi[j]}(y)$ $\forall i,j\in N$. As $x\in f_{n}(P)$, this contradicts $y\in \paretopt(P)$.

\commentOC{Is it obvious that $x \in f_n(P)$ contradicts $y$ being same-ranked and PO?}
\commentBN{Add $\rho_n$ and let $k$ be the level of $y$, we show that $\rho_n < k$.}
%I prefer Remzi's one this time
%\commentOC{Again, a matter of style, but I prefer to avoid long contradictions, which leads to state a lot of things that may (in a sense) be false. Also, it seems to me that an elaboration for $y$ being not same-ranked would not be superfluous. I suggest this  alternative, as a remplacement of the second paragraph.}
%Considering any $x \in f_n(P)$, let us show that $x \in \mu_{\bar{\sigma}}^*(P)$. Because $x \in \paretopt(P)$, suffices to show that $\forall y \in \paretopt(P)$, $\bar{\sigma}(\lambda_P(y)) ≥ \bar{\sigma}(\lambda_P(x))$. Equivalently, picking any $y$ such that $\bar{\sigma}(\lambda_P(y)) < \bar{\sigma}(\lambda_P(x))$, let us show that $y \notin \paretopt(P)$. Given the choice of $\bar{\sigma}$, we must have $\bar{\sigma}(\lambda_{P}(y))=0$. Thus, $\forall i,j\in N: r_{\succ _{i}}(y)=r_{\succ _{j}}(y)$. As $x\in f_{n}(P)$, this contradicts $y\in PO(P)$.
%Because $\rho_n = \min_i r_{\prefi}(x)$, $\exists i \suchthat r_{\prefi}(y) > \rho_n$ (otherwise $\rho_n$ would be smaller). And, because $y \in \paretopt(P)$, $\exists j \suchthat \rho_n > r_{\pref_j}(y)$. We conclude that $\bar{\sigma}(\lambda_P(y)) ≠ 0$, thus $\bar{\sigma}(\lambda_P(y))  ≥ \bar{\sigma}(\lambda_P(x))$.
\end{proof}

\begin{theorem}
	\label{th:BKthreshold}
	Let $n\geq 3$ and $m\geq 3.$ A BK compromise $f_{q}$ with threshold $q \in \intvl{1, n-1}$ is neither ECC nor PCC.
\end{theorem}
\begin{proof}
	%Take any $n\geq 3$ and $m\geq 3.$ Let $A=\left\{ a_{1},\text{ }a_{2,}...
	%\text{ }a_{m}\right\} $. Pick some $q\in \left\{ 1,...,n\right\} $ and
	%consider the BK compromise $f_{q}$. 
	Consider the following profile $P$, where the dots represent the sequence $a_4$ to $a_m$ (also used in the proof of \cref{th:condorcet}):
	\begin{center}
		$
		\begin{array}{cccccc}
		\mathbf{n-1} \quad &a_1&a_2&a_3&\dots\\
		\mathbf{1} \quad &a_3&a_2&\dots&a_1\\
		\end{array}
		$.
	\end{center}
	We have that $f_{q}(P)=\{a_1\}$, and, because $\sigma(\lambda_P(a_2)) = 0$ and $\sigma(\lambda_P(a_1)) > 0$, neither $\musigma(P)$ nor $\mustar(P)$ contain $a_1$ for any $\sigma \in \Sigma$. 
	%Remzi's proof
	%Take any $n\geq 3$ and $m\geq 3.$ Let $A=\left\{ a_{1},\text{ }a_{2,}...%
	%\text{ }a_{m}\right\} $. Pick some $q\in \left\{ 1,...,n\right\} $ and
	%consider the BK compromise $f_{q}$. Consider the profile $P\in L(A)^{N}$ such that 
	%$a_{1}\succ _{i}a_{2}\succ _{i}...\succ _{i}a_{m}$ $\forall i\in N\diagdown
	%\left\{ n\right\} $ and $a_{\pi (1)}\succ _{n}a_{\pi (2)}\succ _{n}...\succ
	%_{n}a_{\pi (m)}$ where $\pi $ is a bijection on $\left\{ 1,\text{ }2,...,%
	%\text{ }m\right\} $ with $\pi (1)=3$, $\pi (2)=2,\pi (3)=1$, $\pi (i)=i+1$ $%
	%\forall i\in \left\{ 4,...,\text{ }m-1\right\} $ and $\pi (m)=4 $, we have $%
	%f_{q}(P)=\left\{ a_{1}\right\} $ while $\mu _{\sigma }(P)=\mu _{\sigma
	%}^{\ast }(P)=\left\{ a_{2}\right\} $ $\forall \sigma \in \Sigma $.
\end{proof}

\subsection{Restrictions on sigma}
The perfect equality recognition condition we adopt for spread measures, i.e., that the spread gets its lowest value $0$ in case of perfect equality and only in this case, is very basic. Unless this condition is violated, $\Sigma$ is the largest set of spread measures we could conceive. On the other hand, it is possible to let $\Sigma$ shrink by imposing additional conditions over spread measures. Nevertheless, as the satisfaction of PC, PCC, EC, or ECC requires the existence of a spread measure, all of our negative results, namely, those expressed by Theorems 1, 2, 3, 4, 5, and 6 prevail when $\Sigma$ is restricted. In a similar vein, the positive results in Theorems 5 and 7 risk to be lost with additional conditions over spread measures.

%A \emph{spread measure} $\sigma: \alllosses → \R_{+}$ satisfies condition gamma iff  $\sigma (m-3,$ $m-1,m-2,...,$ $m-2)$ <$\sigma(m-2,$ $m-1,...1,$0, $\ 0)$.
%			(\lambda_{P}(y))$ that associates a spread value to every possible loss
%vector. We write On the other hand, $\lambda
%			^{P}(x)=(m-3,$ $m-1,m-2,...,$ $m-2)$ and $\lambda_{P}(y)=(m-2,$ $m-1,,...1,$
%			$0,$ $\ 0)$.

\commentOC{I find the profile totally convincing, but the proposed reasonability condition seems too strong, because it (needlessly, I think) requires to care only about the difference of ranks, and not about the ranks themselves.
Here is an alternative proposal, but it is even more complex. Given a rank-vector $r \in \intvl{1, m}^N$ and $k \in \intvl{1, m}$, let $c^{r → k}_i = (r_i, k)$ if $r_i ≤ k$, and $c^{r → k}_i = (k, r_i)$ otherwise, denote the “$i$-th change to $k$”, representing (as a pair) the change required to bring the $i$th rank to $k$. Let $r → k$ denote the “change to $k$”, defined as the multiset of pairs $\left<c^{r → k}_i\right>_{i \in N}$, representing the changes required to bring $r$ to the constant rank-vector $(k)_{i \in N}$. Let $C$ be the set of multisets of $n$ pairs $(a, b)$ in $\intvl{1, m} × \intvl{1, m}$ such that $a ≤ b$. Given two such change multisets $c, c' \in C$, say that $c$ is a smaller change than $c'$ iff some bijection between $c$ and $c'$ is such that each pair $(a, b)$ in $c$ is included in its corresponding pair $(a', b')$ in $c'$, meaning that $a' ≤ a ≤ b ≤ b'$. Let $\rho(r) = \set{r → k}_{k \in N}$ associate to the rank-vector $r$ the set of “change to $k$” for each $k$. Say that $r$ is more equal than $r'$ iff $\forall c \in \rho(r), \exists c' \in \rho(r') \suchthat c$ is a smaller change than $c'$.
TODO get rid of this and rewrite in terms of dis-satifaction of going from some loss to $k$.}

Given a rank vector $r\in \intvl{1,m}^\N$, we write $
\varepsilon ^{k}(r)=(\left\vert r_{i}-k\right\vert )_{i\in N}$ for the
equalizing vector at level $k\in \intvl{1,m}$. We
write $\bar{\varepsilon ^{k}}(r)$ for the vector obtained by reshuffling the entries of $\varepsilon ^{k}(r)$ in a non-decreasing order.
Given any two rank vectors $r,q\in \intvl{1,m}^\N$ we
say that $r$ is more equal than $q$ iff $\exists k^*\in \intvl{1,m} $ such that $\bar{\varepsilon _{i}^{k^*}}(r)\leq $ $\bar{\varepsilon _{i}^{k}}(q)$ $\forall i\in N$, $\forall
k\in \intvl{1,m} $ while the inequality is strict for
some $i\in N$ at some $k\in \intvl{1,m}$.

A spread measure $\sigma $ is reasonable iff given any $l,l'\in
\intvl{0,m-1}^\N$, we have $\sigma (l)<\sigma
(l')$ whenever $(l_{1}+1,\dots,l_{n}+1)$ is more equal than $(l_{1}'+1,\dots,l_{n}'+1)$.

\begin{theorem}
	Let $m\geq 3.$ When $\Sigma $ is furthermore reasonable, there exist $n\geq 3 $ such that both the antiplurality rule $f^{w}$ and $f_{n}$ fail PCC.
	\end{theorem}

	\begin{proof}
		Take any $m\geq 3$ and let $\Sigma $ be reasonable. Take any $n\geq m$, say $%
		n=m$ without loss of generality. Take some $x,$ $y\in A$ and some $P\in
		L(A)^{N}$ with $r_{\prefi[1]}(x)=m-2$, $r_{\prefi[2]}(x)=m,$ $r_{\prefi}(x)=m-1$ $\forall i\in N \setminus \left\{ 1, 2\right\}, r_{\prefi}(y)=m-i$ $\forall i\in N \setminus \left\{ n\right\}$,	$r_{\prefi[n]}(y)=1$. Moreover, for each $z\in A \setminus \left\{ x,y\right\} $, we have $r_{\prefi[1]}(z)=m$ for some $i\in N$. Note that both $f^{w}$ and $f_{n}$ pick only $y$ at $P$. On the other hand, $\lambda^{P}(x)=(m-3, m-1,m-2,...,m-2)$ and $\lambda_{P}(y)=(m-2, m-1,,...1,0, 0)$. As $\sigma$ is reasonable, $\sigma(\lambda_{P}(x)) < \sigma(\lambda_{P}(y))$ $\forall \sigma \in \Sigma$, implying $y\notin \mustar[\bar{\sigma}]$ $\forall \sigma \in \Sigma$.
	\end{proof}

\begin{proposition}
	We say that a spread measure $\sigma$ satisfies condition C iff we have $\sigma(m-3, m-1, m-2, \dots, m-2) < \sigma(m-2, m-3, \dots, 1, 0, 0)$ for some value of m and n.
\end{proposition}

\commentRS{ We can motivate it through some arguments but more importantly, we should be able to show that several spread measures of the literature (which Beatrice is discovering) satisfy condition C. We can than take the subset of capital sigma that consists of spread measures that satisfy condition C and prove the theorem under this restriction.}
 
\section{Two voters case}
In \cref{sec:more2voters} we focused on the analysis of voting rules when the number of voters involved into the decision process is greater than two. Keeping the notation introduced in \cref{sec:notation}, we consider here the case $n=2$. Two agents express their preference over a set of alternatives $A$, and the goal is to find a common agreement on the alternative to select. This class of problems is often referred to as bargaining problems.
\subsection{The Bargaining problem}
Several authors, through the years, have proposed theories that would predict the compromise that the two voters would reach through negotiation. 
\commentOC{Predict or recommend?}
Each of these theories, also called solutions, satisfies a specific set of axioms.
\\ 
In other words, a solution can be seen as the recommendation that an impartial arbitrator would make, and the axioms represent properties (like fairness, monotonicity etc.) that this recommendation should embed. 
\commentOC{May I argue for not jumping to next line without starting a new paragraph?}

The best-known solution is the one introduced by \cite{Nash1950}, where the compromise is obtained by maximizing the product of utility gains from the disagreement point.
Consider the following axioms:
\begin{itemize}
	\item Pareto-optimality: no change could be made that would lead to an improvement for any voter;
	\item Symmetry: the agreement is invariant under all permutations of the agents;
	\item Scale invariance: a transformation of the utility function (that maintains the same preferences ordering) should not alter the outcome;
	\item Independence of irrelevant alternatives (IIA): if an alternative is judged to be the best compromise for a problem, then it should still be judged best for any subproblem that contains it.
\end{itemize}

The Nash solution is the only one satisfying Pareto-optimality, symmetry, scale invariance, and independence of irrelevant alternatives.

His characterization opened the door to several others. The solution proposed by \cite{Kalai1975}, for example, defines as a compromise the maximal point (above the disagreement point) which maintains the ratios of gains. In other words, the agreement point is defined as the highest utility any agent can obtain given that no agent should receive less than her coordinate of the disagreement point.
The Kalai–Smorodinsky solution satisfies all the above mentioned axioms except for IIA. Instead, a monotonicity requirement is added, which says that no agent should lose from the increase in resources.

A solution that includes both IIA and monotonicity, dropping the condition of scale invariance, is the Egalitarian solution (\cite{Kalai1977}), whose central idea is the one of equal gains.

Other solutions have been proposed in literature, and \cite{Thomson1994} collected most of them.

Questions:
\begin{itemize}
	\item Which axioms EC, ECC, PC and PCC satisfy?
	\item Does this depend on $\sigma$? If yes, can we define classes of spreading measures?
\end{itemize}

\subsection{Which SCRs are still compromises?}

\subsubsection{BK-compromises}
In \cref{sec:introduction}, we use the \cref{ex:ex1} and the \cref{ex:ex2} to define what we call “ex-ante compromises”. We say that all BK-compromises, except fallback bargaining (FB) (where $q=n$), seem to pursuit a goal of compromising but they fail in ensuring an outcome where everyone has effectively compromised. In \cref{sec:BKn3}, we further investigate this matter for $n\geq3$ and $m\geq3$ by proving, in \cref{th:BKthreshold}, that any BK-compromises, except fallback bargaining, is neither ECC nor PCC. Morever, in \cref{th:FBn3}, we prove that fallback bargaining $f_{n}$ fails ECC but satisfies PC.

\begin{Theorem}
	Let $n=2$ and $m\geq3$. The BK-compromise $f_{n}$ is neither ECC nor PCC. 
\end{Theorem}

\begin{proof}
	Since $f_{n}$ is Paretian, it fails ECC by \cref{th:nonParetian}. In order to prove that it also fails PCC, consider the following preference profile $P\in \linors^{N}$:
\begin{center}
	$
	\begin{array}{cccccccccc}
	\mathbf{i_1} \quad &x&a_1&\dots&a_k&y&b_1&\dots&b_k\\
	\mathbf{i_2} \quad &b_1&\dots&b_{k}&x&y&a_1&\dots&a_k\\
	\end{array}
	$
\end{center}
Here, $f_n(P)=\{x\}$ while $\mustar=\{y\}$ $\forall \sigma \in \Sigma$. Therefore $\exists P \in  \linors^{N}$ such that $f_n(P) \cap \mustar = \emptyset$, $\forall \sigma \in \Sigma$.
\end{proof}


Questions:
\begin{itemize}
	\item Let $n=2$ and $m\geq3$. No BK-compromise  is neither ECC nor PCC. [I suspect. ToProve]
	\item Do the results we got for $n\geq3$ for the other SCR considered in section 4 still hold?
\end{itemize}



\newpage

\bibliography{biblio}

\newpage
\appendix
\section{Removed Material}
\subsection{Case m=2}

\begin{remark} \label{rmk:ECCandPC}
	\cref{th:nonParetian} does not hold for $m=2$: in that case, the largest Paretian SCR, $f(P) = \paretopt(P)$, is ECC. In fact, $\forall \sigma, P: \paretopt(P) \cap \musigma ≠ \emptyset$. This is because all profiles with two alternatives have this form, with $k < \frac{n}{2}$:
	\begin{center}
		$
		\begin{array}{ccc}
		\mathbf{k} \quad &a&b\\
		\mathbf{n-k} \quad &b&a\\
		\end{array}.
		$
		\end{center}
		If $k ≥ 1$, there are no Pareto dominated alternatives, so $\paretopt(P) = A$, thus $\paretopt(P) \cap \musigma ≠ \emptyset$. And if $k = 0$, all alternatives are ranked at the same position by every voter, so $\forall \sigma \in \Sigma$, the spread of the loss vectors associated to each alternative is $0$, thus, $\musigma=A$ and, again, $\paretopt(P) \cap \musigma ≠ \emptyset$.
		\end{remark}
		
		\begin{remark}\label{rmk:ECandPCC}
			\cref{th:incompatibility} does not hold for $m=2$. Consider the same SCR of \cref*{rmk:ECCandPC}, $f(P) = \paretopt(P)$, and a spread measure $\sigma' \in \Sigma$ that assign $0$ to a loss vector $l \in \alllosses$ if all its components are equal, $1$ otherwise. Following the same reasoning of \cref*{rmk:ECCandPC}, all profiles with two alternatives have the same form. If $k ≥ 1$, then there are no Pareto dominated alternatives, therefore $f(P)=A$ and $\sigma'(\lambda_P(a))= \sigma'(\lambda_P(b))=1$, therefore  $\musigma[\sigma']=\mustar[\sigma']=A$. If $k = 0$, then $\sigma'(\lambda_P(a))= \sigma'(\lambda_P(b))=0$ so $\musigma[\sigma']=A$. Inevitably one of the two alternatives will be Pareto dominated by the other, but $f(P)\subseteq \musigma[\sigma']$ still holds, therefore $f$ is EC. Furthermore, $f(P) \subseteq \mustar[\sigma']$  so $f$ is also PCC.
			\end{remark}
			
			\begin{remark} If $\sigma$ is reasonable then \cref{th:incompatibility} still does not hold for $m=2$. As we showed in \cref{rmk:ECCandPC}, all profiles with two alternatives have the same form. If $k ≥ 1$ then $\sigma(\lambda_P(a))=(0_1, \dots, 0_k, 1_{k+1}, \dots, 1_n )$ and $\sigma(\lambda_P(b))=(1_1, \dots, 1_k, 0_{k+1}, \dots, 0_n )$. Because $\sigma$ is reasonable then $\sigma(\lambda_{P}(a))< \sigma(\lambda_{P}(b))$ if $k>n-k$, vice-versa if $k<n-k$ \commentBN{not sure if k=n}. Consider $f$ the majority rule, then $f(P)=\{a\}$ if $k>n-k$, $f(P)=\{b\}$ otherwise. Thus, $f(P)\subseteq \musigma$ in both cases, and $\musigma=\mustar$. If $k=0$ then $\sigma(\lambda_{P}(b))< \sigma(\lambda_{P}(a))$, hence $f(P)=\musigma=\mustar$.
			\end{remark}

\subsection{Definitions}
Definition of $\musigma$ and $\mustar$
\[
\musigma = \argmin_{x \in A} \sigma(\lambda_P(x)) = \set{ x \in A \suchthat \forall y \in A: \sigma(\lambda_P(x)) ≤ \sigma(\lambda_P(y)) }.
\]
\[	\mustar  = \argmin_{x \in \paretopt(P)} \sigma(\lambda_P(x)) = \{ x \in \paretopt(P) \ | \ \sigma (\lambda_P(x))\leq \sigma (\lambda_P(y)), \forall y \in \paretopt(P) \} \]

\begin{corollary}
	For $n\geq 2, \ m\geq3,$ $\exists P$ such that $\musigma$ and $\mustar$ are disjoint  $\forall \sigma \in \Sigma$.
\end{corollary}

\begin{definition} A SCR $f$ is an EC$^*$ iff \[f(P) \subseteq \musigma, \ \forall P \in \linors^N, \forall \ \sigma \in \Sigma.\]
\end{definition}

\begin{definition} A SCR $f$ is an PC$^*$ iff \[f(P) \subseteq \mustar, \ \forall P \in \linors^N, \forall \ \sigma \in \Sigma.\]
\end{definition}

\begin{definition} A SCR $f$ is ECC$^*$ iff \[f(P) \cap \musigma \neq 0, \ \forall P \in \linors^N, \forall \ \sigma \in \Sigma.\]
\end{definition}

\begin{definition} A SCR $f$ is PCC$^*$ iff \[ f(P) \cap \mustar \neq 0, \ \forall P \in \linors^N, \forall \ \sigma \in \Sigma.\]
\end{definition}

\begin{proposition}
	For $n=2, m=2$, no SCR is EC*, and no SCR is PC*.
\end{proposition}
\begin{proof}
	Consider the profile such that voter $1$ prefers $a$ to $b$ and voter $2$ prefers $b$ to $a$. Observe that $\lambda(a) = (0, 1)$ and $\lambda(b) = (1, 0)$. Define $\sigma_1$ and $\sigma_2$ as ordering differently $\lambda(a)$ and $\lambda(b)$. To be EC*, $f$ must satisfy $f(P) \subseteq \set{a}$ and $f(P) \subseteq \set{b}$, which is impossible. The same constraints apply to be PC*.
\end{proof}

\begin{remark}
	Observe that $\text{ECC} \centernot \implies \text{PCC}$ and $\text{PCC} \centernot \implies \text{ECC}$.
	In fact, considering \cref{th:nonParetian}, the existence of a spread measure $\sigma$ such that our SCR $f$ picks some alternatives which belong to the set $\musigma$ is not a guarantee that they are also Pareto optimal, thus belonging to the set $\mustar$ (and vice-versa).
\end{remark}

\section{Old scoring rules}

We now analyze whether scoring rules satisfy PC or PCC. Our first result
identifies, for a given score vector $w,$ properties for $n$ and $m$ that
lead to a failure of PCC.

\begin{proposition}
	\label{prop:whenPCCfails} Let $n\geq 3$ and $m\geq 3.$ Take any score vector $w$.
	If $k\cdot n\geq m-1$ for some $k\in \normalfont{\intvl{1, m-2}}$ and $\frac{n-1}{n}>w_{m-k}$ then $f^w_n$ fails PCC.
	\end{proposition}
	\commentOC{Is there hope of generalizing \cref{th:bound} so that $w_k < 1$ implies failure of PCC (in suppl to $w_2 < 1$ implying this failure)? Let’s try to simplify this result, under condition that this generalisation holds.}
	\begin{proof}
		Take any $n\geq 3$ and $m\geq 3$ with $k.n\geq m-1$ for some $k\in \intvl{1,m-2} $. Let $A=\left\{ a_{1},...,a_{m}\right\} $. Consider any score vector $w$ with $\frac{n-1}{n}>w_{m-k}$. Take some $P\in L(A)^{N}$ with $r_{\succ _{i}}(a_{1})=1$ $\forall i\in \left\{ 1,...,%
		\text{ }n-1\right\} $, $r_{\succ _{n}}(a_{1})=m,$ $r_{\succ
			_{i}}(a_{m-k})=m-k$ $\forall i\in N.$ Moreover, for every $x\in A\diagdown
			\left\{ a_{m-k}\right\} $, $r_{\succ _{i}}(x)>m-k$ for some $i\in N$ and $%
			r_{\succ _{j}}(x)\neq r_{\succ _{k}}(x)$ for some $j,k\in N$. \commentRS{Such a P exists but we should argue about this.} By construction of $P$, we have $%
			\mustar=\left\{ a_{m-k}\right\} $ $\forall \sigma \in
			\Sigma $. Note that $s^{w}(a_{1};$ $P)=n-1$ and $s^{w}(a_{m-k};$ $P)=n\cdot
			w_{m-k}$. As $\frac{n-1}{n}>w_{m-k}$, we have $s^{w}(a_{1};$ $%
			P)>s^{w}(a_{m-k};$ $P)$, establishing $a_{m-k}\notin f^{w}(P)$, thus $%
			f^{w}(P)\cap $ $\mustar=\varnothing $ $\forall \sigma \in
			\Sigma $.
			\end{proof}
			
			\commentOC{%
			We could focus on the following proposition (sketch). If $n ≥ m-1$ and $\frac{n-1}{n} > w_{m-1}$, then PCC fails. The proof (reusing the logic adopted by Remzi here above) may use the following profile $P$, if I am not mistaken. The number $i \in \intvl{1, n}$ of the left is the identifier of a voter, voters being numbered from $1$ to $n$. The dots represent the alternatives not displayed in the first and last three ranks (in some arbitrary ordering). We can argue (as done above) that $a_m \in f(P)$ is required to satisfy PCC but $a_m \notin f(P)$ as $s(a_1) = n-1 > s(a_m) = n w_{m-1}$.
			\begin{center}
				$
				\begin{array}{ccccccc}
				i = 1 \quad & a_2 & … & a_{m-1} & a_m & a_1\\
				2 ≤ i ≤ m - 2 \quad & a_1 & … & a_{i+1} & a_m & a_i\\
				m - 1 ≤ i ≤ n \quad & a_1 & … & a_{m-2} & a_m & a_{m-1}\\
				\end{array}
				$.
				\end{center}
				}
				
			Among all the scoring rules for which Proposition \ref{prop:whenPCCfails} holds, of particular interest are the ones defined by a scoring vector $w$ such that $w_2<1$. Such SCRs include Borda count and Plurality. \commentRS{Delete this paragraph in order not to refer to plurality and Borda.}
			\commentOC{I’d say this can be kept as is, as I find this remark interesting.}
			
			\begin{proposition}
			Let $n\geq 3$ and $m\geq 3.$ For any score vector $w$ with $w_{2}<1$ \commentRS{this is redundant}, if $%
			\frac{n-1}{n}>w_{2}$ then $f^w_n$ fails PCC.
			\end{proposition}
			\begin{proof}
				\Cref{prop:whenPCCfails} can be applied by taking $k=m-2$ and observing that $n ≥ 3 ≥ \frac{m-1}{m-2}$.
				\end{proof}
				
				As a matter of fact, Proposition \ref{prop:whenPCCfails} paves the way the
				observe a very large class of scoring rules that fail PCC, when $n$ is
				sufficiently large.
				
				\begin{proposition}
					\label{nlargefailspcc} \bigskip Let $m$ $\geq 3.$ For any score vector $w$
					with $w_{m-k}<1$ for some $k\in \normalfont{\intvl{1, m-2}}$, there
					exist $n$ $\geq 3$ such that $f^w_n$ fails PCC.
					\end{proposition}
					\begin{proof}
						Take any $m$ $\geq 3$ and any $w$ with $w_{m-k}<1$ for some $k\in \intvl{1, m-2}$. Define $n = \max\set*{\ceil*{\frac{m-1}{k}}, \floor*{\frac{1}{1-w_{m-k}}}+1}$. \Cref{prop:whenPCCfails} then applies, because $n ≥ \frac{m-1}{k}$, and because $n > \frac{1}{1-w_{m-k}}$ which in turn implies $1-w_{m-k} > \frac{1}{n}$ and $1-\frac{1}{n} > w_{m-k}$.
						\end{proof}
						
						An immediate corollary to \cref{nlargefailspcc} is about the \textit{antiplurality score vector} $w$ which is defined as $w_i = 1, \forall i \in \intvl{1,m−1}$.
						\commentOC{This is actually an equivalent statement. And I find its interest clearer when written in the second way. Thus, I’d remove the previous proposition and directly prove it using the phrasing involving AP. And write it conversely: if  $f^w$ is PCC for any $n ≥ 3$, then $f^w$ is AP. Let’s write it $f^w_n$, by the way.}
						
						\begin{proposition}
							Let $m ≥ 3$. If $\forall n ≥ 3: f^w_n$ is PCC, then $w$ is the antiplurality score vector.
							\end{proposition}
							
							Thus among the class of scoring rules, the antiplurality family of rules $\set{f^w_n \suchthat n \in \N}$ is the only candidate to ensure the satisfaction of PCC which, in fact, is the case, as we state and show below.
							
							\commentRS{Thus, among all score vectors, antiplurality is the only candidate to induce a scoring rule (called “antipluralty rule” for each $f^w_n$) that satisfies PCC. This is, in fact, the case, as we state and show below.}
							
							\begin{proposition}
								\label{prop:antPCC}
								Let $n\geq 3$ and $m\geq 3$ and let $w$ be the antiplurality score vector. The rule $f^w_n$ satisfies PCC but fails PC.
								\end{proposition}
								\begin{proof}
									\commentOC{Not checked.}
									Take any $n\geq 3$ and $m\geq 3.$ Let $A=\left\{ a_{1},\text{ }a_{2,}...%
									\text{ }a_{m}\right\} $. The antiplurality rule $f^w$ is not Paretian,
									hence fails PC, which can bee seen by picking a unanimous profile $P\in
									L(A)^{N}$ with $a_{1}\succ _{i}a_{2}\succ _{i}...\succ _{i}a_{m}$ $\forall
									i\in N$ where $\mustar=\left\{ a_{1}\right\} $ $\forall
									\sigma \in \Sigma $ while $f^w(P)=A\diagdown \left\{ a_{m}\right\} $.
									
									To see that the antiplurality rule $f^w$ satisfies PCC, pick $\bar{%
										\sigma }\in \Sigma $ defined for each $(l_{1},...,$ \ $l_{n})$ $\in \left\{
										0,1,...,\text{ }m-1\right\} ^{N}$ as $\bar{\sigma }(l_{1},...,$ \ $%
										l_{n})=1$ if $l_{i}\neq l_{j}$ for some $i,j\in N$ and $\bar{\sigma }%
										(l_{1},...,$ \ $l_{n})=0$ otherwise. Suppose, for a contradiction, $f^w$
										fails PCC. So, $\mustar[\bar{\sigma }]\cap f^{w}(P)=\varnothing $ for some $P\in L(A)^{N}$. Note that $\bar{\sigma }(\lambda _{P}(x))=\bar{\sigma }(\lambda _{P}(y)) \forall x, y\in \mustar[\bar{\sigma }]$ holds by definition of $\mustar[\bar{\sigma }]$ and by the choice of $\bar{\sigma }$, for $x\in \mustar[\bar{\sigma }]$, either $\bar{\sigma } (\lambda _{P}(x))=1$ or $\bar{\sigma }(\lambda _{P}(x))=0$. In the
										former case, $\mustar[\bar{\sigma }]=PO(P)$, and $\mustar[\bar{\sigma }]\cap $ $f^w(P)=\emptyset $ cannot hold, as the antiplurality rule, although not Paretian, never picks only non-Pareto optimal alternatives. So we consider the case $\bar{\sigma } (\lambda _{P}(x))=0$ $\forall x\in \mustar[\bar{\sigma }]$.
										Take any $x\in \mustar[\bar{\sigma }]$. As $\bar{\sigma } (\lambda _{P}(x))=0,$ we have $\lambda _{i}^{P}(x)=\lambda _{i}^{P}(x) \forall i,j\in N$, hence $r_{\succ _{i}}(x)=r_{\succ _{j}}(x)$ $\forall
										i,j\in N$. However, in case $r_{\succ _{i}}(x)\in \intvl{1,m-1}$, we have $x\in f^w(P)$ and in case $r_{\succ _{i}}(x)=m$,
										we have $x\in f^w(P)$, both cases giving a contradiction.
										\end{proof}
										
										Our findings on scoring rules lead to the following theorem as a corollary.
										
										\begin{theorem}
											Let $m ≥ 3$. Take any score vector $w$. The scoring rule $f^w_n$ is PCC for any $n ≥ 3$ if and only if $w$ is the antiplurality score vector, i.e., if $f^w_n$ is the antiplurality family of rules.
															\end{theorem}
															
\section{Old section about which SCRs are compromises}
\begin{proposition}
	\commentBN{m,n}All Condorcet consistent rules, all BK compromises, all point runoff rules and all scoring rules fail ECC.
\end{proposition}
\begin{proof}
	For proving this proposition is enough to consider Corollary cref{cor:optECC}, and to observe that all these rules, except Anti-plurality, are Pareto optimal.
	
	In order to prove it for Anti-plurality, consider the profile described in the proof of \cref{th:nonParetian}, where the most equal alternative $z$ is ranked last by all voters. This alternative is the only element of the set $\musigma$ but it is not selected by anti-plurality. \commentBN{Change according to the new profile.}%profile st z is the last and the other are shifted by 1}
\end{proof}

\begin{proposition} For $n\geq3$, $m\geq3$ \commentBN{Generalize for n=2 (classical bargaining setting)} \label{prop:rulesnotPCC}
	\begin{item}
		\item[1)] Condorcet consistent rules are not $PCC$
		\item[2)] BK compromise for $q \in \{1, \dots ,n-1\}$ are not $PCC$
		\item[3)] Point runoff rules (like STV) are not $PCC$
		%		\item[4)] Scoring rules (except Antiplurality) are not $PCC$
	\end{item}
\end{proposition}
\begin{proof}
	Consider the following profile $P$
	\begin{center}
		$
		\begin{array}{ccccc}
		\mathbf{1} \quad &c&b&\dots &a\\
		\mathbf{n-1} \quad &a&b&\dots &c\\		
		\end{array}
		$
	\end{center}
	where we have $m$ alternatives disposed such that only one alternative $b$ is ranked at the same position by every voter, and it is ranked second.
	We can see that, $\forall \sigma \in \Sigma$, $\musigma=\{b\}$, but all Condorcet consistent rules, BK compromise for $q \in \{1, \dots n-1\}$ and ranked runoff rules would select $a$ as the winner. Thus the propositions $1), 2) \text{ and } 3)$ are proved. 
\end{proof}

\noindent Note that these impossibility results prevail for any restrictions $D \subseteq \Sigma$. \commentBN{first comment Remzi}

\begin{proposition}
	\label{th:bound}
	For $n>2$, $m\geq3$ some scoring rules are  $PCC$ \commentBN{Generalize for n=2}
\end{proposition}
\begin{proof}
	\commentBN{proof olivier}
	Let $(w_1, \dots, w_m)$ be a scoring vector such that $w_i$ represents the score given to an alternative ranked $i$-th in some preference order. Consider $1 = w_1 \geq w_2 \geq \dots \geq w_{m} = 0$. Let $s:\{1,\dots,m\}\rightarrow [0,1]$ be a scoring function that assigns to each alternative $x \in A$ a value depending on the rank obtained in each of the preference orders: 
	\[ s(x) = \sum_{i=1}^{m} \alpha^{x}_i w_i \ ,\]
	where $\alpha^{x}_i$ is the number of times that alternative $x$ was ranked in the $i$-th position. Let $f$ be the SCR that selects the alternatives with the highest score $f(P)=\{x \in A | s(x)\geq s(y), \ \forall y \in A\}$.
	Consider again the profile $P$ defined in Proposition \ref{prop:rulesnotPCC}.
	\begin{center}
		$
		\begin{array}{ccccc}
		\mathbf{1} \quad &c&b&\dots &a\\
		\mathbf{n-1} \quad &a&b&\dots &c\\		
		\end{array}
		$
	\end{center}
	We have that $s(a)=n-1$, $s(b)=n\cdot \lambda$, $s(c)=1$. We can observe that $s(a)$ is always greater than $s(c)$ since $n>2$, and that $s(b)\geq s(x)$ for any other alternative $x\in A \setminus \{a,c\}$. Recall that $b$ is the only alternative equally ranked by everyone, so $\sigma(\lambda_P(b))=0$ and $\sigma(\lambda_P(x))>0$, $\forall x \in A \ \{b\}$; thus, $\mustar=\{b\}$. In order for $f$ to be $PCC$, $f(P)$ must contain $b$. This happens if $s(b)\geq s(a)$, therefore if 
	\begin{align}
	n\cdot\lambda &\geq n-1 \\
	\lambda &\geq \frac{n-1}{n} 
	\end{align}
	\commentBN{Can we generalize it for each possible profile P?}
\end{proof}

\begin{proposition}
	\commentBN{for m n check} Anti-plurality is not $PC$ but it is $PCC$.
\end{proposition}
\begin{proof} 
	In order to prove the first statement, consider the following profile $P$
	\begin{center}
		$
		\begin{array}{cccc}
		\mathbf{1} \quad &a&b&c\\
		\mathbf{n-1} \quad &a&b&c\\
		\end{array}
		$
	\end{center}
	The only element of the set $\musigma$ is $a$, but anti-plurality selects $a$ and $b$. So it is not PC.
	
	For proving the second statement, take any profile $P$. Either
	\begin{itemize}
		\item[case 1.] there exists an alternative $x$ ranked at the same position by every voter and it is Pareto Optimal,
		\item[case 2.] or not.
	\end{itemize}
	In the first case $x$ is never last, thus $x\in \mustar $ and $ x \in AP(P)$, where $AP(P)$ is the set of alternatives selected by anti-plurality.
	
	In the second case, consider a spread measure $\bar{\sigma} \in \Sigma$ such that \[
	\bar{\sigma}(l_1, \dots,l_n)= \begin{cases}
	0 \text{ if } l_i=l_j \ \forall i,j \\ 1 \text{ otherwise} \\
	\end{cases}. \]
	Then, $\exists \ y \in AP(P)$ and also $ y \in \mustar[\bar{\sigma}]$.
\end{proof}

\commentRS{ It is worth noting that the negative results are quite robust because they establish the non-existence of a spread measure within the very large class capital sigma. On the other hand, the positive result that anti plurality is PCC may not be very robust and could turn to be negative when capital sigma is restricted by some further conditions over spread measures. This is an issue to be explored. \\ Another issue to be explored is whether FB satisfies PC or PCC. In an earlier version, we had a nice 2-person example about FB, which seems to be lost in this version. Can we find it? Yes:	\bigskip \\ {\color{black}$x$ $a_{1}...a_{k}$ $y$ $b_{1}....b_{k}$ \\	$b_{1}....b_{k-1}$ $y$ $x$ $b_{k}$ $a_{1}...a_{k}$} \bigskip \\
	A further question to explore is whether we can suggest interesting SCRs that satisfy PCC. The particular “bargaining” case of n=2 many deserve specific consideration. \bigskip \\
	If our results don't hold for n=2 we can say that }


\section{Spread Measures}
In our idea of compromise, we aim to determine the “most equally distributed” vector within $\left\{ \lambda_P(x)\right\} _{x\in A}$. We thus need to adopt a spread measure which is a mapping $\sigma \in \SAll: \R_{+}^{N}\longrightarrow \R_{+}$ which satisfies $\sigma(r_i, \dots, r_n)=0 \iff r_i=r_j \forall i,j$. In what follows, we will consider various classes of possible spread measures.

\begin{definition}[Pairwise Pareto dominance]
	\label{def:PPD}
	For all $r$, $s\in \R_{+}^{N}$: 
	\[\left[\left\vert r_{i}-r_{j}\right\vert \leq \left\vert s_{i}-s_{j}\right\vert \forall i, j\in N\right] ⇒ \sigma (r)\leq \sigma (s).\] 
\end{definition}
We write $\SPPd \subseteq \SAll$ for the class of spread measures that satisfy also PPd.
Given a vector $r \in \R^N$ of $n$ elements, some examples of spread measures are the following.
\commentOC{We dropped this because it says that $(0, 2)$ is more equal than $(10^6, 10^6 + 3)$.}

\subsection{Mean Absolute Difference}
\[\sigma_{mad}(r)= \frac{1}{n^2} \sum_{i=1}^{n}\sum_{j=1}^{n}|r_i-r_j| \]
The mean absolute difference is a measure of dispersion calculated as the arithmetic mean of the absolute value of all possible differences. 
\begin{proposition}
	$\sigma_{mad}$ is a spread measure and it satisfies PPd.
\end{proposition}
\begin{proof} 
	In order to be a spread measure as we defined $\sigma_{mad}$ should be zero when all the elements of $r$ are equals, $r_i-r_j = 0 \ \forall i,j\in {1, \dots, n}$. In this case the sum term $\sum_{i=1}^{n}\sum_{j=1}^{n}|r_i-r_j|$ will always be zero, and thus $\sigma_{mad}(r)=0$ for such $r$. Hence $\sigma_{mad}$ is a spread measure.
	In order to prove the satisfaction of PPd we can observe that, given two vectors $r,s \in \R^N$, if \mbox{$|r_{i}-r_{j}| \leq |s_{i}-s_{j}| \ \forall i,j \in N$} then also \[\sum_{i=1}^{n}\sum_{j=1}^{n}|r_i-r_j| \leq \sum_{i=1}^{n}\sum_{j=1}^{n}|s_i-s_j| \]
	The remaining term $\frac{1}{n^2}$ is constant and it is equal for both sides, so $\sigma_{mad}(r) \leq \sigma_{mad}(s)$ and PPd is satisfied.
\end{proof}
\begin{example}
	\label{ex:spreadVectors}
	\begin{align}
	s=(3,3,3,3) \\
	t=(1,2,3,4) \\
	w=(1,3,5,7) \\
	\end{align}
	In order for a function $\sigma$ to be a spread measure it must be that $\sigma(s)=0$, and to satisfy PPd it must be $\sigma(t)<\sigma(w)$.
	
	Compute now $\sigma_{mad}$ for these vectors, we have:
	\begin{align}
	\sigma_{mad}(s) &= \frac{1}{4^2} \cdot 0 =0 \\
	\sigma_{mad}(t) &= \frac{1+2+3+1+1+2+2+1+1+3+2+1}{4^2}=\frac{20}{16} \\
	\sigma_{mad}(w) &= \frac{2+4+6+2+2+4+4+2+2+6+4+2}{4^2}=\frac{40}{16} \\
	\end{align}
	\begin{align}
	\sigma_{mad}(s) = 0 \ \surd \\
	\sigma_{mad}(t) \leq \sigma_{mad}(w) \rightarrow PPd \ \surd \\
	\end{align}
\end{example}




\subsection{Average Absolute Deviation}
\[\sigma_{ad}(r)= \frac{\sum_{i=1}^{n}|r_i-\bar{r}|}{n}\]
The average absolute deviation considers the differences between the values of the vector and their arithmetic mean $\bar{r}$.

\begin{proposition}
	\label{prop:ad}
	$\sigma_{ad}$ is a spread measure and it satisfies PPd.
\end{proposition}
\begin{proof}
	When $r_i=r_j \ \forall i,j\in {1, \dots, n}$ then $\bar{r}=\frac{\sum_{i=1}^{n}r_i}{n}=r_i$ for any $i \in N$. It follows that $\sum_{i=1}^{n}|r_i-\bar{r}|=0$ and $\sigma_{ad}(r)=0$.
	
	In order to prove that $\sigma_{ad}$ satisfies PPd we have to show that:
	
	\begin{align}
	\sigma_{ad}(r) &\leq \sigma_{ad}(s) \\
	\frac{\sum_{i=1}^{n}|r_i-\bar{r}|}{n} &\leq \frac{\sum_{i=1}^{n}|s_i-\bar{s}|}{n} \\
	\sum_{i=1}^{n}|r_i-\bar{r}| &\leq \sum_{i=1}^{n}|s_i-\bar{s}| \\
	\sum_{i=1}^{n}|r_i-\frac{\sum_{j=1}^{n}r_{j}}{n}| &\leq \sum_{i=1}^{n}|s_i-\frac{\sum_{j=1}^{n}s_{j}}{n}| \\
	\sum_{i=1}^{n}|\frac{n \cdot r_i}{n}-\frac{\sum_{j=1}^{n}r_{j}}{n}| &\leq \sum_{i=1}^{n}|\frac{n \cdot s_i}{n}-\frac{\sum_{j=1}^{n}s_{j}}{n}| \\
	\sum_{i=1}^{n}|\frac{\sum_{j=1}^{n}r_i}{n}-\frac{\sum_{j=1}^{n}r_{j}}{n}| &\leq \sum_{i=1}^{n}|\frac{\sum_{j=1}^{n}s_i}{n}-\frac{\sum_{j=1}^{n}s_{j}}{n}| \\
	\sum_{i=1}^{n}|\frac{\sum_{j=1}^{n}r_i - r_j}{n}| &\leq \sum_{i=1}^{n}|\frac{\sum_{j=1}^{n}s_i - s_j}{n}| \\
	\sum_{i=1}^{n}|\sum_{j=1}^{n}r_i - r_j| &\leq \sum_{i=1}^{n}|\sum_{j=1}^{n}s_i - s_j| \\
	\end{align}
	But we can consider that 
	\begin{itemize}
		\item $\sum_{i=1}^{n}|\sum_{j=1}^{n}r_i - r_j| \geq |\sum_{i=1}^{n}\sum_{j=1}^{n}r_i - r_j|$
		\item $\sum_{i=1}^{n}|\sum_{j=1}^{n}s_i - s_j| \leq \sum_{i=1}^{n}\sum_{j=1}^{n}|s_i - s_j|$
	\end{itemize}
	and, therefore, go back to our proof showing that
	\begin{align}
	|\sum_{i=1}^{n}\sum_{j=1}^{n}r_i - r_j| \leq \sum_{i=1}^{n}|\sum_{j=1}^{n}r_i - r_j| &\leq \sum_{i=1}^{n}|\sum_{j=1}^{n}s_i - s_j| \leq \sum_{i=1}^{n}\sum_{j=1}^{n}|s_i - s_j| \\
	|\sum_{i=1}^{n}\sum_{j=1}^{n}r_i - r_j| &\leq \sum_{i=1}^{n}\sum_{j=1}^{n}|s_i - s_j|
	\end{align}
	By assumption	
	\begin{align}
	|r_{i}-r_{j}| &\leq |s_{i}-s_{j}| \ \forall i,j \in N \\ \sum_{i=1}^{n}\sum_{j=1}^{n}|r_i-r_j| &\leq \sum_{i=1}^{n}\sum_{j=1}^{n}|s_i-s_j| \\
	\end{align}
	and by triangle inequality
	\begin{align}
	|\sum_{i=1}^{n}\sum_{j=1}^{n}r_i-r_j| &\leq \sum_{i=1}^{n}\sum_{j=1}^{n}|r_i-r_j| \\
	\end{align}
	it follows that 
	\begin{align}
	|\sum_{i=1}^{n}\sum_{j=1}^{n}r_i-r_j| \leq \sum_{i=1}^{n}\sum_{j=1}^{n}|r_i-r_j| \leq \sum_{i=1}^{n}\sum_{j=1}^{n}|s_i-s_j| \\
	\end{align}
	This concludes the proof.
\end{proof}

Consider the vectors of the Example \ref{ex:spreadVectors}.
If we compute $\sigma_{ad}$ for those vectors we have:
\begin{align}
\sigma_{ad}(s) &= \frac{\sum_{i=1}^{4} |3-3|}{4} =0 \\
\sigma_{ad}(t) &= \frac{|1-2.5|+|2-2.5|+|3-2.5|+|4-2.5|}{4}=\frac{4}{4}=1 \\
\sigma_{ad}(w) &= \frac{|1-4|+|3-4|+|5-4|+|7-4|}{4}=\frac{8}{4}=2 \\
\end{align}
\begin{align}
\sigma_{ad}(s) = 0 \ \surd \\
\sigma_{ad}(t) \leq \sigma_{ad}(w) \rightarrow PPd \ \surd \\
\end{align}

\subsection{Standard Deviation}
\[\sigma_{sd}(r)= \sqrt{\frac{\sum_{i=1}^{n}(r_i-\bar{r})^2}{n}}\]
\begin{proposition}
	$\sigma_{sd}$ is a spread measure and it satisfies PPd.
\end{proposition}
\begin{proof}
	If $r_i=r_j \ \forall i,j\in {1, \dots, n}$ then $\bar{r}=\frac{\sum_{i=1}^{n}r_i}{n}=r_i$ for any $i \in N$, thus the sum will be zero and $\sigma_{sd}(r)=0$.
	\\ To prove that $\sigma_{sd}$ satisfies PPd we need to show that
	\[\sum_{i=1}^{n}(r_i-\bar{r})^2 \leq \sum_{i=1}^{n}(s_i-\bar{s})^2\]
	Since 
	\[(r_i-\bar{r})^2 = |r_i-\bar{r}|^2 \text{ and } (s_i-\bar{s})^2 = |s_i-\bar{s}|^2 \] 
	we can rewrite it as
	\[\sum_{i=1}^{n}|r_i-\bar{r}|^2 \leq \sum_{i=1}^{n}|s_i-\bar{s}|^2\]
	We know that 
	\[ \sum_{i=1}^{n}|r_i-\bar{r}|^2 \leq (\sum_{i=1}^{n}|r_i-\bar{r}|)^2 \text{ and }
	\sum_{i=1}^{n}|s_i-\bar{s}|^2 \leq (\sum_{i=1}^{n}|s_i-\bar{s}|)^2 \]
	indeed, in case $n=2$, $(a-b)^2 = a^2+2ab+b^2$ so $(a-b)^2 \geq a^2+b^2$ only if $2ab > 0$; in our case $a$ and $b$ are the terms of the summation, respectively $|r_1-\bar{r}|$ and $|r_2-\bar{r}|$ (or $|s_1-\bar{s}|$ and $|s_2-\bar{s}|$) so they are always positive. So we can rewrite our expression as 
	\[\sum_{i=1}^{n}|r_i-\bar{r}|^2 \leq \sum_{i=1}^{n}|s_i-\bar{s}|^2  \leq (\sum_{i=1}^{n}|s_i-\bar{s}|)^2 \]
	and if we show that $\sum_{i=1}^{n}|r_i-\bar{r}|^2 \leq (\sum_{i=1}^{n}|s_i-\bar{s}|)^2 $ then the proof is concluded.
	From the proof of Proposition \ref{prop:ad}, we know that
	\begin{align}
	\sum_{i=1}^{n}|r_i-\bar{r}| &\leq \sum_{i=1}^{n}|s_i-\bar{s}| \\
	(\sum_{i=1}^{n}|r_i-\bar{r}|)^2 &\leq (\sum_{i=1}^{n}|s_i-\bar{s}|)^2
	\end{align}
	so we can affirm that 
	\[\sum_{i=1}^{n}|r_i-\bar{r}|^2 \leq (\sum_{i=1}^{n}|r_i-\bar{r}|)^2 \leq (\sum_{i=1}^{n}|s_i-\bar{s}|)^2 \]
	Therefore $\sum_{i=1}^{n}|r_i-\bar{r}|^2 \leq (\sum_{i=1}^{n}|s_i-\bar{s}|)^2 $ and this concludes the proof.
\end{proof}
Consider the vectors of Example \ref{ex:spreadVectors}.
If we compute $\sigma_{sd}$ for those vectors we have:
\begin{align}
\sigma_{sd}(s) &= \sqrt{\frac{\sum_{i=1}^{4} (3-3)^2}{4}}=0 \\
\sigma_{sd}(t) &= \sqrt{\frac{(1-2.5)^2+(2-2.5)^2+(3-2.5)^2+(4-2.5)^2}{4}}=\sqrt{\frac{5}{4}} \\
\sigma_{sd}(w) &= \sqrt{\frac{(1-4)^2+(3-4)^2+(5-4)^2+(7-4)^2}{4}}=\sqrt{\frac{20}{4}}=\sqrt{5} \\
\end{align}
\begin{align}
\sigma_{sd}(s) = 0 \ \surd \\
\sigma_{sd}(t) \leq \sigma_{sd}(w) \rightarrow PPd \ \surd \\
\end{align}

\subsection{Gini Coefficient}
\[\sigma_{G}(r)= \frac{\sum_{i=1}^{n}\sum_{j=1}^{n}|r_i-r_j|}{2 \cdot n \cdot \sum_{i=1}^{n} r_i}\]
\begin{proposition}
	$\sigma_{G}$ is a spread measure but it does not satisfy PPd.
\end{proposition}
\begin{proof}
	When $r_i=r_j \ \forall i,j\in {1, \dots, n}$ then the numerator of $\sigma_{G}$ is $0$, thus $\sigma_{G}(r)=0$. When $|r_{i}-r_{j}| \leq |s_{i}-s_{j}| \ \forall i,j \in N $ then the numerator of $\sigma_{G}(r)$ is less than or equal to the numerator of $\sigma_{G}(s)$ but it could be that the sum of the elements of $s$ is bigger than the one of $r$, resulting into $\sigma_{G}(s)$ having a greater denominator than $\sigma_{G}(r)$ and therefore $\sigma_{G}(s) \leq \sigma_{G}(r)$. To see that consider the following vectors:
	\begin{align}
	t'=(1,1,2,2) \\
	w'=(2,2,3,3) \\
	\end{align}
	\begin{align}
	\sigma_{G}(t') &= \frac{(1+1)\cdot 4}{2 \cdot 4 \cdot 6}=\frac{1}{6} \\
	\sigma_{G}(w') &= \frac{(1+1)\cdot 4}{2 \cdot 4 \cdot 10}=\frac{1}{10} \\
	\end{align}
	$|t'_{i}-t'_{j}| \leq |w'_{i}-w'_{j}| \forall i, j$ but $\sigma_{G}(t') \nleq \sigma_{G}(w')$
\end{proof}

TODO: Give reference to the relevant literature, also to the Pareto dominance condition.

\end{document}
