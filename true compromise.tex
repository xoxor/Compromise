
\documentclass[12pt,notitlepage,a4paper]{article}
%%%%%%%%%%%%%%%%%%%%%%%%%%%%%%%%%%%%%%%%%%%%%%%%%%%%%%%%%%%%%%%%%%%%%%%%%%%%%%%%%%%%%%%%%%%%%%%%%%%%%%%%%%%%%%%%%%%%%%%%%%%%%%%%%%%%%%%%%%%%%%%%%%%%%%%%%%%%%%%%%%%%%%%%%%%%%%%%%%%%%%%%%%%%%%%%%%%%%%%%%%%%%%%%%%%%%%%%%%%%%%%%%%%%%%%%%%%%%%%%%%%%%%%%%%%%
\usepackage{amsfonts}
\usepackage{amsmath}

\setcounter{MaxMatrixCols}{10}
%TCIDATA{OutputFilter=LATEX.DLL}
%TCIDATA{Version=5.50.0.2953}
%TCIDATA{<META NAME="SaveForMode" CONTENT="1">}
%TCIDATA{BibliographyScheme=Manual}
%TCIDATA{Created=Tuesday, August 09, 2005 15:34:58}
%TCIDATA{LastRevised=Thursday, March 07, 2019 13:36:58}
%TCIDATA{<META NAME="GraphicsSave" CONTENT="32">}
%TCIDATA{<META NAME="DocumentShell" CONTENT="Scientific Notebook\Blank with Theorem Tags">}
%TCIDATA{Language=American English}
%TCIDATA{CSTFile=Math.cst}
%TCIDATA{PageSetup=14,14,57,57,0}
%TCIDATA{AllPages=
%H=36
%F=36,\PARA{038<p type="texpara" tag="Body Text" >\hfill \thepage}
%}


\newtheorem{theorem}{Theorem}[section]
\newtheorem{acknowledgement}[theorem]{Acknowledgement}
\newtheorem{algorithm}[theorem]{Algorithm}
\newtheorem{axiom}{Axiom}[section]
\newtheorem{case}{Case}[section]
\newtheorem{claim}{Claim}[section]
\newtheorem{conclusion}{Conclusion}[section]
\newtheorem{condition}{Condition}[section]
\newtheorem{conjecture}{Conjecture}[section]
\newtheorem{corollary}{Corallary}[section]
\newtheorem{criterion}{Criterion}[section]
\newtheorem{definition}{Definition}[section]
\newtheorem{example}{Example}[section]
\newtheorem{exercise}{Exercise}[section]
\newtheorem{lemma}{Lemma}[section]
\newtheorem{notation}{Notation}[section]
\newtheorem{problem}{Problem}[section]
\newtheorem{proposition}{Proposition}[section]
\newtheorem{remark}{Remark}[section]
\newtheorem{solution}{Solution}[section]
\newtheorem{summary}{Summary}[section]
\newenvironment{proof}[1][Proof]{\noindent\textbf{#1.} }{\ \rule{0.5em}{0.5em}}
\begin{document}


\section{\protect\bigskip Posing the problem}

Consider a set $N$ of individuals with $\#N=n\geq 2$ and a set $A$ of
alternatives with $\#A=m\geq 2$. We write $L(A)$ for the set of linear
orders over $A$ and $P_{i}\in L(A)$ stands for a preference of $i\in N$. For
each $k\in \left\{ 1,...,\text{ }m\right\} $, $r_{k}(P_{i})$ is the
alternative ranked at level $k$ by $P_{i}$. An (ordinal) social choice rule
(SCR) is a mapping $f:L(A)^{N}\rightarrow 2^{A}\backslash \{\emptyset \}$.

We call BK-compromises the class of $q-$approval fallback bargaining rules
introduced by Brams and Kilgour (2001). We start by two examples based on an
observation made by Laslier at Buyukada. Consider the following preference
profile $P\in L(A)^{N}$ with $n=100$

$51$ $a$ $b$ $c$

$49$ $c$ $b$ $a$

which represents 51 individuals who prefer $a$ to $b$, $b$ to $c$, hence $a$
to $c$; and 49 individuals who prefer $c$ to $b$, $b$ to $a$, hence $c$ to $a
$. At $P,$ when $q\in \left\{ 1,...,\text{ }\lfloor \frac{n}{2}\rfloor
+1\right\} $, all BK-compromises pick $a$, which does not appear as a
compromise, as, after all, 51 voters reach their best alternative while the
remaining 49 voters have to be contented with their worst one.

This point can be extended to values of $q$ that exceed a majority. Again
for $n=100$ but now with 4 alternatives, take any $q\in \left\{ \lfloor 
\frac{n}{2}\rfloor +2,...,\text{ }n-1\right\} $ and construct the preference
profile

$26$ $\ \ \ \ \ \ \ \ \ \ a$ $b$ $c$ $d$

$25$ $\ \ \ \ \ \ \ \ \ \ c$ $b$ $a$ $d$

$q-51$ $\ \ \ \ d$ $b$ $a$ $c$

$100-q$ $\ \ d$ $a$ $c$ $b$

\bigskip where the corresponding BK-compromise picks $b$. Here, the first
three block of voters compromise among each other but the SCR doesn't seem
to care about the fourth block of $100-q$ voters.

These two examples illustrate that BK-compromises (except fallback
bargaining (FB) where $q=n$) are \textquotedblleft ex-ante
compromises\textquotedblright\ or \textquotedblleft procedural
compromises\textquotedblright , i.e., they impose over individuals a
willingness to compromise but they don't ensure an outcome where everyone
has effectively compromised. So what is a "true compromise"? To define it,
we switch to a utilitarian world.

\section{\protect\bigskip Cardinal compromises}

Let $U(A)$ be the set of real-valued \textquotedblleft
utility\textquotedblright\ functions defined over $A$ with $u_{i}(x)\neq
u_{i}(y)$ $\forall x,y\in A$, $\forall u_{i}\in U$. Note that each $u_{i}\in
U$, induces a unique $P_{i}(u_{i})\in L(A)$. For each $k\in \left\{ 1,...,%
\text{ }m\right\} $, we write $r_{k}(u_{i})$ for $r_{k}(P_{i}(u_{i}))$.

A (cardinal) SCR is a mapping $f:U(A)^{N}\rightarrow 2^{A}\backslash
\{\emptyset \}$. We will not specify whether a SCR is cardinal or ordinal
whenever this is clear from its domain.

We adopt the \textquotedblleft equal loss principle\textquotedblright\ which
Chun (1988, Economics Letters), Chun and Peters (1991, MASS) introduce and
characterize for bargaining problems. It seems that the concept is also used
in bankruptcy and matching problems. For each $u\in U(A)^{N}$, each $x\in A$
and each $i\in N$, let $\lambda _{i}^{u}(x)=u_{i}(r_{1}(u_{i}))-u_{i}(x)$,
and $\lambda ^{u}(x)=(\lambda _{i}^{u}(x))_{i\in N}$.

Unlike bargaining, our problem does not induce a convex space of achievable
utilities. As a result, we cannot ensure the existence of $x\in A$ with $%
\lambda _{i}^{u}(x)=\lambda _{j}^{u}(x)$ for all $i,j\in N$. Hence, in case
we aim to determine the \textquotedblleft most equally
distributed\textquotedblright\ vector within $\left\{ \lambda
^{u}(x)\right\} _{x\in A}$, we need to adopt a spread measure which is a
mapping $\sigma :%
%TCIMACRO{\U{211d} }%
%BeginExpansion
\mathbb{R}
%EndExpansion
_{+}^{N}\longrightarrow 
%TCIMACRO{\U{211d} }%
%BeginExpansion
\mathbb{R}
%EndExpansion
_{+}$. We write $\Sigma $ for the class of spread measures that satisfy the
following pairwise Pareto dominance condition: $\sigma (r)\leq \sigma (s)$
for all $r$, $s\in 
%TCIMACRO{\U{211d} }%
%BeginExpansion
\mathbb{R}
%EndExpansion
_{+}^{N}$ with $\left\vert r_{i}-r_{j}\right\vert \leq \left\vert
s_{i}-s_{j}\right\vert $ $\forall i,$ $j\in N$.

TO DO: Give reference to the relevant literature, also to the Pareto
dominance condition. Give examples of spread measure that do and do not
satisfy the condition.

Given any $\sigma \in \Sigma $, we define a (cardinal) compromise as the SCR 
$C^{\sigma }:U(A)^{N}\rightarrow 2^{A}\backslash \{\emptyset \}$ such that $%
C^{\sigma }(u)=\left\{ x\in A:\sigma (\lambda ^{u}(x))\leq \sigma (\lambda
^{u}(y))\text{ }\forall y\in A\right\} $. Let $PO(u)$ be the set of
alternatives which are Pareto optimal at $u\in U(A)^{N}$. As a matter of
fact, $C^{\sigma }(u)$ can pick alternatives outside $PO(u)$. In fact, $%
C^{\sigma }(u)$ $\cap $ $PO(u)$ can even be empty. To see this, with two
voters and five alternatives, conisder $u$ with $u_{1}(a)=5$, $u_{1}(b)=4$, $%
u_{1}(c)=3$, $u_{1}(d)=2$, $u_{1}(e)=1$, $u_{2}(d)=5$, $u_{2}(c)=4$, $%
u_{2}(b)=3$, $u_{2}(a)=2$, $u_{2}(e)=1$. Note that for any $\sigma \in
\Sigma $, $C^{\sigma }(u)=\left\{ e\right\} $ while $PO(u)=\left\{ a,\text{ }%
b,\text{ }c,\text{ }d\right\} $. Given the spirit of fairness that underlies 
$C^{\sigma }$, picking $e$ at $u$ may have a merit. 

TO DO: Motivate this with a discussion.

However, from now on, we will ensure the Pareto optimality of cardinal
compromises by defining them as $C^{\sigma }(u)=\left\{ x\in PO(u):\sigma
(\lambda ^{u}(x))\leq \sigma (\lambda ^{u}(y))\text{ }\forall y\in A\right\} 
$.

OPEN QUESTION: Which axioms would charaterize $C^{\sigma }$?

\section{Ordinal compromises}

We now address the question of relating $C^{\sigma }$ to ordinal SCRs.

A utility assignment (UA) is a strictly decreasing function $v:\left\{ 1,...,%
\text{ }m\right\} \longrightarrow 
%TCIMACRO{\U{211d} }%
%BeginExpansion
\mathbb{R}
%EndExpansion
$ which assigns utilities to ranks. For each $P_{i}\in L(A)$, $v$ induces a
utility function $v_{P_{i}}$ $\in U(A)$ with $v_{P_{i}}(x)=v(k)$ for each $%
x=r_{k}(P_{i})\in A$. So for every $P\in L(A)^{N}$, $%
v_{P}=(v_{P_{1}},...,v_{P_{n}})$ $\in U(A)^{N}$ is the utility profile that $%
v$ induces from the preference profile $P$.

Given any UA $v$, an ordinal compromise is the SCR $C^{\sigma }\circ
v:L(A)^{N}\rightarrow 2^{A}\backslash \{\emptyset \}$ such that $C^{\sigma
}\circ v(P)=C^{\sigma }(v_{P})$ for every $P\in L(A)^{N}$.

\begin{proposition}
\label{equivalence} Given any $\sigma \in \Sigma $ and any two UAs $v$ and $%
v^{\prime }$, there exists a $\sigma ^{\prime }\in \Sigma $ such that $%
C^{\sigma }\circ v=C^{\sigma ^{\prime }}\circ v^{\prime }$.
\end{proposition}

\begin{proof}
Take any $\sigma \in \Sigma $ and any two UAs $v$ and $v^{\prime }$. Take
any $(l_{1},...,l_{n})\in 
%TCIMACRO{\U{211d} }%
%BeginExpansion
\mathbb{R}
%EndExpansion
_{+}^{n}$ such that for each $i\in \left\{ 1,...,\text{ }n\right\} $, we
have $l_{i}=v^{\prime }(1)-v^{\prime }(r(l_{i}))$ for some $r(l_{i})\in
\left\{ 1,...,\text{ }m\right\} $. We set $\sigma ^{\prime
}(l_{1},...,l_{n})=$ $\sigma (v(1)-v(r(l_{1})),...,$ $v(1)-v(r(l_{n}))).$
Now, take any $P\in L(A)^{N}$. I think, with this setting of $\sigma
^{\prime }$, we can show $C^{\sigma }\circ v(P)=C^{\sigma ^{\prime }}\circ
v^{\prime }(P).$ Of course we also need to show $\sigma ^{\prime }\in \Sigma 
$.
\end{proof}

TO DO: Complete the proof.

\begin{definition}
\label{D1} \bigskip We say that an SCR $f:L(A)^{N}\rightarrow
2^{A}\backslash \{\emptyset \}$ is an ordinal compromise iff there exists $%
\sigma \in \Sigma $ and an UA $v$ such that $C^{\sigma }\circ v=f$.
\end{definition}

Note that by Proposition \ref{equivalence}, Definition \ref{D1} is
equivalent to the apparently following stronger version:

\begin{definition}
\label{D2} We say that an SCR $f:L(A)^{N}\rightarrow 2^{A}\backslash
\{\emptyset \}$ is an ordinal compromise iff for every UA $v$ there exists $%
\sigma \in \Sigma $ such that $C^{\sigma }\circ v=f$.
\end{definition}

Many ordinal SCRs fail to be ordinal compromises. To see this, consider the
following preference profile

$n-1$ $a$ $b$ $c$

$1$ $\ \ \ \ \ \ c$ $b$ $a$

where $C^{\sigma }\circ v$ picks $b$ for any UA $v$ and any $\sigma \in
\Sigma $ while all BK-compromises (except when $q=n$) pick $a$. Moreover,
for any scoring rule (except anti plurality), there is a choice of $n$ where
the scoring rule picks $a$. Finally, all Condorcet consistent rules pick $a$
as well.

Regarding FB, consider the following preference profile with $n=2$ and $%
m=2k+2$

\bigskip

$x$ $a_{1}...a_{k}$ $y$ $b_{1}....b_{k}$

$b_{1}....b_{k-1}$ $y$ $x$ $b_{k}$ $a_{1}...a_{k}$

\bigskip

where under any $\sigma \in \Sigma $, $C^{\sigma }\circ v$ picks $y$ for the
UA $v(k)=n-k+1$ while FB picks $x$. So FB fails Definition \ref{D2}, hence
Definition \ref{D1}.

Note that this example can be generalized to any $n$ by replicating the
first voter as much as one wishes. This illustrates in a clear and simple
way that $C^{\sigma }$ is ready to pay arbitrary amounts of utility
(computed as a total loss of utility over all voters) to satisfy someone's
envy.

To see that anti plurality also fails Definition 2, consider the following
preference profile with $n=12$ and $m=5$

$3$ $.....d$ $a$

$3$ $.....d$ $b$

$3$ $.....d$ $c$

$2$ $......$ $d$

$1$ $.....d$ $e$

\bigskip

where under any $\sigma \in \Sigma $, $C^{\sigma }\circ v$ picks $d$ for the
UA $v(k)=n-k+1$ while anti plurality picks $e$.

We thus observe that all BK-compromises, all scoring rules and all Condorcet
consistent rules fail to be ordinal compromises.

TO DO: See whether the analysis above prevails when the UA $v$ is allowed to
vary among individuals.

The two person example that shows that FB is not an ordinal compromise of
special interest because $C^{\sigma }$ seems to particularly appealing in
two person discrete bargaining environments. In that example, $y$ seems to
be a more natural bargaining solution than $x$. So we can try to present $%
C^{\sigma }$ as a discrete bargaining solution. In this literature there is
FB of Brams and Kilgour which is later rectified by Kibris and Sertel by
adding the disagreement outcome that was originally missing. It will be
certainly interesting to introduce and analyze $C^{\sigma }$ in this ordinal
context where ranks are utilities. Clearly, minimized envy will be a key
condition which will differentiate $C^{\sigma }$ from FB.

Another direction of research to which Ali had pointed: As a spirit of
compromise, we pick the alternative whose lambda is \textquotedblleft most
equally distributed\textquotedblright . But once we have lambdas, it makes
sense to ask what else we can do with these. Ali suggested to minimize the
median value of lambda. Sounds plausible but what does this mean? What else
can we do with our lambdas? I feel that there are plenty of interesting
questions here.

\bigskip

\end{document}
