\RequirePackage[l2tabu, orthodox]{nag}
\documentclass[version=3.21, pagesize, twoside=off, bibliography=totoc, DIV=calc, fontsize=12pt, a4paper]{scrartcl}
%Permits to copy eg x ⪰ y ⇔ v(x) ≥ v(y) from PDF to unicode data, and to search. From pdfTeX users manual. See https://tex.stackexchange.com/posts/comments/1203887.
	\input glyphtounicode
	\pdfgentounicode=1
%Latin Modern has more glyphs than Computer Modern, such as diacritical characters. fntguide commands to load the font before fontenc, to prevent default loading of cmr.
	\usepackage{lmodern}
%Encode resulting accented characters correctly in resulting PDF, permits copy from PDF.
	\usepackage[T1]{fontenc}
%UTF8 seems to be the default in recent TeX installations, but not all, see https://tex.stackexchange.com/a/370280.
	\usepackage[utf8]{inputenc}
%Provides \newunicodechar for easy definition of supplementary UTF8 characters such as → or ≤ for use in source code.
	\usepackage{newunicodechar}
%Text Companion fonts, much used together with CM-like fonts. Provides \texteuro and commands for text mode characters such as \textminus, \textrightarrow, \textlbrackdbl.
	\usepackage{textcomp}
%St Mary’s Road symbol font, used for ⟦ = \llbracket.
	\usepackage{stmaryrd}
\usepackage{centernot}
%Solves bug in lmodern, https://tex.stackexchange.com/a/261188; probably useful only for unusually big font sizes; and probably better to use exscale instead. Note that the authors of exscale write against this trick.
	%\DeclareFontShape{OMX}{cmex}{m}{n}{
		%<-7.5> cmex7
		%<7.5-8.5> cmex8
		%<8.5-9.5> cmex9
		%<9.5-> cmex10
	%}{}
	%\SetSymbolFont{largesymbols}{normal}{OMX}{cmex}{m}{n}
%More symbols (such as \sum) available in bold version, see https://github.com/latex3/latex2e/issues/71.
	\DeclareFontShape{OMX}{cmex}{bx}{n}{%
	   <->sfixed*cmexb10%
	   }{}
	\SetSymbolFont{largesymbols}{bold}{OMX}{cmex}{bx}{n}
%For small caps also in italics, see https://tex.stackexchange.com/questions/32942/italic-shape-needed-in-small-caps-fonts, https://tex.stackexchange.com/questions/284338/italic-small-caps-not-working.
	\usepackage{slantsc}
	\AtBeginDocument{%
		%“Since nearly no font family will contain real italic small caps variants, the best approach is to substitute them by slanted variants.” -- slantsc doc
		%\DeclareFontShape{T1}{lmr}{m}{scit}{<->ssub*lmr/m/scsl}{}%
		%There’s no bold small caps in Latin Modern, we switch to Computer Modern for bold small caps, see https://tex.stackexchange.com/a/22241
		%\DeclareFontShape{T1}{lmr}{bx}{sc}{<->ssub*cmr/bx/sc}{}%
		%\DeclareFontShape{T1}{lmr}{bx}{scit}{<->ssub*cmr/bx/scsl}{}%
	}
%Warn about missing characters.
	\tracinglostchars=2
%Nicer tables: provides \toprule, \midrule, \bottomrule.
	%\usepackage{booktabs}
%For new column type X which stretches; can be used together with booktabs, see https://tex.stackexchange.com/a/97137. “tabularx modifies the widths of the columns, whereas tabular* modifies the widths of the inter-column spaces.” Loads array.
	%\usepackage{tabularx}
%math-mode version of "l" column type. Requires \usepackage{array}.
	%\usepackage{array}
	%\newcolumntype{L}{>{$}l<{$}}
%Provides \xpretocmd and loads etoolbox which provides \apptocmd, \patchcmd, \newtoggle… Also loads xparse, which provides \NewDocumentCommand and similar commands intended as replacement of \newcommand in LaTeX3 for defining commands (see https://tex.stackexchange.com/q/98152 and https://github.com/latex3/latex2e/issues/89).
	\usepackage{xpatch}
%ntheorem doc says: “empheq provides an enhanced vertical placement of the endmarks”; must be loaded before ntheorem. Loads the mathtools package, which loads and fixes some bugs in amsmath and provides \DeclarePairedDelimiter. amsmath is considered a basic, mandatory package nowadays (Grätzer, More Math Into LaTeX).
	\usepackage[ntheorem]{empheq}
%Package frenchb asks to load natbib before babel-french. Package hyperref asks to load natbib before hyperref.
	\usepackage{natbib}

\newtoggle{LCpres}
	\newtoggle{LCart}
	\newtoggle{LCposter}
	\makeatletter
	\@ifclassloaded{beamer}{
		\toggletrue{LCpres}
		\togglefalse{LCart}
		\togglefalse{LCposter}
		\wlog{Presentation mode}
	}{
		\@ifclassloaded{tikzposter}{
			\toggletrue{LCposter}
			\togglefalse{LCpres}
			\togglefalse{LCart}
			\wlog{Poster mode}
		}{
			\toggletrue{LCart}
			\togglefalse{LCpres}
			\togglefalse{LCposter}
			\wlog{Article mode}
		}
	}
	\makeatother%

%Language options ([french, english]) should be on the document level (last is main); except with tikzposter: put [french, english] options next to \usepackage{babel} to avoid warning. beamer uses the \translate command for the appendix: omitting babel results in a warning, see https://github.com/josephwright/beamer/issues/449. Babel also seems required for \refname.
	\iftoggle{LCpres}{
		\usepackage{babel}
	}{
	}
	%\frenchbsetup{AutoSpacePunctuation=false}
%listings (1.7) does not allow multi-byte encodings. listingsutf8 works around this only for characters that can be represented in a known one-byte encoding and only for \lstinputlisting. Other workarounds: use literate mechanism; or escape to LaTeX (but breaks alignment).
	%\usepackage{listings}
	%\lstset{tabsize=2, basicstyle=\ttfamily, escapechar=§, literate={é}{{\'e}}1}
%I favor acro over acronym because the former is more recently updated (2018 VS 2015 at time of writing); has a longer user manual (about 40 pages VS 6 pages if not counting the example and implementation parts); has a command for capitalization; and acronym suffers a nasty bug when ac used in section, see https://tex.stackexchange.com/q/103483 (though this might be the fault of the silence package and might be solved in more recent versions, I do not know) and from a bug when used with cleveref, see https://tex.stackexchange.com/q/71364. However, loading it makes compilation time (one pass on this template) go from 0.6 to 1.4 seconds, see https://bitbucket.org/cgnieder/acro/issues/115. Option short-format not usable in the package options as it is fragile, see https://tex.stackexchange.com/q/466882.
	\usepackage[single]{acro}
	%\acsetup{short-format = {\MakeUppercase}}
	\DeclareAcronym{AMCD}{short=amcd, long={Aide Multicritère à la Décision}}
\DeclareAcronym{AR}{short=ar, long={Argumentative Recommender}}
\DeclareAcronym{DA}{short=da, long={Decision Analysis}}
\DeclareAcronym{DJ}{short=dj, long={Deliberated Judgment}}
\DeclareAcronym{DM}{short=dm, long={Decision Maker}}
\DeclareAcronym{DP}{short=dp, long={Deliberated Preference}}
\DeclareAcronym{MAVT}{short=mavt, long={Multiple Attribute Value Theory}}
\DeclareAcronym{MCDA}{short=mcda, long={Multicriteria Decision Aid}}
\DeclareAcronym{MIP}{short=mip, long={Mixed Integer Program}}


\iftoggle{LCpres}{
	%I favor fmtcount over nth because it is loaded by datetime anyway; and fmtcount warns about possible conflicts when loaded after nth.
	\usepackage{fmtcount}
	%For nice input of date of presentation. Must be loaded after the babel package. Has possible problems with srcletter: https://golatex.de/verwendung-von-babel-und-datetime-in-scrlttr2-schlaegt-fehlt-t14779.html.
	\usepackage[nodayofweek]{datetime}
}{
}
%For presentations, Beamer implicitely uses the pdfusetitle option. ntheorem doc says to load hyperref “before the first use of \newtheorem”. autonum doc mandates option hypertexnames=false. I want to highlight links only if necessary for the reader to recognize it as a link, to reduce distraction. In presentations, this is already taken care of by beamer (https://tex.stackexchange.com/a/262014). If using colorlinks=true in a presentation, see https://tex.stackexchange.com/q/203056. Crashes the first compilation with tikzposter, just compile again and the problem disappears, see https://tex.stackexchange.com/q/254257.
\makeatletter
\iftoggle{LCpres}{
	\usepackage{hyperref}
}{
	\usepackage[hypertexnames=false, pdfusetitle, linkbordercolor={1 1 1}, citebordercolor={1 1 1}, urlbordercolor={1 1 1}]{hyperref}
	%https://tex.stackexchange.com/a/466235
	\pdfstringdefDisableCommands{%
		\let\thanks\@gobble
	}
}
\makeatother
%urlbordercolor is used both for \url and \doi, which I think shouldn’t be colored, and for \href, thus might want to color manually when required. Requires xcolor.
	\NewDocumentCommand{\hrefblue}{mm}{\textcolor{blue}{\href{#1}{#2}}}
%hyperref doc says: “Package bookmark replaces hyperref’s bookmark organization by a new algorithm (...) Therefore I recommend using this package”.
	\usepackage{bookmark}
%Need to invoke hyperref explicitly to link to line numbers: \hyperlink{lintarget:mylinelabel}{\ref*{lin:mylinelabel}}, with \ref* to disable automatic link. Also see https://tex.stackexchange.com/q/428656 for referencing lines from another document.
	%\usepackage{lineno}
	%\NewDocumentCommand{\llabel}{m}{\hypertarget{lintarget:#1}{}\linelabel{lin:#1}}
	%\setlength\linenumbersep{9mm}
%For complex authors blocks. Seems like authblk wants to be later than hyperref, but sooner than silence. See https://tex.stackexchange.com/q/475513 for the patch to hyperref pdfauthor.
	\ExplSyntaxOn
	\seq_new:N \g_oc_hrauthor_seq
	\NewDocumentCommand{\addhrauthor}{m}{
		\seq_gput_right:Nn \g_oc_hrauthor_seq { #1 }
	}
	%Should be \NewExpandableDocumentCommand, but this is not yet provided by my version of xparse
	\DeclareExpandableDocumentCommand{\hrauthor}{}{
		\seq_use:Nn \g_oc_hrauthor_seq {,~}
	}
	\ExplSyntaxOff
	{
		\catcode`#=11\relax
		\gdef\fixauthor{\xpretocmd{\author}{\addhrauthor{#2}}{}{}}%
	}
	\iftoggle{LCart}{
		\usepackage{authblk}
		\renewcommand\Affilfont{\small}
		\fixauthor
		\AtBeginDocument{
		    \hypersetup{pdfauthor={\hrauthor}}
		}
	}{
	}
%I do not use floatrow, because it requires an ugly hack for proper functioning with KOMA script (see scrhack doc). Instead, the following command centers all floats (using \centering, as the center environment adds space, http://texblog.net/latex-archive/layout/center-centering/), and I manually place my table captions above and figure captions below their contents (https://tex.stackexchange.com/a/3253).
	\makeatletter
	\g@addto@macro\@floatboxreset\centering
	\makeatother
%Permits to customize enumeration display and references
	%\nottoggle{LCpres}{
		\usepackage{enumitem} %follow list environments by a string to customize enumeration, example: \begin{description}[itemindent=8em, labelwidth=!] or \begin{enumerate}[label=({\roman*}), ref={\roman*}].
	%}{
	%}
%Provides \Cen­ter­ing, \RaggedLeft, and \RaggedRight and en­vi­ron­ments Cen­ter, FlushLeft, and FlushRight, which al­low hy­phen­ation. With tikzposter, seems to cause 1=1 to be printed in the middle of the poster.
	%\usepackage{ragged2e}
%To typeset units by closely following the “official” rules.
	%\usepackage[strict]{siunitx}
%Turns the doi provided by some bibliography styles into URLs. However, uses old-style dx.doi url (see 3.8 DOI system Proxy Server technical details, “Users may resolve DOI names that are structured to use the DOI system Proxy Server (https://doi.org (current, preferred) or earlier syntax http://dx.doi.org).”, https://www.doi.org/doi_handbook/3_Resolution.html). The patch solves this.
	\usepackage{doi}
	\makeatletter
	\patchcmd{\@doi}{http://dx.doi.org}{https://doi.org}{}{}
	\makeatother
%Makes sure upper case greek letters are italic as well.
	\usepackage{fixmath}
%Provides \mathbb; obsoletes latexsym (see http://tug.ctan.org/macros/latex/base/latexsym.dtx). Relatedly, \usepackage{eucal} to change the mathcal font and \usepackage[mathscr]{eucal} (apparently equivalent to \usepackage[mathscr]{euscript}) to supplement \mathcal with \mathscr. This last option is not very useful as both fonts are similar, and the intent of the authors of eucal was to provide a replacement to mathcal (see doc euscript). Also provides \mathfrak for supplementary letters.
	\usepackage{amsfonts}
%Provides a beautiful (IMHO) \mathscr and really different than \mathcal, for supplementary uppercase letters. But there is no bold version. Alternative: mathrsfs (more slanted), but when used with tikzposter, it warns about size substitution, see https://tex.stackexchange.com/q/495167.
	\usepackage[scr]{rsfso}
%Multiple means to produce bold math: \mathbf, \boldmath (defined to be \mathversion{bold}, see fntguide), \pmb, \boldsymbol (all legacy, from LaTeX base and AMS), \bm (the most recommended one), \mathbold from package fixmath (I don’t see its advantage over \boldsymbol).
%“The \boldsymbol command is obtained preferably by using the bm package, which provides a newer, more powerful version than the one provided by the amsmath package. Generally speaking, it is ill-advised to apply \boldsymbol to more than one symbol at a time.” — AMS Short math guide. “If no bold font appears to be available for a particular symbol, \bm will use ‘poor man’s bold’” — bm. It is “best to load the package after any packages that define new symbol fonts” – bm. bm defines \boldsymbol as synonym to \bm. \boldmath accesses the correct font if it exists; it is used by \bm when appropriate. See https://tex.stackexchange.com/a/10643 and https://github.com/latex3/latex2e/issues/71 for some difficulties with \bm.
	\usepackage{bm}
	\nottoggle{LCpres}{
	%https://ctan.org/pkg/amsmath recommends ntheorem, which supersedes amsthm, which corrects the spacing of proclamations and allows for theoremstyle. Option standard loads amssymb and latexsym. Must be loaded after amsmath (from ntheorem doc). From cleveref doc, “ntheorem is fully supported and even recommended”; says to load cleveref after ntheorem. When used with tikzposter, warns about size substitution for the lasy (latexsym) font when using \url, because ntheorem loads latexsym; relatedly (but not directly related to ntheorem), size substitution warning with the cmex font happens when loading amsmath and using \url.
		\usepackage[thmmarks, amsmath, standard, hyperref]{ntheorem}
		%empheq doc says to do this after loading ntheorem
		\usetagform{default}
	%Provides \cref. Unfortunately, cref fails when the language is French and referring to a label whose name contains a colon (https://tex.stackexchange.com/q/83798). Use \cref{sec\string:intro} to work around this. cleveref should go “laster” than hyperref.
		\usepackage[capitalise]{cleveref}
	}{
	}
	\nottoggle{LCposter}{
	%Equations get numbers iff they are referenced. Loading order should be “amsmath → hyperref → cleveref → autonum”, according to autonum doc. Use this in preference to the showonlyrefs option from mathtools, see https://tex.stackexchange.com/q/459918 and autonum doc. See https://tex.stackexchange.com/a/285953 for the etex line. Incompatible with my version of tikzposter (produces “! Improper \prevdepth”).
		\expandafter\def\csname ver@etex.sty\endcsname{3000/12/31}\let\globcount\newcount
		\usepackage{autonum}
	}{
	}
%Also loaded by tikz.
	\usepackage{xcolor}
\iftoggle{LCpres}{
	\usepackage{tikz}
	%\usetikzlibrary{babel, matrix, fit, plotmarks, calc, trees, shapes.geometric, positioning, plothandlers, arrows, shapes.multipart}
}{
}
%Vizualization, on top of TikZ
	%\usepackage{pgfplots}
	%\pgfplotsset{compat=1.14}
\usepackage{graphicx}
	\graphicspath{{graphics/}}

%Provides \print­length{length}, useful for debugging.
	%\usepackage{printlen}
	%\uselengthunit{mm}

\iftoggle{LCpres}{
	\usepackage{appendixnumberbeamer}
	%I have yet to see anyone actually use these navigation symbols; let’s disable them
	\setbeamertemplate{navigation symbols}{} 
	\usepackage{preamble/beamerthemeParisFrance}
	\setcounter{tocdepth}{10}
}{
}

%Do not use the displaymath environment: use equation. Do not use the eqnarray or eqnarray* environments: use align(*). This improves spacing. (See l2tabu or amsldoc.)


%Requires package xcolor.
\newcommand{\commentOC}[1]{\textcolor{blue}{\small$\big[$OC: #1$\big]$}}
%Requires package babel and option [french]. According to babel doc, need two braces around \selectlanguage to make the changes really local.
\newcommand{\commentOCf}[1]{\textcolor{blue}{{\small\selectlanguage{french}$\big[$OC : #1$\big]$}}}
\newcommand{\commentYM}[1]{\textcolor{red}{\small$\big[$YM: #1$\big]$}}
\newcommand{\commentYMf}[1]{\textcolor{red}{{\small\selectlanguage{french}$\big[$YM : #1$\big]$}}}

\bibliographystyle{abbrvnat}

%https://tex.stackexchange.com/a/467188 - uncomment if one of those symbols is used.
%\DeclareFontFamily{U} {MnSymbolD}{}
%\DeclareFontShape{U}{MnSymbolD}{m}{n}{
%  <-6> MnSymbolD5
%  <6-7> MnSymbolD6
%  <7-8> MnSymbolD7
%  <8-9> MnSymbolD8
%  <9-10> MnSymbolD9
%  <10-12> MnSymbolD10
%  <12-> MnSymbolD12}{}
%\DeclareFontShape{U}{MnSymbolD}{b}{n}{
%  <-6> MnSymbolD-Bold5
%  <6-7> MnSymbolD-Bold6
%  <7-8> MnSymbolD-Bold7
%  <8-9> MnSymbolD-Bold8
%  <9-10> MnSymbolD-Bold9
%  <10-12> MnSymbolD-Bold10
%  <12-> MnSymbolD-Bold12}{}
%\DeclareSymbolFont{MnSyD} {U} {MnSymbolD}{m}{n}
%\DeclareMathSymbol{\ntriplesim}{\mathrel}{MnSyD}{126}
%\DeclareMathSymbol{\nlessgtr}{\mathrel}{MnSyD}{192}
%\DeclareMathSymbol{\ngtrless}{\mathrel}{MnSyD}{193}
%\DeclareMathSymbol{\nlesseqgtr}{\mathrel}{MnSyD}{194}
%\DeclareMathSymbol{\ngtreqless}{\mathrel}{MnSyD}{195}
%\DeclareMathSymbol{\nlesseqgtrslant}{\mathrel}{MnSyD}{198}
%\DeclareMathSymbol{\ngtreqlessslant}{\mathrel}{MnSyD}{199}
%\DeclareMathSymbol{\npreccurlyeq}{\mathrel}{MnSyD}{228}
%\DeclareMathSymbol{\nsucccurlyeq}{\mathrel}{MnSyD}{229}

%03B3 Greek Small Letter Gamma
\newunicodechar{γ}{\gamma}
%03B4 Greek Small Letter Delta
\newunicodechar{δ}{\delta}
%2115 Double-Struck Capital N
\newunicodechar{ℕ}{\mathbb{N}}
%211D Double-Struck Capital R
\newunicodechar{ℝ}{\mathbb{R}}
%21CF Rightwards Double Arrow with Stroke
\newunicodechar{⇏}{\nRightarrow}
%21D2 Rightwards Double Arrow
\newunicodechar{⇒}{\ensuremath{\Rightarrow}}
%21D4 Left Right Double Arrow
\newunicodechar{⇔}{\Leftrightarrow}
%21DD Rightwards Squiggle Arrow
\newunicodechar{⇝}{\rightsquigarrow}
%2212 Minus Sign
\newunicodechar{−}{\ifmmode{-}\else\textminus\fi}
%2227 Logical And
\newunicodechar{∧}{\land}
%2228 Logical Or
\newunicodechar{∨}{\lor}
%2229 Intersection
\newunicodechar{∩}{\cap}
%222A Union
\newunicodechar{∪}{\cup}
%2260 Not Equal To (handy also as text in informal writing)
\newunicodechar{≠}{\ensuremath{\neq}}
%2264 Less-Than or Equal To
\newunicodechar{≤}{\leq}
%2265 Greater-Than or Equal To
\newunicodechar{≥}{\geq}
%2270 Neither Less-Than nor Equal To
\newunicodechar{≰}{\nleq}
%2271 Neither Greater-Than nor Equal To
\newunicodechar{≱}{\ngeq}
%2272 Less-Than or Equivalent To
\newunicodechar{≲}{\lesssim}
%2273 Greater-Than or Equivalent To
\newunicodechar{≳}{\gtrsim}
%2274 Neither Less-Than nor Equivalent To – also, from MnSymbol: \nprecsim, a more exact match to the Unicode symbol; and \npreccurlyeq, too small
\newunicodechar{≴}{\not\preccurlyeq}
%2275 Neither Greater-Than nor Equivalent To
\newunicodechar{≵}{\not\succcurlyeq}
%2279 Neither Greater-Than nor Less-Than – requires MnSymbol; also \nlessgtr from txfonts/pxfonts, \ngtreqless from MnSymbol (but much higher), \ngtrless from MnSymbol (a more exact match to the Unicode symbol); for incomparability (not matching this Unicode symbol), may also consider \ntriplesim from MnSymbol,\nparallelslant from fourier, \between from mathabx, or ⋈
\newunicodechar{≹}{\ngtreqlessslant}
%227A Precedes
\newunicodechar{≺}{\prec}
%227B Succeeds
\newunicodechar{≻}{\succ}
%227C Precedes or Equal To
\newunicodechar{≼}{\preccurlyeq}
%227D Succeeds or Equal To
\newunicodechar{≽}{\succcurlyeq}
%227E Precedes or Equivalent To
\newunicodechar{≾}{\precsim}
%227F Succeeds or Equivalent To
\newunicodechar{≿}{\succsim}
%2280 Does Not Precede
\newunicodechar{⊀}{\nprec}
%2281 Does Not Succeed
\newunicodechar{⊁}{\nsucc}
%22B2 Normal Subgroup Of – \triangleleft is too small compared to \trianglelefteq and the like
\newunicodechar{⊲}{\lhd}
%22B3 Contains as Normal Subgroup
\newunicodechar{⊳}{\rhd}
%22B4 Normal Subgroup of or Equal To
\newunicodechar{⊴}{\trianglelefteq}
%22B5 Contains as Normal Subgroup or Equal To
\newunicodechar{⊵}{\trianglerighteq}
%22C8 Bowtie
\newunicodechar{⋈}{\bowtie}
%22EA Not Normal Subgroup Of
\newunicodechar{⋪}{\ntriangleleft}
%22EB Does Not Contain As Normal Subgroup
\newunicodechar{⋫}{\ntriangleright}
%22EC Not Normal Subgroup of or Equal To
\newunicodechar{⋬}{\ntrianglelefteq}
%22ED Does Not Contain as Normal Subgroup or Equal
\newunicodechar{⋭}{\ntrianglerighteq}
%25A1 White Square
\newunicodechar{□}{\Box}
%27E6 Mathematical Left White Square Bracket – there’s also \llbracket from stmaryrd
\newunicodechar{⟦}{\text{\textlbrackdbl}}
%27E7 Mathematical Right White Square Bracket – there’s also \rrbracket from stmaryrd
\newunicodechar{⟧}{\text{\textrbrackdbl}}
%27FC Long Rightwards Arrow from Bar
\newunicodechar{⟼}{\longmapsto}
%2AB0 Succeeds Above Single-Line Equals Sign
\newunicodechar{⪰}{\succeq}
%301A Left White Square Bracket
\newunicodechar{〚}{\textlbrackdbl}
%301B Right White Square Bracket
\newunicodechar{〛}{\textrbrackdbl}
%→ is defined by default as \textrightarrow, which is invalid in math mode. Same thing for the three other commands. I redefine those four using \DeclareUnicodeCharacter instead of \newunicodechar because the latter warns about the previous definition.
%→ Rightwards Arrow
\DeclareUnicodeCharacter{2192}{\ifmmode\rightarrow\else\textrightarrow\fi}
%¬ Not Sign
\DeclareUnicodeCharacter{00AC}{\ifmmode\lnot\else\textlnot\fi}
%… Horizontal Ellipsis
\DeclareUnicodeCharacter{2026}{\ifmmode\dots\else\textellipsis\fi}
%× Multiplication Sign
\DeclareUnicodeCharacter{00D7}{\ifmmode\times\else\texttimes\fi}


\NewDocumentCommand{\R}{}{ℝ}
\NewDocumentCommand{\N}{}{ℕ}
%\mathscr is rounder than \mathcal.
\NewDocumentCommand{\powerset}{m}{\mathscr{P}(#1)}
%Powerset without zero.
\NewDocumentCommand{\powersetz}{m}{\mathscr{P}^*(#1)}
%https://tex.stackexchange.com/a/45732, works within both \set and \set*, same spacing than \mid (https://tex.stackexchange.com/a/52905).
\NewDocumentCommand{\suchthat}{}{\;\ifnum\currentgrouptype=16 \middle\fi|\;}
%Integer interval.
\NewDocumentCommand{\intvl}{m}{⟦#1⟧}
%Allows for \abs and \abs*, which resizes the delimiters.
\DeclarePairedDelimiter\abs{\lvert}{\rvert}
\DeclarePairedDelimiter\card{\lvert}{\rvert}
%Perhaps should use U+2016 ‖ DOUBLE VERTICAL LINE here?
\DeclarePairedDelimiter\norm{\lVert}{\rVert}
%Better than using the package braket because braket introduces possibly undesirable space. Then: \begin{equation}\set*{x \in \R^2 \suchthat \norm{x}<5}\end{equation}.
\DeclarePairedDelimiter\set{\{}{\}}
\DeclarePairedDelimiter\ceil{\lceil}{\rceil}
\DeclarePairedDelimiter\floor{\lfloor}{\rfloor}
\DeclareMathOperator*{\argmax}{arg\,max}
\DeclareMathOperator*{\argmin}{arg\,min}

%We want the straight form of \phi for mathematics, as recommended in UTR #25: Unicode support for mathematics, and thus use \phi for the mathematical symbol and not \varphi; and similarly \epsilon is preferred to \varepsilon for the mathematical symbol.

%The amssymb solution.
%\NewDocumentCommand{\restr}{mm}{{#1}_{\restriction #2}}
%Another acceptable solution.
%\NewDocumentCommand{\restr}{mm}{{#1|}_{#2}}
%https://tex.stackexchange.com/a/278631; drawback being that sometimes the text collides with the line below.
\NewDocumentCommand\restr{mm}{#1\raisebox{-.5ex}{$|$}_{#2}}


%Decision Theory (MCDA and SC)
\NewDocumentCommand{\allalts}{}{A}
\NewDocumentCommand{\allcrits}{}{\mathscr{C}}
\NewDocumentCommand{\alts}{}{A}
\NewDocumentCommand{\dm}{}{i}
\NewDocumentCommand{\allF}{}{\mathscr{F}}
\NewDocumentCommand{\allvoters}{}{\mathscr{N}}
\NewDocumentCommand{\voters}{}{N}
\NewDocumentCommand{\allprofs}{}{\boldsymbol{\mathcal{R}}}
\NewDocumentCommand{\prof}{}{P}
\NewDocumentCommand{\ibar}{}{\overline{i}}
\NewDocumentCommand{\lprof}{}{\lambda_P}
\NewDocumentCommand{\lprofi}{O{x}}{\lambda_P(#1)_i}
\NewDocumentCommand{\lprofibar}{O{x}}{\lambda_P(#1)_{\overline{i}}}
\NewDocumentCommand{\ineq}{}{(\sigma \circ \lambda_P)}

\NewDocumentCommand{\linors}{}{\mathcal{L}(\allalts)}
%Thanks to https://tex.stackexchange.com/q/154549
	%\makeatletter
	%\def\@myRgood@#1#2{\mathrel{R^X_{#2}}}
	%\def\myRgood{\@ifnextchar_{\@myRgood@}{\mathrel{R^X}}}
	%\makeatother
\NewDocumentCommand{\pref}{}{\succ}
\NewDocumentCommand{\prefi}{O{i}}{\succ_{#1}}
\NewDocumentCommand{\paretopt}{}{\text{PO}}
\NewDocumentCommand{\SPPd}{}{\Sigma^\text{PPd}}
\NewDocumentCommand{\SAll}{}{\Sigma^\text{All}}
\NewDocumentCommand{\SThreshold}{}{\Sigma_\text{threshold}}
\NewDocumentCommand{\vpr}{}{\boldsymbol{v}}

\NewDocumentCommand{\musigma}{O{\sigma}O{P}}{\argmin_{A}({#1}\circ\lambda_{{#2}})}
\NewDocumentCommand{\mustar}{O{\sigma}O{P}}{\argmin_{\paretopt({#2})} ({#1} \circ \lambda_{#2})}
\NewDocumentCommand{\minineq}{O{\allalts}}{\argmin_{#1}(\sigma \circ \lambda)}
\NewDocumentCommand{\FBP}{}{\text{FB}(P)}
\NewDocumentCommand{\POP}{}{\text{PO}(P)}

\NewDocumentCommand{\alllosses}{}{\intvl{0, m-1}^N}

\NewDocumentCommand{\Ptop}{}{\bar{P}}
\NewDocumentCommand{\sigmatop}{}{\bar{\sigma}}

\NewDocumentCommand{\fltwo}{}{\floor{\bar{l_2}}}
\NewDocumentCommand{\bltwo}{}{\bar{l_2}}

\newtheorem{conjecture}{Conjecture}


%I find these settings useful in draft mode. Should be removed for final versions.
	%Which line breaks are chosen: accept worse lines, therefore reducing risk of overfull lines. Default = 200.
		\tolerance=2000
	%Accept overfull hbox up to...
		\hfuzz=2cm
	%Reduces verbosity about the bad line breaks.
		\hbadness 5000
	%Reduces verbosity about the underful vboxes.
		\vbadness=1300

\title{Ex-Ante versus Ex-Post Compromise}
\author{Submission ID}
\hypersetup{
	pdfsubject={Social choice},
	pdfkeywords={axiomatic analysis},
}

\begin{document}
\maketitle

\begin{abstract}
	A classical social choice setting is composed of a group of individuals, or voters, that express their preferences over a set of alternatives. The social choice problem consists in defining a procedure able to determine a collective choice for this group of voters, starting from their individual preferences. Such procedure is called social choice rule and it can be defined as a function mapping preference profiles to alternatives. Depending on the properties that this function satisfies, very different outcomes can be produced starting from the same initial profile. The plurality rule is one of the most common social choice rule and it consists in selecting, as a winner, the alternative that is considered the best by the largest number of voters forming the society. Yet, this rule can pick, as a winner, an alternative that is considered the worst by a strict majority of voters. Such outcome may be undesirable. Several procedures, the so-called compromise rules, have been proposed in the literature that aim to find a compromise. Nevertheless, all those rules can be defined as \emph{ex-ante compromises} or \emph{procedural compromises}, i.e., they impose over individuals a willingness to compromise but they do not ensure an outcome where everyone has effectively compromised. In this work, we approach the problem of compromise from an \emph{ex-post} perspective, favoring an outcome where every voter gives up her most preferred positions if this increases equality. We propose a new notion of compromise in the social choice context, considering ordinal utilities.
\end{abstract}

\section{Introduction}
\label{sec:introduction}

In a classical social choice scenario, several individuals express their preferences over a set of alternatives and there is no unique procedure for selecting a common agreement between them. Nevertheless, there is an accepted understanding that collective choices must reflect compromises. One of the first to explicitly refer to a social choice rule (SCR) as a compromise is \citet{Sertel1986} introducing the \textit{majoritarian compromise}. This SCR, further analyzed by \citet{Sertel1999}, is a rediscovery of a method suggested by James W. Bucklin at the beginning of the 20th century (for more details see \citet{Erdelyi2015}). It falls back from considering everyone’s ideal alternative to considering the voters’ second, third and more generally k-\emph{th} best until one of the alternatives is among the first k best for a majority. \citet{Brams2001} generalize this concept and introduce a class of SCRs called $q-$\textit{approval fall-back bargaining }where $q$ is the level of required support which can vary from a single voter up to unanimity. Naturally, different choices of $q$ lead to different SCRs, such as $q=1$ giving the plurality rule; $q$ being majority giving the majoritarian compromise and $q$ being unanimity giving a bargaining procedure called \textit{fall-back bargaining} which has been further analyzed by \citet{Kibris2007} and \citet{Congar2012}. 

As \citet{OezkalSanver2004} discuss, the concept of compromising is mostly understood as the trade off between the number of voters supporting an alternative (i.e., the quantity of support) and the distance of that alternative from the supporters' ideal alternative (i.e., the quality of support). This trade off, which is explicit for $q-$approval fall-back bargaining, is also the basis for several other SCRs such as the \textit{median voting rule} proposed by \citet{Bassett1999} and further analyzed by \citet{Gehrlein2003} or the \textit{Condorcet practical method }described by \citet{Nurmi1999}.

\citet{Merlin2019} identify and analyze a large class of \textit{compromise rules} which are based on balancing the trade off between the quality and the quantity of support. On the other hand, as conflicting
individual preferences over the available alternatives makes impossible to ensure the best outcome for every member of the group, one can argue that making a collective choice \textit{per se} implies compromising. In that sense, every SCR incorporates some understanding of what a compromise means. To be sure, there are instances where this understanding may contradict common sense, such as dictatorships where one voter always ensures his best outcome whatever the others prefer. Nevertheless, interesting SCRs base the collective choice on the principle that all voters may have to fall back from their ideal position. Whether, at the end of the day, all voters do effectively fall back or not is another issue which is the subject matter of this paper.

Sometimes, indeed, they do not. This observation was the basis for an objection made by Jean-François Laslier to the nomenclature on compromising.\footnote{This happened at a CNRS workshop on compromising hosted by Istanbul Bilgi University at Buyukada, Istanbul on fall 2018.} Consider the following example.
\begin{example}
	\label{ex:ex1}
	Let $N$ be a set of $n ≥ 3$ voters and $A$ a set of alternatives. $\linors$ represents the set of linear orders over $A$. Consider the following preference profile $P\in \linors^{N}$:
	\begin{center}
		$
		\begin{array}{cccc}
		\mathbf{1} \quad &c&b&a\\
		\mathbf{n-1} \quad &a&b&c\\
		\end{array},
		$
	\end{center}
	which represents one individual who prefers $c$ to $b$, $b$ to $a$, hence $c$ to $a$; and $n-1$ individuals who prefer $a$ to $b$, $b$ to $c$, hence $a$ to $c$. At $P$, all BK-compromises, except fall-back bargaining i.e. when $q=n$, will ignore the single voter and will pick $a$ as the collective outcome.
\end{example}

As a matter of fact, almost every interesting SCR will ignore this “marginal minority” and choose $a$ in this situation. While this choice is defensible on the grounds of qualified majoritarianism, the presence of $b$, which receives unanimous support when each voter falls back one step from his ideal point, renders questionable whether $a$ can be qualified as a compromise. The question becomes even more acute for majoritarian SCRs, including the majoritarian compromise, where $a$ would
remain the collective choice even when the ignored group is much larger.

\begin{example}
	\label{ex:ex2}
	Consider the following preference profile with $n=100$:
	\begin{center}
		$
		\begin{array}{cccc}
		\mathbf{49} \quad &c&b&a\\
		\mathbf{51} \quad &a&b&c\\
		\end{array}.
		$
	\end{center}
	When $q\in \intvl{1,\frac{n}{2}+1} $, all BK-compromises pick $a$, and, again, it does not appear as a compromise as 51 voters reach their best alternative while the remaining 49 voters have to be contented with their worst one. Note that for $q\in \intvl{1,\frac{n}{2}-1} $ the set of possible common agreements determined by the fall-back bargaining procedure is $\{a,c\}$. Nevertheless, $a$ receives the highest support, thus it is elected.
\end{example}

It is important to observe that all these SCRs impose to voters a willingness to compromise which does not mean that under the collective choice that will be made, all voters will be effectively compromising. In other words, the term “compromise” in this literature refers to procedural or \textit{ex-ante} compromises, which is different than outcome oriented or \textit{ex-post} compromises, a conceptual distinction that seems to be overlooked in the literature.

To define an ex-post compromise, we adopt to our framework a concept of equal losses that is prevalent in the literature that considers the allocation of continuous utilities. This principle is used for bargaining problems \citep{Chun1988}, \citep{Chun1991} as well as for bankruptcy problems \citep{Herrero2001}. 
We introduce two definitions for being a compromise. In both of them, we pick a spread measure that determines how equally a given vector of real numbers is distributed and make a collective choice where voters give up from their ideal points “as equal as possible". \footnote{A discussion of the literature on spread measures is given in \cref{sec:RestrictionOnSigma}.} However, one of them, called \textit{egalitarian compromise}, insists on equality at the expense of Pareto efficiency while the other, called \textit{Paretian compromise}, is constrained to pick among the Pareto efficient alternatives. 
The two concepts are logically incompatible. As a result, Pareto efficient SCRs cannot ensure egalitarian compromises and this is valid under any spread measure. Moreover, several well-knowm SCRs of the literature such as Condorcet extensions, scoring rules, $q-$approval fall-back bargaining, all fail to be Paretian compromises under any spread measure. In fact, we are able to observe the existence of instances where being a Paretian compromise necessitates to pick an alternative that is, although Pareto optimal, ranked so low by all voters that this alternative wouldn't be picked by any of the popular SCRs of the literature. All these observations make the equal-loss principle appear quite inadequate for collective choice problems, unless envy-freeness is a major concern.

Collective choice models with two individuals present instances where envy-freeness matters. Here, the model is interpreted as a bargaining problem and bargaining procedures replace voting rules. As prominent examples, we have fallback bargaining proposed by \citet{Brams2001}; the unanimity compromise and the rational compromise introduced by \citet{Kibris2007}; the veto-rank and short listing procedures analyzed by \citet{Clippel2014} and the Pareto-and-veto rules analyzed by \citet{Laslier2020}. As is ours, these are all models with discrete alternatives which are not contained by the classical \citet{Nash1950} bargaining environment with convex utilities. However, as \citet{Mariotti1998} and \citet{Nagahisa2002} illustrate, the two worlds can be interconnected, as we do for the equal-loss principle of \citet{Chun1988} and \citet{Chun1991}.

examples of envy-freeness 2-person
\begin{example}
	\begin{center}
		$
		\begin{array}{cccccccccc}
		\mathbf{i_1} \quad &x&a_1&\dots&&a_k&y&b_1&\dots&b_k\\
		\mathbf{i_2} \quad &b_1&\dots&b_{k-1}&y&x&b_k&a_1&\dots&a_k\\
		\end{array}
		$
	\end{center}
\end{example}

computational?

\section{Basic notions and notation}
\label{sec:notation}
Consider a finite set $N$ of individuals with $\#N=n\geq 2$ and a finite set $A$ of alternatives with $\#A=m\geq 3$. We write $\linors$ for the set of linear orders over $A$.
A generic element $\prefi$ of $\linors$ stands for a preference of $i\in N$.%
\footnote{So given any $x ≠ y\in A$, precisely one of $x \prefi y$ and $y\prefi x$ holds while $x \prefi x$ holds for no $x\in A.$ Moreover, $x\prefi y$ and $y\prefi z$ implies $x\prefi z$ $\forall x,y,z\in A$.}
A \emph{profile} $P: N → \linors$ associates with each individual $i \in N$ a preference order  $P(i) = {\prefi}$. A \emph{social choice rule} (SCR) is a mapping $f:\linors^{N}\rightarrow 2^{A} \setminus \{\emptyset \}$. 

We write $r_{\prefi}(x)=\#\{y\in A \suchthat y \prefi x\}+1$ for the \emph{rank} of $x\in A$ at ${\prefi} \in \linors$. We denote by $\lambda_{\prefi}(x)=r_{\prefi}(x)-1$ the loss in terms of ranks for $i\in N$ with preference $\prefi$, when $x$ is elected instead of the best alternative
for $i$. The mapping $\lambda_P: A → \alllosses$ assigns to each $x\in A$ the loss vector $\lambda_{P}(x)=(\lambda_{\prefi}(x))_{i\in N}$ induced by the election of $x$.\footnote{We use double brackets to denote intervals in the integers.}

We are interested in measuring the spread of loss vectors. To this end, we adopt a \emph{spread measure} $\sigma: \alllosses → \R_{+}$ that associates a spread value to every possible loss
vector. We write $\Sigma$ for the set of spread measures $\sigma$ that satisfy for every $l\in\alllosses$, $\sigma(l)=0\iff l_{i}=l_{j}$ $\forall i,j\in N$. Thus, the spread of $l$ gets its lowest value $0$ in case of perfect equality and only in this case. 

Given any distinct $x,y\in A$, we say that $x$ \emph{Pareto dominates} $y$ at $P \in\linors^{N}$ (or equivalenty $y$ is \emph{Pareto dominated} by $x $ at $P$) iff $x\prefi y,\forall i\in N$. We denote
$\paretopt(P)= \set{x \in A \suchthat \forall y ≠ x \in A, \exists i \in N \suchthat x \pref_i y}$ the set of \emph{Pareto optimal} alternatives at $P$.
A SCR $f$ is \emph{Paretian} iff $f(P)\subseteq\paretopt(P)$ $\forall P\in\linors^{N}$.

\section{Egalitarian versus Paretian compromises}
\subsection{Egalitarian compromises}
\label{sec:EgCompromise}
We denote the minimal elements of $X\in2^{A}\setminus \{\emptyset\}$ according to $(\sigma\circ\lambda_{P})$ with $\min_{\sigma \circ \lambda_P} (X) = \set{x \in X \suchthat \forall y \in X: \sigma(\lambda_P(x)) ≤ \sigma(\lambda_P(y))}$. Thus, $\min_{\sigma\circ\lambda_{P}}(X)$ denotes the alternatives in X whose loss vectors are the most equally distributed according to the spread measure $\sigma$.

In what follows, we define some classes of SCRs that we are interested in analyzing. 


\begin{definition} A SCR $f$ is an \emph{Egalitarian Compromise} (EC) iff \[\exists \sigma \in \Sigma \suchthat \forall P \in \linors^N \text{ we have }f(P) \subseteq \musigma.\]
\end{definition}

\begin{definition} A SCR $f$ is \emph{Egalitarian Compromise Compatible} (ECC) iff \[\exists \sigma \in \Sigma \suchthat \forall P \in \linors^N \text{ we have } f(P) \cap \musigma \neq \emptyset.\]
\end{definition}

Under a SCR that is EC (resp., ECC), \emph{all} (resp., \emph{some}) winners are among the alternatives with most equally distributed losses. Clearly, EC is a subclass of ECC. Perhaps less obviously, being ECC (or EC) is incompatible with being Paretian. This will be deduced from the following proposition, which will also be useful to prove other theorems.% \cref{th:incompatibility}.


\begin{proposition} \label{prop:muSigmaLast}
	For $n ≥ 2, m ≥ 3$, there exists a profile $P \in \linors^N$ and an alternative $a_m$ such that $\forall i \in N$: $r_{\prefi}(a_m)=m$, and such that $\forall \sigma \in \Sigma: \musigma = \set{a_m}$; hence, $\musigma \cap \paretopt(P) = \emptyset$.
\end{proposition}
\begin{proof}
	Consider the following profile $P$:
	\begin{center}
		$
		\begin{array}{cccccc}
		\mathbf{1} \quad &a_1&a_2&\dots&a_{m-1}&a_m\\
		\mathbf{n-1} \quad &a_{\pi_(1)}&a_{\pi_(2)}&\dots&a_{\pi_(m-1)}&a_m\\
		\end{array}
		$,
	\end{center}
	where $\pi$ is the following permutation over $\intvl{1, m-1}$:
	\[
	\pi(i) = 
	\begin{cases}
	i+1 & \text{if } i \in \intvl{1, m-2} \\
	1 & \text{if } i = m-1
	\end{cases}.
	\]
	In $P$, $a_m$ is the only alternative such that $r_{\prefi}(a_m)=m$, $\forall i \in N$; hence, $\sigma(\lambda_P(a) > 0$, $\forall a \in A\setminus \{a_m\}$, $\forall \sigma \in \Sigma$. Thus, the set $\musigma$ consists of the sole element $a_m$, and, because $a_m$ is Pareto dominated, $\musigma \cap \paretopt(P) = \emptyset$.
\end{proof}

Our main result for \cref{sec:EgCompromise} follows easily.
\begin{theorem} \label{th:nonParetian}
	For $n\geq 2, \ m\geq3$, no Paretian SCR is ECC.
\end{theorem}
\begin{proof}
	Proving this amounts to show that $\forall \sigma \in \Sigma, \exists P \in \linors^N \suchthat \paretopt(P) \cap \musigma = \emptyset$. Suffices to use \cref{prop:muSigmaLast}, which asserts that there exists a profile $P$ such that $\forall \sigma \in \Sigma: \musigma \cap \paretopt(P) = \emptyset$.
\end{proof}

\subsection{Paretian compromises}
Having seen the tension for a SCR being Paretian and ECC, we investigate the consequences of inverting the order of priorities by insisting that at least some of the winning alternatives are Pareto optimal, and considering the most equally distributed loss vectors among those.

We consider two classes of SCRs. 
Observe that $\mustar$ denotes the set of Pareto optimal alternatives whose loss vectors are the most equally distributed according to the spread measure $\sigma$.

\begin{definition} A SCR $f$ is a \emph{Paretian Compromise} PC iff \[\exists \sigma \in \Sigma \suchthat \forall P \in \linors^N \text{ we have } f(P) \subseteq \mustar.\]
\end{definition}

\begin{definition} A SCR $f$ is \emph{Paretian Compromise Compatible} PCC iff \[\exists \sigma \in \Sigma \suchthat \forall P \in \linors^N \text{ we have } f(P) \cap \mustar \neq \emptyset.\]
\end{definition}

Again, it is clear that PC is a subclass of PCC. It will also probably come with no surprise that for a SCR, being PC is incompatible with being ECC, as being PC requires to be Paretian, which permits to use \cref{th:nonParetian}. On the other hand, it is less immediate that being EC is incompatible with
being PCC, because being PCC does not require to be Paretian. This is however true.

\begin{theorem} \label{th:incompatibility} 
	For $n ≥ 2, m ≥ 3$, no SCR is both EC and PCC.
\end{theorem}
\begin{proof}	
	Letting $\Ptop$ denote the profile of \cref{prop:muSigmaLast}, with $a_m$ the alternative mentioned there, and considering any EC $f$ and any $\sigma \in \Sigma$, suffices to prove that $f(\Ptop) \cap {\mustar[\sigma][\Ptop]} = \emptyset$.
	
	First, from \cref{prop:muSigmaLast}, $\set{a_m} \cap \paretopt(\Ptop) = \emptyset$, hence $\set{a_m} \cap {\mustar[\sigma][\Ptop]} = \emptyset$. 
	
	Second, because $f$ is an EC, for some $\sigmatop$, $f(\Ptop) \subseteq {\musigma[\sigmatop][\Ptop]}$. Using \cref{prop:muSigmaLast} again, we see that ${\musigma[\sigmatop][\Ptop]} = \set{a_m}$, hence $f(\Ptop) = \set{a_m}$.
	
	That $f(\Ptop) \cap {\mustar[\sigma][\Ptop]} = \emptyset$ follow from these two facts.
\end{proof}

It is interesting to note that the incompatibility is not complete, however.

\begin{remark}
	For $n ≥ 2$, $m ≥ 3$, there exist SCRs that are both ECC and PCC, such as the SCR that selects the whole set of alternatives at every profile. However, this SCR fails to be Paretian, as it must be for every SCR that is ECC.
\end{remark}


\section{Which SCRs are compromises?}
\label{sec:more2voters}
In this section we assume $n\geq 3$ and leave the analysis of $n=2$ to the
next section.

\subsection{Condorcet consistent rules}

An alternative $x\in A$ is a \textit{Condorcet winner} at $P\in L(A)^{N}$ iff for all $y\in A \setminus \set{x} $, $\#\set{i \in N \suchthat x \prefi y} >\#\set{i \in N \suchthat y \prefi x}$. So each profile admits
either no or a unique Condorcet winner. An SCR $f$ is \textit{Condorcet
consistent} iff $f(P)=$ $\left\{ x\right\} $ at each $P\in L(A)^{N}$ that
admits $x$ as the unique Condorcet winner.

\begin{theorem} \label{th:condorcet}
Let $n\geq 3$ and $m\geq 3$. A Condorcet consistent SCR $f$ is neither ECC nor PCC.
\end{theorem}
\begin{proof}
Consider the following profile $P$, where the dots represent the sequence $a_4$ to $a_m$:
	\begin{center}
		$
		\begin{array}{cccccc}
		\mathbf{n-1} \quad &a_1&a_2&a_3&\dots\\
		\mathbf{1} \quad &a_3&a_2&\dots&a_1\\
		\end{array}
		$.
	\end{center}

Consider any Condorcet consistent SCR $f$, then $f(P)=\{a_1\}$. However, $\musigma=\mustar=\{a_2\}$ $\forall \sigma \in \Sigma$, so there exists a profile $P$ such that both $f(P)\cap \musigma$ and $f(P)\cap \mustar$ are empty.
\end{proof}

Note that Condorcet consistent rules need not be Paretian so the fact that they all fail ECC does not follow from \cref{th:nonParetian}. 

\subsection{Scoring rules}
A \emph{score vector} is an $m-$tuple $w=(w_{1},\dots,$ $w_{m})\in \intvl{0, 1}^{m}$ with $w_{1}=1$, $w_{m}=0$ and $w_{i}\geq w_{i+1}$ $\forall
i\in \intvl{1, m-1}$. Given a score vector $w$, we write $s^{w}(x,P)=\sum_{i\in N}w_{r_{\prefi}(x)}$ for the score of $x\in A$ at $P\in L(A)^{N}$. Every score vector $w$ identifies a \emph{scoring rule} $f^w_n$ defined as $f^w_n(P)=\left\{ x\in A:s^{w}(x,P)\geq s^{w}(y,P) \ \forall y\in A\right\}$ for every $P\in L(A)^{N}$.

We first show that no scoring rule is ECC, for any value of $n$ and $m$ at least 3.

\begin{theorem}\label{th:srECC}
Let $n\geq 3$ and $m\geq 3.$ No score vector $w$ induces a scoring rule $f^w_n$ that is ECC.
\end{theorem}
\begin{proof}
Take any score vector $w$. Consider the profile $P$ of \cref{prop:muSigmaLast}. Observe that $\musigma=\{a_m\} \ \forall \sigma \in \Sigma $. However, as $w_{1}>w_{m}$, we have $s^{w}(a_{1},P)>s^{w}(a_{m},P)$ which implies $a_{m}\notin f^{w}(P)$.
\end{proof}

We call antiplurality score vector the score vector $w$ formed such that $w_{i} = 1, \forall i \in \intvl{1, m-1}$ and $w_{m}=0$.

\begin{theorem}
	\label{th:AntSatsPCC}
	Let $m\geq 3$ and let $w$ be the antiplurality score vector. The SCR $f_{n}^{w}$ satisfies PCC for all $n\geq 3$.
\end{theorem}
\begin{proof}
	Define $\bar\sigma \in \Sigma$ as, $\forall l \in \intvl{0,m-1}^N$: $\bar\sigma(l) = 1$ iff $\exists i, j \in N \suchthat l_i ≠ l_j$; $\bar\sigma(l) = 0$ otherwise.
	We show the non-emptyness of $f^w_n(P) \cap \mustar[\bar\sigma]$ for any profile $P$.

	Let $k = \min_{x \in \paretopt(P)} \set{(\bar\sigma \circ \lambda_P)(x)}$ be the minimal value attained by $\bar\sigma \circ \lambda_P$ over $\paretopt(P)$.
	By construction of $\bar\sigma$, $k$ equals either $0$ or $1$.
	
	For $k = 1$, take any $x \in f^w_n(P) \cap \paretopt(P)$, which exists because the antiplurality rule, although not Paretian, never picks only non-Pareto optimal alternatives. 
	By definition of $\bar\sigma$, $\bar\sigma(x) ≤ 1$, hence, $x \in \mustar[\bar\sigma]$.
%	We then have that $x \in \mustar[\bar\sigma]$ as by definition of $\bar\sigma$, $\bar\sigma(x) ≤ 1$.
	
	If $k = 0$, take any $x \in \mustar[\bar\sigma]$. As $\bar\sigma (\lambda _{P}(x))=0$, we have, $\forall i, j \in N$: $\lambda_i^P(x) = \lambda_j^P(x)$, hence, $\forall i, j \in N$: $r_{\succ_i}(x) = r_{\succ_j}(x)$. 
	The case $r_{\succ_i}(x) = m, \forall i \in N$ is ruled out by $x \in \paretopt(P)$. Hence, $r_{\succ_i}(x) ≤ m - 1, \forall i \in N$, hence, $x \in f^w_n(P)$.
\end{proof}

It is worth noting that the antiplurality rule $f_{n}^{w}$ is not Paretian, hence fails PC  for all $n\geq 3$. This, can be seen by picking a unanimous profile $P\in \linors^{N}$ with $a_{1}\prefi a_{2}\prefi \dots \prefi a_{m}$ $\forall i\in N$, where $\mustar=\left\{ a_{1}\right\} \forall \sigma \in \Sigma $ while $f_{n}^{w}(P)=A \setminus \left\{ a_{m}\right\}$.

\begin{theorem}
	\label{th:srPCC}
	Let $m\geq 3.$ Take any score vector $w$ which is not the antiplurality score vector. The SCR $f_{n}^{w}$ fails PCC for some $n\geq 3$.
\end{theorem}

\begin{proof}
	Take any $m\geq 3$ and a score vector $w$ such that it is not the antiplurality score vector. Therefore, $w_{m-1}<1$. Observe the existence of two natural numbers $n_{1}$, $n_{2}\geq 3$ with $n_{1}\geq m-1$ and $\frac{n_{2}-1}{n_{2}}>w_{m-1}$.
	Let $n=\max \left\{ n_{1,}n_{2}\right\} $ and let $A=\left\{ a_{1}, \dots, a_{m}\right\} $. Take some $P\in L(A)^{N}$ with
	
	\begin{center}
		$
		\begin{array}{cccccc}
		i = 1 \quad & a_2 & … & a_m & a_1\\
		2 ≤ i ≤ m - 2 \quad & a_1 & … & a_m & a_i\\
		m - 1 ≤ i ≤ n \quad & a_1 & … & a_m & a_{m-1}\\
		\end{array}
		$,
	\end{center}
	where all alternatives except $a_m$ appear at least once in the last rank.
	Thus, for every $\sigma \in \Sigma$, we have 
	$\sigma (\lambda _{P}(x))>0$ $\forall x\in A \setminus \left\{ a_{m}\right\}$
	while
	$\sigma (\lambda_{P}(a_{m}))=0$. 
	Moreover, $a_{m}\in \paretopt(P)$. Thus, $\mustar=\left\{ a_{m}\right\} $ $\forall \sigma \in \Sigma $. On the
	other hand, $s^{w}(a_{1}; P)=n-1$, $s^{w}(a_{m}; P)=n\cdot w_{m-1}$ and
	as $\frac{n-1}{n}>w_{m-1}$, we have $s^{w}(a_{1}; P)>s^{w}(a_{m};$ $P)$,
	establishing $a_{m}\notin f^{w}(P)$, thus $f^{w}(P)\cap \mustar=\emptyset $ $\forall \sigma \in \Sigma $.
\end{proof}

\subsection{BK-compromises}
\label{sec:BKn3}
Given any $k\in \intvl{1, m}$, we write $n_{k}(x,P)=\#\{i\in
N\mid r_{\prefi}(x)\leq k\}$ for the \emph{$k$-support} that $x$ gets at $P$, that is, the number of individuals for whom the rank of alternative $x\in A$ is lower than or equal to $k$ in the profile $P\in $ $L(A)^{N}$.
Note that $n_{k}(x,P)\in \intvl{1, n}$ is non-decreasing on $k$ and $n_{m}(x,P)=n.$ For each $q\in \intvl{1,n}$, we define $\rho_{q}(x,P)=\min \{k\in \intvl{1,m} \suchthat n_{k}(x,P)\geq q\}$ as the minimal rank $k$ at which the $k$-support that $x$ gets at $P$ is at least $q$. We
write $\rho _{q}(P) = \min_{x \in A} \set{\rho_{q}(x, P)}$ for the minimal rank $k$ at which the $k$-support that some alternative gets at $P$ is at least $q$. \textit{A Brams and Kilgour (BK) compromise with threshold }$q$ is the
SCR $f_{q}$ defined for each $P\in \linors^N$ as $f_{q}(P)=\{x\in A | n_{\rho _{q}(P)}(x,P)\geq n_{\rho _{q}(P)}(y,P)$ $\forall y\in A\}.$

\begin{theorem}
	\label{th:FBsatsPC}
Let $n\geq 3$ and $m\geq 3.$ The BK compromise $f_{n}$ satisfies PC.
\end{theorem}

\begin{proof}
Define $\bar{\sigma } \in \Sigma$ as, $\forall l \in \intvl{0,m-1}^N$: $\bar\sigma(l) = 1$ iff $\exists i, j \in N \suchthat l_i ≠ l_j$; $\bar\sigma(l) = 0$ otherwise.
Considering any $x \in f_n(P)$, let us show that $x \in \mustar[\bar{\sigma}]$. Because $x \in f_n(P)$, $x \in \paretopt(P)$, and therefore, suffices to show that $\forall y \in \paretopt(P)$, $\bar{\sigma}(\lambda_P(y)) ≥ \bar{\sigma}(\lambda_P(x))$. Given the choice of $\bar{\sigma}$, picking any $y \in \paretopt(P)$ with $y≠x$, suffices to show that $\bar{\sigma}(\lambda_P(y)) = 1$, equivalently, that $\exists i, j \in N \suchthat r_{\prefi}(y) ≠ r_{\pref_j}(y)$. 
Because $x \in f_n(P)$, $\rho_n(P) = \rho_n(x, P) = \max_{i \in N} r_{\prefi}(x)$.
It follows from $\rho_n(P) = \min_{z \in A} \set{\rho_n(z, P)}$ that $\rho_n(y, P) ≥ \rho_n(x, P)$, thus, $\exists i \in N \suchthat r_{\prefi}(y) ≥ \rho_n(P)$. 
Also, $y \in \paretopt(P)$ implies that $\exists j \in N \suchthat r_{\pref_j}(y) < r_{\pref_j}(x)$, thus $\exists j \in N \suchthat r_{\pref_j}(y) < \rho_n(P)$. 
Therefore, $r_{\prefi}(y) ≠ r_{\pref_j}(y)$.
\end{proof}

\begin{theorem}
	\label{th:FBfailsECC}
	Let $n\geq 3$ and $m\geq 3.$ The BK compromise $f_{n}$ fails ECC. 
\end{theorem}
\begin{proof}
	As $f_{n}$ is Paretian, the proof comes straightforward from \cref{th:nonParetian}.
\end{proof}

\begin{theorem}
	\label{th:BKthreshold}
	Let $n\geq 3$ and $m\geq 3.$ A BK compromise $f_{q}$ with threshold $q \in \intvl{1, n-1}$ is neither ECC nor PCC.
\end{theorem}
\begin{proof}
	%Take any $n\geq 3$ and $m\geq 3.$ Let $A=\left\{ a_{1},\text{ }a_{2,}...
	%\text{ }a_{m}\right\} $. Pick some $q\in \left\{ 1,...,n\right\} $ and
	%consider the BK compromise $f_{q}$. 
	Consider the following profile $P$, where the dots represent the sequence $a_4$ to $a_m$ (also used in the proof of \cref{th:condorcet}):
	\begin{center}
		$
		\begin{array}{cccccc}
		\mathbf{n-1} \quad &a_1&a_2&a_3&\dots\\
		\mathbf{1} \quad &a_3&a_2&\dots&a_1\\
		\end{array}
		$.
	\end{center}
	We have that $f_{q}(P)=\{a_1\}$, and, because $\sigma(\lambda_P(a_2)) = 0$ and $\sigma(\lambda_P(a_1)) > 0$, neither $\musigma(P)$ nor $\mustar(P)$ contain $a_1$ for any $\sigma \in \Sigma$. 
	%Remzi's proof
	%Take any $n\geq 3$ and $m\geq 3.$ Let $A=\left\{ a_{1},\text{ }a_{2,}...%
	%\text{ }a_{m}\right\} $. Pick some $q\in \left\{ 1,...,n\right\} $ and
	%consider the BK compromise $f_{q}$. Consider the profile $P\in L(A)^{N}$ such that 
	%$a_{1}\succ _{i}a_{2}\succ _{i}...\succ _{i}a_{m}$ $\forall i\in N\diagdown
	%\left\{ n\right\} $ and $a_{\pi (1)}\succ _{n}a_{\pi (2)}\succ _{n}...\succ
	%_{n}a_{\pi (m)}$ where $\pi $ is a bijection on $\left\{ 1,\text{ }2,...,%
	%\text{ }m\right\} $ with $\pi (1)=3$, $\pi (2)=2,\pi (3)=1$, $\pi (i)=i+1$ $%
	%\forall i\in \left\{ 4,...,\text{ }m-1\right\} $ and $\pi (m)=4 $, we have $%
	%f_{q}(P)=\left\{ a_{1}\right\} $ while $\mu _{\sigma }(P)=\mu _{\sigma
	%}^{\ast }(P)=\left\{ a_{2}\right\} $ $\forall \sigma \in \Sigma $.
\end{proof}

\subsection{Restrictions on sigma}
\label{sec:RestrictionOnSigma}
The perfect equality recognition condition we adopt for spread measures, i.e., that the spread gets its lowest value $0$ in case of perfect equality and only in this case, is very basic. Unless this condition is violated, $\Sigma$ is the largest set of spread measures we could conceive. On the other hand, it is possible to let $\Sigma$ shrink by imposing additional conditions over spread measures. Nevertheless, as the satisfaction of PC, PCC, EC, or ECC requires the existence of a spread measure, all of our negative results, namely, those expressed by Theorems \ref{th:nonParetian}, \ref{th:incompatibility}, \ref{th:condorcet}, \ref{th:srECC}, \ref{th:srPCC}, \ref{th:FBfailsECC} and \ref{th:BKthreshold} prevail when $\Sigma$ is restricted. In a similar vein, the positive results in Theorems \ref{th:AntSatsPCC} and \ref{th:FBsatsPC} risk to be lost with additional conditions over spread measures.

\begin{definition}
	\label{def:conditionC}
	Given any $m\geq3$ and $n\geq \max\{3,m-1\}$, we say that a spread measure $\sigma$ satisfies condition $C_{m,n}$ iff we have $\sigma(m-3, m-1, m-2, \dots, m-2) < \sigma(m-2, m-3, \dots, 1, 0, \dots, 0)$.
\end{definition}

As both vectors are $n$ dimensional, the term $m-2$ repeats $n-2$ times in the first vector and the term $0$ repeats $n-m+2$ times in the second vector.

The condition is more convincing for larger values of $m$ and $n$. In fact, asking for $\sigma(0,2,1)$ to be smaller than $\sigma(1,0,0)$ is very demanding while asking for $\sigma(3,5,4,4,4,4,4)$ to be smaller than $\sigma(6,5,4,3,2,1,0)$ reflects a mild assumption. In any case, as we state below, several well-known spread measures of the literature (see xxx for a comprehensive account) satisfy \cref{def:conditionC} for reasonably small values of $m$ and $n$. Letting $\bar{l}=\frac{\sum_{i=1}^{n}l_i}{n}$ denote the arithmetic mean of the values of $l = (l_1, …, l_n)$, we consider the following measures:

\begin{itemize}
	\item the Gini coefficient $\sigma_{G}(l)= \frac{\sum_{i=1}^{n}\sum_{j=1}^{n}|l_i-l_j|}{2 \cdot n \cdot \sum_{i=1}^{n} l_i}$;
	\item the standard deviation $\sigma_{sd}(l)= \sqrt{\frac{\sum_{i=1}^{n}(l_i-\bar{l})^2}{n}}$;
	\item the mean absolute difference $\sigma_{mad}(l)= \frac{1}{n^2} \sum_{i=1}^{n}\sum_{j=1}^{n}|l_i-l_j|$;
	\item the average absolute deviation $\sigma_{ad}(l)= \frac{\sum_{i=1}^{n}|l_i-\bar{l}|}{n}$.
\end{itemize} 

\begin{proposition}
\label{prop:spreadMeas}
	$\sigma_{mad}$, $\sigma_{ad}$, $\sigma_{sd}$ and $\sigma_{G}$ all satisfy condition C $\forall m\geq4$ and $n\geq\max\{4,m-1\}$.
\end{proposition}

The proof of \cref{prop:spreadMeas} is given in \cref{apdx:proofSM}.

%A \emph{spread measure} $\sigma: \alllosses → \R_{+}$ satisfies condition gamma iff  $\sigma (m-3,$ $m-1,m-2,...,$ $m-2)$ <$\sigma(m-2,$ $m-1,...1,$0, $\ 0)$.
%			(\lambda_{P}(y))$ that associates a spread value to every possible loss
%vector. We write On the other hand, $\lambda
%			^{P}(x)=(m-3,$ $m-1,m-2,...,$ $m-2)$ and $\lambda_{P}(y)=(m-2,$ $m-1,,...1,$
%			$0,$ $\ 0)$.

We write $\Sigma^{C_{m,n}} \subseteq \Sigma$ for the set of spread measures that satisfy condition $C_{m,n}$. 
\begin{theorem}
	For all $m\geq 3$, $n\geq \max\{3,m-1\}$, under $\Sigma^{C_{m,n}}$,
	\begin{itemize}
	    \item [1)] $f_n^{w}$ fails PCC when $w$ is the antiplurality score vector;
	    \item [2)] the BK compromise  $f_n$ fails PCC.
	\end{itemize}
\end{theorem}

	\begin{proof}
		Take any $m\geq 3$ and any $n \geq \max\{3,m-1\}$ and consider $\Sigma^{C_{m,n}} \subseteq \Sigma$, the set of spread measures that satisfy condition $C_{m,n}$. Take some $x,y\in A$ and some $P\in \linors^{N}$ with $r_{\prefi[1]}(x)=m-2$, $r_{\prefi[2]}(x)=m,$ $r_{\prefi}(x)=m-1$ $\forall i\in N \setminus \left\{ 1, 2\right\}$, and $r_{\prefi}(y)=m-i$ $\forall i\in \intvl{1,m-1}$, $r_{\prefi[n]}(y)=1$ $\forall i\in \intvl{m,n}$. Moreover, for each $z\in A \setminus \left\{ x,y\right\} $, we have $r_{\prefi[1]}(z)=m$ for some $i\in N$. Note that both $f_n^{w}$ and $f_{n}$ pick only $y$ at $P$. On the other hand, $\lambda^{P}(x)=(m-3, m-1,m-2,\dots,m-2)$ and $\lambda_{P}(y)=(m-2, m-3,\dots,1,0, \dots, 0)$. As all the spread measure $\sigma \in \Sigma^{C_{m,n}}$ satisfy the condition $C_{m,n}$, then $\sigma(\lambda_{P}(x)) < \sigma(\lambda_{P}(y))$ $\forall \sigma \in \Sigma^{C_{m,n}}$, implying $y\notin \mustar[\bar{\sigma}]$ $\forall \sigma \in \Sigma^{C_{m,n}}$.
	\end{proof}


 
\section{Two voters case}
In \cref{sec:more2voters} we focused on the analysis of voting rules when the number of voters involved into the decision process is greater than two. Keeping the notation introduced in \cref{sec:notation}, we consider here the case $n=2$. Two individuals express their preference over a set of alternatives $A$, and the goal is to find a common agreement on the alternative to select. This class of problems is often referred to as bargaining problems. In addition to \textit{fallback bargaining (FB)} \citep{Brams2001} (defined in \cref{sec:BKn3}) we consider three prominent solutions of the literature:

\textit{Veto Ranking (VR)} is commonly used in the selection of arbitrators \citep[see][]{Clippel2014}. Given a list of $m$ (odd) alternatives (that are candidates to be arbitrators), each of the two voters (that are the two parties that must agree on an arbitrator) simultaneously vetoes his worst $\frac{m-1}{2}$ alternatives. The selected alternatives are the ones with the highest Borda score among the non-vetoed alternatives.

Again within the context of selecting arbitrators, \citet{Clippel2014} propose and analyze \textit{Shortlisting (SL)} where one of the two parties starts by vetoing her worst $\frac{m-1}{2}$ alternatives ($m$ being odd), and then the second party chooses her best alternative out of the remaining ones. As the outcome of the procedure depends on the party that starts, symmetry among players is ensured by defining the solution as the union of the two outcomes where one and the other party starts.

\textit{Pareto-and-Veto (PV)} procedure \citep{Laslier2020} distributes a power to veto $v_1$ and $v_2$ alternatives to voters 1 and 2, respectively, with $v_1+v_2=m-1$. Every voter $v_i$ simultaneously vetoes his worst $v_i$ alternatives. The solution picks all non-vetoed and Pareto optimal alternatives.

\begin{definition}
    Given any $m \geq 4$, a spread measure $\sigma$ satisfies condition $D_m$ iff 
    $\sigma(\ceil{\frac{m}{2}}, \floor{\frac{m}{2}} - 1) < \sigma(0, \floor{\frac{m}{2}})$ and 
    $\sigma(\floor{\frac{m}{2}} - 1, \ceil{\frac{m}{2}}) < \sigma(\floor{\frac{m}{2}}, 0)$.
\end{definition}
\commentBN{I would change $k$, it can be confusing when we define $m=2k+1$ later. Also, I believe the condition should be defined for a specific k: $D_{m,k}$ iff we have $\sigma(k+1,k-1)<\sigma(0,k)$ for $k\in \intvl{1,m-1}$. See next comment for details.}

We write $\Sigma^{D_{m}} \subseteq \Sigma$ for the set of spread measures that satisfy condition $D_{m}$. 

\begin{theorem}
	Let $m \geq 5$. Under $\Sigma^{D_{m}}$, FB and PV fail PCC. Furthermore, when $m$ is odd, VR and SL also fail PCC. 
\end{theorem}
\commentBN{I'm not sure the current definition of the condition D is enough to prove our point. In the profile we use as proof, we define $m=2k+1$. Therefore, for $k=1$, $m=3$, and the condition D is not defined. If it was, we would have had the profile
	\begin{center}
		$
		\begin{array}{cccc}
			\mathbf{i_1} \quad &x&a_1&y\\
			\mathbf{i_2} \quad &y&x&a_1\\
		\end{array}
		$
	\end{center}
where $\sigma(\lambda_{P}(y)) = \sigma(2,0)$ and $\sigma(\lambda_{P}(x)) = \sigma(0,1)$. 
\\ For $k=2$, then $m=5$ and the resulting profile is:
\begin{center}
	$
	\begin{array}{cccccc}
		\mathbf{i_1} \quad &x&a_1&a_2&y&b_1\\
		\mathbf{i_2} \quad &b_1&y&x&a_1&a_2\\
	\end{array}
	$
\end{center}
In this case $\sigma(\lambda_{P}(y)) = \sigma(3,1)$ and $\sigma(\lambda_{P}(x)) = \sigma(0,2)$. Let's consider $\sigma_{sd}$, it satisfies condition $D_5$, indeed there exist a value $j<5$ such that $\sigma_{sd}(j+1,j-1)<\sigma_{sd}(0,j)$. Take for example $j=4$, then $1=\sigma_{sd}(5,3)<\sigma_{sd}(0,4)=2$. However, if we consider our example profile we have $1=\sigma_{sd}(3,1) \nless \sigma_{sd}(0,2)=1$.
}
\begin{proof}
    Let $\sigma$ be any spread measure in $\Sigma^{D_m}$. Define $\alpha = \ceil{\frac{m}{2}} - 1$ and $\beta = \floor{\frac{m}{2}} - 1$. Note that $\sigma(\alpha + 1, \beta) < \sigma(0, \beta + 1)$ and $\sigma(\beta, \alpha + 1) < \sigma(\beta + 1, 0)$, and $\alpha + \beta + 2 = m$.
    
	Consider the following profile P:
		\begin{center}
			$
			\begin{array}{cccccccccc}
				\mathbf{i_1} \quad &x&a_1&\dots&&a_\alpha&y&b_1&\dots&b_\beta\\
				\mathbf{i_2} \quad &b_1&\dots&b_\beta&y&x&a_1&\dots&&a_\alpha\\
			\end{array}
			$
		\end{center}
	Note that $\sigma(\lambda_{P}(y)) = \sigma(\alpha + 1, \beta)$ and that $\sigma(\lambda_{P}(x)) = \sigma(0, \beta + 1)$. 
	Therefore, $\sigma(\lambda_{P}(y)) < \sigma(\lambda_{P}(x))$. Because $y$ is not Pareto-dominated, any PCC rule will not pick $x$ at $P$.
	Now consider the profile $P'$, the same as $P$ when inverting the preference of $i_1$ and $i_2$. 
	By a similar argument, we see that any PCC rule will not pick $x$ at $P'$.
	
	The proof will be concluded by showing that FB, PV, and (when $m$ is odd) VR and SL all pick only $x$ at $P$ or at $P'$.
	
	We readily see that FB will pick only $x$ at $P$ (and at $P'$) since it is the first alternative which reaches the unanimous consent.
	
	About the PV rule, if $v1 ≥ v2$ (thus $v1 ≥ \frac{m-1}{2}$), consider the profile $P$. Observe that the first voter vetoes at least $y$ and all $b_j$ alternatives ($1 ≤ j ≤ \beta$), and that no voter veto $x$. Because $x$ Pareto-dominates all $a_j$ alternatives ($1 ≤ j ≤ \alpha$), PV picks only $x$ at $P$. If $v2 ≥ v1$, a similar reasoning yields that PV picks only $x$ at $P'$. 
	
	Assume $m$ is odd.
	
	The VR rule will pick only $x$ at $P$ by a reasoning similar to the one applied to PV: the last $\frac{m-1}{2}$ alternatives get vetoed and, among the $\frac{m-1}{2}$ ones remaining, $x$ has the highest Borda score, at it is the only non-dominated alternative of that set, so it would be selected as the sole winner.
	
	Finally, the SL rule would also pick $x$, as it is chosen no matter which voter starts the veto phase.
\end{proof}
 

\commentOC{Justification for the rule: ultimatum game. People are willing to give up even more than our rule.}

We define an ex-post compromise as an outcome where individuals give up equally from their ideal points. With three or more individuals, several interesting SCRs of the literature (that can be interpreted as voting rules), namely Condorcet extensions, scoring rules (except antiplurality) and BK-compromises (except fallback bargaining where threshold is unanimity) fail to pick ex-post compromises, under any reasonable meaning attributed to “giving up equally”. This failure is valid whether Pareto optimality is adopted or not while the positive results for antiplurality and fallback bargaining quickly vanish when a mild condition is imposed over the definition of `giving up equally`. Thus, we can conclude that all these SCRs are not compatible with being ex-post compromises. Even BK-compromises are procedural or ex-ante compromises. They impose a willingness to compromise over individuals but they may eventually pick an outcome so that some individuals do not effectively compromise at all.
These findings are rather expected, as voting rules are typically unconcerned about envy-freeness among voters while this is the underlying notion of an ex-post compromise. On the other hand, even with two individuals where the collective choice problem is interpreted as a bargaining problem that requires mutual consent, all well-known SCRs of the literature, namely fallback bargaining, Pareto and veto rules, short listing, veto rank, all fail to pick ex-post compromises. 


\newpage

\bibliography{biblio}

\newpage
\appendix

\section{Spread Measures (proposition)}
\newcommand{\smad}{\sigma_\text{mad}}
\begin{proof}[for $\smad$]
	Recall that $\smad(l) = 1/n^2 \sum_{ij} |l_i - l_j|$.
	Define $f_l(i) = \sum_j |l_i - l_j|$.
	Define $s(l) = n^2 \smad(l)$. Thus: $n^2 \smad(l) = s(l) = \sum_i f_l(i)$.
	Recall that $l_1 = (m-3, m-1, m-2, …)$, containing $n-2$ terms $m-2$, and $l2 = (m-2, m - 3, …, 1, 0, …)$, containing $m-2$ terms descending from $m-2$ to $1$, then $n - m + 2$ zeros.
	The thesis is now that $s(l_1) < s(l_2)$.
	
	To compute $s(l_1)$, observe that $f_{l_1}(1) = 2 + (n - 2) = n$, $f_{l_1}(2) = 2 + (n - 2) = n$, and for $3 ≤ i ≤ n$, $f_{l_1}(i) = 1 + 1 = 2$. Thus, $s(l_1) = 2n + (n - 2) (2) = 4n - 4$.
	
	To compute $s(l_2)$, let us distinguish the cases $1 ≤ i ≤ m - 2$ and $m - 1 ≤ i$.
	
	For $1 ≤ i ≤ m - 2$, 
	\begin{align}
		f_{l_2}(i) &= [(i - 1) + (i - 2) + … + 0] + [1 + … + (m - 2 - i)] \\
		&\quad + (m - 1 - i) (n - m + 2)\\
	%	[that’s i terms, then m - 2 - i terms, then n - m + 2 terms]
		&= \frac{(i - 1) i}{2} + \frac{(m - 2 - i) (m - 1 - i)}{2} + (m - 1 - i) (n - m + 2)\\
		&= i^2 / 2 - i / 2 + \frac{(m - 2) (m - 1) - i (m - 2 + m - 1) + i^2}{2} \\
		&\quad + (m - 1) (n - m + 2) - i (n - m + 2)\\
		&= i^2 - i \frac{1 + 2m - 3 + 2(n - m + 2)}{2} + (m - 1) \frac{m - 2 + 2 (n - m + 2)}{2}\\
		&= i^2 - i (n + 1) + (m - 1) \frac{-m + 2n + 2}{2}.
	\end{align}
	
	For $m - 1 ≤ i$, $f_{l_2}(i) = m - 2 + m - 3 + … + 1 = (m - 2) (m - 1) / 2$.
	
	Thus, 
	\begin{align}
		s(l_2) &= \sum_{1 ≤ i ≤ m - 2}[i^2 - i (n + 1) + (m - 1) \frac{-m + 2n + 2}{2}] \\
		&\quad + (n - (m - 2)) (m - 2) (m - 1) / 2\\
		&= (m - 2) (m - 1) (2m - 3) / 6 - (n + 1) (m - 2) (m - 1) / 2 \\
		&\quad + (m - 2) (m - 1) (- m + 2n + 2) / 2 \\
		&\quad + (n - m + 2) (m - 2) (m - 1) / 2\\
%		&\quad + n (m - 2) (m - 1) / 2 - (m - 2)^2 (m - 1) / 2\\
		&= (m - 2) (m - 1) \left(\frac{2m - 3}{6} - \frac{n + 1}{2} + \frac{- m + 2n + 2}{2} + \frac{n - m + 2}{2}\right)\\
		&= (m - 2) (m - 1) \left(\frac{2m - 3}{6} + \frac{- 2m + 3 + 2n}{2}\right)\\
		&= (m - 2) (m - 1) \left(\frac{-2m + 3}{3} + n\right).
	\end{align}
		
	Our thesis is now that $4n - 4 < (m - 2) (m - 1) [(-2m + 3) / 3 + n]$, or equivalently, that $(2m - 3) (m - 2) (m - 1) / 3 - 4 < n [(m - 2) (m - 1) - 4]$.
	
	When $m = 4$, using the fact that $4 ≤ n$, the inequality holds: $5 (2) (3) / 3 - 4 < 4 [2] ≤ n [2]$.
	
	Now assume that $m ≥ 5$. Using the fact that $m - 1 ≤ n$, suffices to show that
	$(2m - 3) (m - 2) (m - 1) / 3 - 4 < (m - 1) [(m - 2) (m - 1) - 4]$, or equivalently, that
	$-12 < (m - 1) [3 (m - 2) (m - 1) - 12 - (2m - 3) (m - 2)]$.
	Note that the right hand side equals $(m - 1) [(m - 2) (3 (m - 1) - (2m - 3)) - 12] = (m - 1) (m^2 - 2m - 12) = (m - 1) (m - 1 + \sqrt{13}) (m - 1 - \sqrt{13})$. As all multiplicands are positive when $m ≥ 5$, the inequality is true when $m ≥ 5$.
\end{proof}

\section{Spread Measures}
\label{apdx:proofSM}
In what follows are showed the proofs for the Theorems presented in \cref{sec:RestrictionOnSigma}.
\begin{proof} for $\sigma_{mad}$. \\
	Consider the two vectors $l_1=(m-3, m-1, m-2, \dots, m-2)$ and $l_2=\sigma(m-2, m-3, \dots, 1, 0, \dots, 0)$; their spread using the Mean Absolute Difference will be: 
	\begin{equation}
		\begin{split}
			\sigma_{mad}(l_1) &=\frac{1}{n^2} \cdot [2(2+(n-2))+(n-2)\cdot2)]= \frac{1}{n^2} \cdot [4(n-1)] \\ \\
			\sigma_{mad}(l_2)&=\frac{1}{n^2} \cdot \{\sum_{j=0}^{m-3} \ [\sum_{i=0}^{m-3-j}i \ + \ (m-2-j)(n-(m-2)) \ + \ \sum_{i=0}^{j}i] + (n-(m-2))\cdot \sum_{j=0}^{m-2} j \}= \\ &= \frac{1}{n^2} \cdot [\frac{1}{3}\cdot(m-2)(m-1)(3n-2m+3)]
		\end{split}
	\end{equation}
	
	In order for $\sigma_{mad}$ to satisfy condition C, must be $\sigma_{mad}(l_1) < \sigma_{mad}(l_2)$, so
	\[\frac{1}{n^2} \cdot 4(n-1)<\frac{1}{n^2} \cdot \frac{1}{3}\cdot(m-2)(m-1)(3n-2m+3).\]
	Since $\frac{1}{n^2}$ is a positive term equal on both sides, we limit our study to
	\[4(n-1)<\frac{1}{3}\cdot(m-2)(m-1)(3n-2m+3)\]
	which is true for all $m> \frac{1}{2}(3+\sqrt{17})\approx 3.56$ and $n>\frac{2 \cdot m^3 - 9\cdot m^2 + 13\cdot m - 18}{3 (m^2 - 3 \cdot m - 2)}$. Moreover, $m-1>\frac{2 \cdot m^3 - 9\cdot m^2 + 13\cdot m - 18}{3 (m^2 - 3 \cdot m - 2)}$ $\ \forall m>4$, therefore we can conclude that $\sigma_{mad}(l_1)<\sigma_{mad}(l_2)$ $\forall m\geq4$ and $n\geq\max\{4,m-1\}$.
	\commentBN{The inequality works also for other values of m and n lower than $3$. Do we need to mention this?}
\end{proof}

\begin{proof} for $\sigma_{ad}$. \\
	Consider the two vectors $l_1=(m-3, m-1, m-2, \dots, m-2)$ and $l_2=(m-2, m-3, \dots, 1, 0, \dots, 0)$. For $l_1$, the arithmetic mean of its values is $m-2$ and the spread using the Average Absolute Deviation is: 
	\[\sigma_{ad}(l_1)=\frac{1}{n}[|m-3-m+2|+|m-1-m+2|+ 0 + \dots + 0]= \frac{2}{n}\]
	While for $l_2$, the arithmetic mean is 
	\[\bar{l_2}=m-2+m-3+\dots+m-(m-1)+0+\dots+0= \frac{1}{n}\sum_{i=2}^{m-1}{m-i}= \frac{m^2-3m+2}{2n}\]
	\[\sigma_{ad}(l_2)=\frac{1}{n}[\sum_{i=2}^{m-1}{|(m-i)-\frac{m^2-3m+2}{2n}|+(n-(m-2))|\frac{m^2-3m+2}{2n}|}|]\]
	I'm stuck with the computation because even if I use the triangular inequality to remove the absolute value, it gives me an upper bound which is not useful. 
	\\\newline We can see that $\sigma_{ad}(3,1,2,2)<\sigma_{ad}(2,1,0,0)$ for $m=n=4$. We can also observe that the value of $\sigma_{ad}(l_1)$ depends only on $n$. So when increasing $m$ $\sigma_{ad}(l_1)$ does not change and $\sigma_{ad}(l_2)$ increases. When increasing $n$, both of the values decrease so the inequality is still satisfied. \commentBN{To be proved formally. I believe it's true because either we increment both m and n, then even if $\sigma_{ad}(l_2)$ decreases, it's bigger than $\sigma_{ad}(l_1)$ because of the increment of m; either we keep m constant but both of them decrease, and $\sigma_{ad}(l_1)$ decreases much faster. Maybe it can be proved by induction over two variables.}
\end{proof}

\begin{proof} for $\sigma_{sd}$. \\
	Consider the two vectors $l_1=(m-3, m-1, m-2, \dots, m-2)$ and $l_2=(m-2, m-3, \dots, 1, 0, \dots, 0)$. For $l_1$, the arithmetic mean of its values is $m-2$ and the spread using the Standard Deviation is: 
	\[\sigma_{sd}(l_1)=\sqrt{\frac{(m-3-m+2)^2+(m-1-m+2)^2+ 0 + \dots + 0}{n}}= \sqrt{\frac{2}{n}}\]
	While for $l_2$, the arithmetic mean is 
	\[\bar{l_2}=m-2+m-3+\dots+m-(m-1)+0+\dots+0= \frac{1}{n}\sum_{i=2}^{m-1}{m-i}= \frac{m^2-3m+2}{2n}\]
	and the spread using the Standard Deviation is: 
	\begin{align}
		\sigma_{sd}(l_2)=&\sqrt{\frac{\sum_{i=2}^{m-1}{[(m-i)-\frac{m^2-3m+2}{2n}]^2+(n-(m-2))[0-\frac{m^2-3m+2}{2n}]^2}}{n}}= \\
		& =\textstyle{\frac{1}{2 \sqrt{3}} \sqrt{\frac{(m - 2) (m - 1) (2 (2 m - 3) n^2 + 3 (m - 4) (m - 1) (m - 2) n - 3 (m - 3) (m - 1) (m - 2)^2}{n^3}}}
	\end{align}
	Therefore $\sigma_{sd}(l_1)<\sigma_{sd}(l_2)$ is true $\forall m \geq 4 n\geq 4$.
\end{proof}

\begin{proof} for $\sigma_{G}$. \\
	Consider the two vectors $l_1=(m-3, m-1, m-2, \dots, m-2)$ and $l_2=\sigma(m-2, m-3, \dots, 1, 0, \dots, 0)$; their spread using the Gini Coefficient is: 
	\begin{equation}
		\begin{split}
			\sigma_{G}(l_1) &=\frac{4(n-1)}{n^2\cdot 2 \cdot (m-2)}= \frac{2(n-2)}{(m-2)n^2} \\ \\
			\sigma_{G}(l_2)&=\frac{\frac{1}{3}\cdot(m-2)(m-1)(3n-2m+3)}{n^2\cdot 2 \cdot \frac{m^2-3m+2}{2n}}=\frac{3n-2m+3}{3\cdot n}
		\end{split}
	\end{equation}
	\commentBN{I'm not convinced of this. The \href{https://bit.ly/3hSNvSM}{results} exclude some values for which is true instead.}
\end{proof}


\begin{definition}[OLD Definition: Pairwise Pareto dominance]
	\label{def:PPD}
	For all $r$, $s\in \R_{+}^{N}$: 
	\[\left[\left\vert r_{i}-r_{j}\right\vert \leq \left\vert s_{i}-s_{j}\right\vert \forall i, j\in N\right] ⇒ \sigma (r)\leq \sigma (s).\] 
\end{definition}
We write $\SPPd \subseteq \SAll$ for the class of spread measures that satisfy also PPd.
Given a vector $r \in \R^N$ of $n$ elements, some examples of spread measures are the following.
\commentOC{We dropped this because it says that $(0, 2)$ is more equal than $(10^6, 10^6 + 3)$.}


\end{document}
