\newtheorem{theorem}{Theorem}[section]
\newtheorem{acknowledgement}[theorem]{Acknowledgement}
\newtheorem{algorithm}[theorem]{Algorithm}
\newtheorem{axiom}{Axiom}[section]
\newtheorem{case}{Case}[section]
\newtheorem{claim}{Claim}[section]
\newtheorem{conclusion}{Conclusion}[section]
\newtheorem{condition}{Condition}[section]
\newtheorem{conjecture}{Conjecture}[section]
\newtheorem{corollary}{Corollary}[section]
\newtheorem{criterion}{Criterion}[section]
\newtheorem{definition}{Definition}[section]
\newtheorem{example}{Example}[section]
\newtheorem{exercise}{Exercise}[section]
\newtheorem{lemma}{Lemma}[section]
\newtheorem{notation}{Notation}[section]
\newtheorem{problem}{Problem}[section]
\newtheorem{proposition}{Proposition}[section]
\newtheorem{remark}{Remark}[section]
\newtheorem{solution}{Solution}[section]
\newtheorem{summary}{Summary}[section]
\newtheorem*{note}{Note}
%\newenvironment{proof}[1][Proof]{\noindent\textbf{#1.} }{\ \rule{0.5em}{0.5em}}

\newunicodechar{⇒}{\Rightarrow}
\newunicodechar{≠}{\ensuremath{\neq}}
\newunicodechar{≤}{\leq}
\newunicodechar{≥}{\geq}
%… Horizontal Ellipsis
\DeclareUnicodeCharacter{2026}{\ifmmode\dots\else\textellipsis\fi}
\newunicodechar{∧}{\land}
\newunicodechar{∨}{\lor}
\newunicodechar{∩}{\cap}
\newunicodechar{∪}{\cup}
%¬ Not Sign
\DeclareUnicodeCharacter{00AC}{\ifmmode\lnot\else\textlnot\fi}
\newunicodechar{⇔}{\Leftrightarrow}
\newcommand{\N}{ℕ}
\newunicodechar{ℕ}{\mathbb{N}}
\newcommand{\R}{ℝ}
\newunicodechar{ℝ}{\mathbb{R}}

\newcommand{\pref}{\succ}%real, connected pref, strict
\newcommand{\prefeq}{\succeq}%real, connected pref, strict
\newcommand{\prefr}{{\succ}^\text{r}}%real, connected pref, strict
\newcommand{\pprefeq}{\succeq^\text{p}}%partial pref
\newcommand{\ppref}{\succ^\text{p}}%partial pref
\newcommand{\pprefinv}{\prec^\text{p}}%partial pref
\newcommand{\nppref}{\nsucc^\text{p}}%negated partial pref
\newcommand{\linors}{\mathcal{L}(A)}
\newcommand{\suchthat}{\;\ifnum\currentgrouptype=16 \middle\fi|\;}

\newcommand{\oquot}{``}
\newcommand{\cquot}{''}

\DeclarePairedDelimiter\set{\{}{\}}
\DeclareMathOperator*{\argmax}{arg\,max}
\DeclareMathOperator*{\argmin}{arg\,min}
