
\documentclass{article}
%%%%%%%%%%%%%%%%%%%%%%%%%%%%%%%%%%%%%%%%%%%%%%%%%%%%%%%%%%%%%%%%%%%%%%%%%%%%%%%%%%%%%%%%%%%%%%%%%%%%%%%%%%%%%%%%%%%%%%%%%%%%%%%%%%%%%%%%%%%%%%%%%%%%%%%%%%%%%%%%%%%%%%%%%%%%%%%%%%%%%%%%%%%%%%%%%%%%%%%%%%%%%%%%%%%%%%%%%%%%%%%%%%%%%%%%%%%%%%%%%%%%%%%%%%%%
\usepackage{amssymb}

%TCIDATA{OutputFilter=LATEX.DLL}
%TCIDATA{Version=5.50.0.2960}
%TCIDATA{<META NAME="SaveForMode" CONTENT="1">}
%TCIDATA{BibliographyScheme=Manual}
%TCIDATA{Created=Saturday, March 28, 2020 13:25:21}
%TCIDATA{LastRevised=Saturday, March 28, 2020 22:19:42}
%TCIDATA{<META NAME="GraphicsSave" CONTENT="32">}
%TCIDATA{<META NAME="DocumentShell" CONTENT="Standard LaTeX\Blank - Standard LaTeX Article">}
%TCIDATA{CSTFile=40 LaTeX article.cst}

\newtheorem{theorem}{Theorem}
\newtheorem{acknowledgement}[theorem]{Acknowledgement}
\newtheorem{algorithm}[theorem]{Algorithm}
\newtheorem{axiom}[theorem]{Axiom}
\newtheorem{case}[theorem]{Case}
\newtheorem{claim}[theorem]{Claim}
\newtheorem{conclusion}[theorem]{Conclusion}
\newtheorem{condition}[theorem]{Condition}
\newtheorem{conjecture}[theorem]{Conjecture}
\newtheorem{corollary}[theorem]{Corollary}
\newtheorem{criterion}[theorem]{Criterion}
\newtheorem{definition}[theorem]{Definition}
\newtheorem{example}[theorem]{Example}
\newtheorem{exercise}[theorem]{Exercise}
\newtheorem{lemma}[theorem]{Lemma}
\newtheorem{notation}[theorem]{Notation}
\newtheorem{problem}[theorem]{Problem}
\newtheorem{proposition}[theorem]{Proposition}
\newtheorem{remark}[theorem]{Remark}
\newtheorem{solution}[theorem]{Solution}
\newtheorem{summary}[theorem]{Summary}
\newenvironment{proof}[1][Proof]{\noindent\textbf{#1.} }{\ \rule{0.5em}{0.5em}}
\begin{document}


\section{Which SCRs are compromises?}

In this section we assume $n\geq 3$ and leave the analysis of $n=2$ to the
next section.

\subsection{Condorcet consistent rules}

We say that $x\in A$ is a \textit{Condorcet winner} at $P\in L(A)^{N}$ iff $%
\#\left\{ i\in N:x\succ _{i}y\right\} >\#\left\{ i\in N:y\succ _{i}x\right\} 
$ for all $y\in A\diagdown \left\{ x\right\} $. So each profile admits
either no or a unique Condorcet winner. An SCR $f$ is \textit{Condorcet
consistent} iff $f(P)=$ $\left\{ x\right\} $ at each $P\in L(A)^{N}$ that
admits $x$ as the unique Condorcet winner.

\begin{theorem}
Let $n\geq 3$ and $m\geq 3.$ A Condorcet consistent SCR $f$ is neither ECC
nor PCC.
\end{theorem}

\begin{proof}
Take any $n\geq 3$ and $m\geq 3.$ Let $A=\left\{ a_{1},a_{2,}...a_{m}\right\} $. Consider any Condorcet consistent SCR $f$ and some $%
P\in L(A)^{N}$ such that $a_{1}\succ _{i}a_{2}\succ _{i}...\succ _{i}a_{m}$ $%
\forall i\in N\diagdown \left\{ n\right\} $ and $a_{\pi (1)}\succ _{n}a_{\pi
(2)}\succ _{n}...\succ _{n}a_{\pi (m)}$ where $\pi $ is a bijection on $%
\{ 1,\text{ }2,...,\text{ }m\} $ with $\pi (1)=3$, $\pi (2)=2,\pi
(3)=1$, $\pi (i)=i+1$ $\forall i\in \left\{ 4,...,m-1\right\} $ and $%
\pi (m)=4$. As $f$ is Condorcet consistent, we have $f(P)=\left\{
a_{1}\right\} $ while $\mu _{\sigma }(P)=\mu _{\sigma }^{\ast }(P)=\left\{
a_{2}\right\} $ $\forall \sigma \in \Sigma $.
\end{proof}

Note that Condorcet consistent rules need not be Paretian so the fact that
they all fail ECC does not follow from Proposition 3.1 \textsc{(or whatever
its new name will be). }

\subsection{Scoring rules}

\textit{A score vector }is an $m-$tuple $w=(w_{1},...,$ $w_{m})\in \left[ 0,1%
\right] ^{m}$ with $w_{1}=1$, $w_{m}=0$ and $w_{i}\geq w_{i+1}$ $\forall
i\in \left\{ 1,...,\text{ }m-1\right\} .$ A score vector $w$ is \textit{%
strict}, if, furthermore, $w_{i}>w_{i+1}$ $\forall i\in \left\{ 1,...,\text{ 
}m-1\right\} .$ Given a score vector $w$, we write $s^{w}(x;$ $%
P)=\dsum\limits_{i\in N}w_{r_{\succ _{i}}(x)}$ for the score of $x\in A$ at $%
P\in L(A)^{N}$. Every score vector $\ w$ induces a SCR called the \textit{%
scoring rule }$f$ $^{w}$ defined as $f$ $^{w}(P)=\left\{ x\in
A:s^{w}(x;P)\geq s^{w}(y;P)\text{ }\forall y\in A\right\} $ for every $P\in
L(A)^{N}$. We call $f$ $^{w}$ a \textit{strict scoring rule }whenever $w$ is
strict.

We first show that all scoring rules fail ECC for any value of $n$ and $m.$

\begin{theorem}
Let $n\geq 3$ and $m\geq 3.$ No score vector $w$ induces a scoring rule $f$ $%
^{w}$ that is ECC.
\end{theorem}

\begin{proof}
Take any $n\geq 3$ and $m\geq 3.$ Let $A=\left\{ a_{1},\text{ }a_{2,}...%
\text{ }a_{m}\right\} $. Take any score vector $w$ . If $w$ is strict then $%
f $ $^{w}$ is Paretian which, by Proposition 3.1 \textsc{(or whatever its
new name will be)}, implies that $f$ $^{w}$ is not ECC. Now, let $w$ not be
strict. Consider some $P\in L(A)^{N}$ such that $a_{1}\succ _{i}a_{2}\succ
_{i}...\succ _{i}a_{m}$ $\forall i\in N\diagdown \left\{ n\right\} $ and $%
a_{\pi (1)}\succ _{n}a_{\pi (2)}\succ _{n}...\succ _{n}a_{\pi (m)}$ where $%
\pi $ is a bijection on $\left\{ 1,\text{ }2,...,\text{ }m\right\} $ with $%
\pi (i)=i+1$ $\forall i\in \left\{ 1,...,\text{ }m-2\right\} $, $\pi (m-1)=1$
and $\pi (m)=m$. Observe that $\mu _{\sigma }(P)=\left\{ a_{m}\right\} $ $%
\forall \sigma \in \Sigma $. However, as $w_{1}>w_{m}$, we have $%
s^{w}(a_{1}; $ $P)$ $>s^{w}(a_{m};$ $P)$ which implies $a_{m}\notin $ $f$ $%
^{w}(P)$.
\end{proof}

We now analyze whether scoring rules satisfy PC or PCC. Our first result
identifies, for a given score vector $w,$ properties for $n$ and $m$ that
lead to a failure of PCC.

\begin{proposition}
\label{whenPCCfails} Let $n\geq 3$ and $m\geq 3.$ Take any score vector $w$.
If $k.n\geq m-1$ for some $k\in \left\{ 1,...,\text{ }m-2\right\} $ and $%
\frac{n-1}{n}>w_{m-k}$ then $f$ $^{w}$ fails PCC.
\end{proposition}

\begin{proof}
Take any $n\geq 3$ and $m\geq 3$ with $k.n\geq m-1$ for some $k\in \left\{
1,...,\text{ }m-2\right\} $. Let $A=\left\{ a_{1},...,\text{ }a_{m}\right\} $%
. Consider any score vector $w$ with $\frac{n-1}{n}>w_{m-k}$. Take some $%
P\in L(A)^{N}$ with $r_{\succ _{i}}(a_{1})=1$ $\forall i\in \left\{ 1,...,%
\text{ }n-1\right\} $, $r_{\succ _{n}}(a_{1})=m,$ $r_{\succ
_{i}}(a_{m-k})=m-k$ $\forall i\in N.$ Moreover, for every $x\in A\diagdown
\left\{ a_{m-k}\right\} $, $r_{\succ _{i}}(x)>m-k$ for some $i\in N$ and $%
r_{\succ _{j}}(x)\neq r_{\succ _{k}}(x)$ for some $j,k\in N$. \textsc{Such a
P exists but we should argue about this. }By construction of $P$, we have $%
\mu _{\sigma }^{\ast }(P)=\left\{ a_{m-k}\right\} $ $\forall \sigma \in
\Sigma $. Note that $s^{w}(a_{1};$ $P)=n-1$ and $s^{w}(a_{m-k};$ $P)=n\cdot
w_{m-k}$. As $\frac{n-1}{n}>w_{m-k}$, we have $s^{w}(a_{1};$ $%
P)>s^{w}(a_{m-k};$ $P)$, establishing $a_{m-k}\notin f^{w}(P)$, thus $%
f^{w}(P)\cap $ $\mu _{\sigma }^{\ast }(P)=\varnothing $ $\forall \sigma \in
\Sigma $.
\end{proof}

As an interesting particular case covered by Proposition \ref{whenPCCfails}
we have scoring vectors with $w_{2}<1$ which also includes strict scoring
rules.

\begin{proposition}
Let $n\geq 3$ and $m\geq 3.$ For any score vector $w$ with $w_{2}<1$, if $%
\frac{n-1}{n}>w_{2}$ then $f$ $^{w}$ fails PCC.
\end{proposition}

\begin{proof}
Take any $n\geq 3$ and $m\geq 3.$ Note that when $k=m-2$, we have $k.n\geq
m-1$ for any $n$ and $m.$ Thus, by Proposition \ref{whenPCCfails}, $\frac{n-1%
}{n}>w_{2}$ ensures the failure of PCC.
\end{proof}

As a matter of fact, Proposition \ref{whenPCCfails} paves the way the
observe a very large class of scoring rules that fail PCC, when $n$ is
sufficiently large.

\begin{proposition}
\label{nlargefailspcc} \bigskip Let $m$ $\geq 3.$ For any score vector $w$
with $w_{m-k}<1$ for some $k\in \left\{ 1,...,\text{ }m-2\right\} $, there
exist $n$ $\geq 3$ such that $f$ $^{w}$ fails PCC.
\end{proposition}

\begin{proof}
Take any $m$ $\geq 3$ and any $w$ with $w_{m-k}<1$ for some $k\in \left\{
1,...,\text{ }m-2\right\} $. First, observe the existence of $n_{1}\geq 3$
satisfying $k.n_{1}\geq m-1$. Next, note that $\frac{n-1}{n}>w_{m-k}$ can be
ensured with a sufficienty large $n$ if and only if $w_{m-k}<1$. So pick
some $n_{2}\geq 3$ with $\frac{n_{2}-1}{n_{2}}>w_{m-k}$ . Taking $n=\max
\left\{ n_{1},\text{ }n_{2}\right\} $, through Proposition \ref{whenPCCfails}%
, we see that $f^{w}$ fails PCC.
\end{proof}

An immediate corollary to Proposition \ref{nlargefailspcc} is about the 
\textit{antiplurality rule }which is the scoring rule $f^{w}$ with $w_{i}=1$ 
$\forall i\in \left\{ 1,...,\text{ }m-1\right\} .$

\begin{proposition}
Let $m\geq 3$. If $f$ $^{w}$ is a scoring rule which isn't the antiplurality
rule then there exist $n\geq 3$ such that $f$ $^{w}$ is not PCC.
\end{proposition}

Thus among the class of scoring rules, the antiplurality rule is the only
candidate to ensure the satisfaction of PCC which, in fact, is the case, as
we state and show below.

\begin{proposition}
Let $n\geq 3$ and $m\geq 3.$ The antiplurality rule $f$ $^{w}$ satisfies PCC
but fails PC.
\end{proposition}

\begin{proof}
Take any $n\geq 3$ and $m\geq 3.$ Let $A=\left\{ a_{1},\text{ }a_{2,}...%
\text{ }a_{m}\right\} $. The antiplurality rule $f$ $^{w}$ is not Paretian,
hence fails PC, which can bee seen by picking a unanimous profile $P\in
L(A)^{N}$ with $a_{1}\succ _{i}a_{2}\succ _{i}...\succ _{i}a_{m}$ $\forall
i\in N$ where $\mu _{\sigma }^{\ast }(P)=\left\{ a_{1}\right\} $ $\forall
\sigma \in \Sigma $ while $f$ $^{w}(P)=A\diagdown \left\{ a_{m}\right\} $.

To see that the antiplurality rule $f$ $^{w}$ satisfies PCC, pick $\overline{%
\sigma }\in \Sigma $ defined for each $(l_{1},...,$ \ $l_{n})$ $\in \left\{
0,1,...,\text{ }m-1\right\} ^{N}$ as $\overline{\sigma }(l_{1},...,$ \ $%
l_{n})=1$ if $l_{i}\neq l_{j}$ for some $i,j\in N$ and $\overline{\sigma }%
(l_{1},...,$ \ $l_{n})=0$ otherwise. Suppose, for a contradiction, $f$ $^{w}$
fails PCC. So, $\mu _{\overline{\sigma }}^{\ast }(P)\cap $ $f$ $%
^{w}(P)=\varnothing $ for some $P\in $ $L(A)^{N}$. Note that $\overline{%
\sigma }(\lambda _{P}(x))=\overline{\sigma }(\lambda _{P}(y))$ $\forall x,$ $%
y\in \mu _{\overline{\sigma }}^{\ast }(P)$ holds by definition of $\mu _{%
\overline{\sigma }}^{\ast }(P)$ and by the choice of $\overline{\sigma }$,
for $x\in \mu _{\overline{\sigma }}^{\ast }(P)$, either $\overline{\sigma }%
(\lambda _{P}(x))=1$ or $\overline{\sigma }(\lambda _{P}(x))=0$. In the
former case, $\mu _{\overline{\sigma }}^{\ast }(P)=PO(P)$, and $\mu _{%
\overline{\sigma }}^{\ast }(P)\cap $ $f$ $^{w}(P)=\varnothing $ cannot hold,
as the antiplurality rule, although not Paretian, never picks only
non-Pareto optimal alternatives. So we consider the case $\overline{\sigma }%
(\lambda _{P}(x))=0$ $\forall x\in \mu _{\overline{\sigma }}^{\ast }(P)$.
Take any $x\in \mu _{\overline{\sigma }}^{\ast }(P)$. As $\overline{\sigma }%
(\lambda _{P}(x))=0,$ we have $\lambda _{i}^{P}(x)=\lambda _{i}^{P}(x)$ $%
\forall i,j\in N$, hence $r_{\succ _{i}}(x)=r_{\succ _{j}}(x)$ $\forall
i,j\in N$. However, in case $r_{\succ _{i}}(x)\in \left\{ 1,...,\text{ }%
m-1\right\} $, we have $x\in f$ $^{w}(P)$ and in case $r_{\succ _{i}}(x)=m$,
we have $x\in f$ $^{w}(P)$, both cases giving a contradiction.
\end{proof}

Our findings on scoring rules lead to the following theorem as a corollary.

\begin{theorem}
Let $m\geq 3.$ A score vector $w$ induces a scoring rule $f$ $^{w}$ that is
PCC for any $n\geq 3$ if and only if $f$ $^{w}$ is the antiplurality rule.
\end{theorem}

\subsection{BK-compromises}

Given any $r\in \left\{ 1,\ldots ,m\right\} ,$ we write $n_{r}(x,P)=\#\{i\in
N\mid r_{\succ _{i}}(x)\leq r\}$ for the number of individuals for whom the
rank of alternative $x\in A$ is lower than or equal to $r$ in the profile $%
P\in $ $L(A)^{N}$. We call $n_{r}(x,P)$ the $r-$support that $x$ gets at $P.$
Note that $n_{r}(x,P)\in \{1,\ldots ,n\}$ is non-decreasing on $r$ and $%
n_{m}(x,P)=n.$ For each $q\in \left\{ 1,...,n\right\} $, we define $\rho
_{q}(x,P)=\min \{r\in \{1,\ldots ,m\}:n_{r}(x,P)\geq q\}$ as the minimal
rank $r$ at which the $r-$support that $x$ gets at $P$ is at least $q$. We
write $\rho _{q}(P)=\min \{\rho _{q}(x,P)\}_{x\in A}$ for the minimal rank $%
r $ at which the $r-$support that some alternative gets at $P$ is at least $q
$. \textit{A Brams and Kilgour (BK) compromise with threshold }$q$ is the
SCR $f_{q}$ defined for each $P\in L(A)^{N}$ as $f_{q}(P)=\{x\in A\mid
n_{\rho _{q}(P)}(x,P)\geq n_{\rho _{q}(P)}(y,P)$ $\forall y\in A\}.$

\begin{theorem}
Let $n\geq 3$ and $m\geq 3.$ A BK compromise $f_{q}$ with threshold $q\in
\left\{ 1,...,n-1\right\} $ is neither ECC nor PCC.
\end{theorem}

\begin{proof}
Take any $n\geq 3$ and $m\geq 3.$ Let $A=\left\{ a_{1},\text{ }a_{2,}...%
\text{ }a_{m}\right\} $. Pick some $q\in \left\{ 1,...,n\right\} $ and
consider the BK compromise $f_{q}$. At the profile $P\in L(A)^{N}$ such that 
$a_{1}\succ _{i}a_{2}\succ _{i}...\succ _{i}a_{m}$ $\forall i\in N\diagdown
\left\{ n\right\} $ and $a_{\pi (1)}\succ _{n}a_{\pi (2)}\succ _{n}...\succ
_{n}a_{\pi (m)}$ where $\pi $ is a bijection on $\left\{ 1,\text{ }2,...,%
\text{ }m\right\} $ with $\pi (1)=3$, $\pi (2)=2,\pi (3)=1$, $\pi (i)=i+1$ $%
\forall i\in \left\{ 4,...,\text{ }m-1\right\} $ and $\pi (m)=4 $, we have $%
f_{q}(P)=\left\{ a_{1}\right\} $ while $\mu _{\sigma }(P)=\mu _{\sigma
}^{\ast }(P)=\left\{ a_{2}\right\} $ $\forall \sigma \in \Sigma $.
\end{proof}

\begin{theorem}
Let $n\geq 3$ and $m\geq 3.$ The BK compromise $f_{n}$ fails ECC but
satisfies PC.
\end{theorem}

\begin{proof}
Take any $n\geq 3$ and $m\geq 3.$ As $f_{n}$ is Paretian, it fails ECC by
Proposition 3.1. To see that $f_{n}$ satisfies PC, pick $\overline{\sigma }%
\in \Sigma $ defined for each $(l_{1},...,$ \ $l_{n})$ $\in \left\{ 0,1,...,%
\text{ }m-1\right\} ^{N}$ as $\overline{\sigma }(l_{1},...,$ \ $l_{n})=1$ if 
$l_{i}\neq l_{j}$ for some $i,j\in N$ and $\overline{\sigma }(l_{1},...,$ \ $%
l_{n})=0$ otherwise. Suppose, for a contradiction, $f_{n}$ fails PC. So,
there exist some $P\in $ $L(A)^{N}$ for which $f_{n}(P)\diagdown \mu _{%
\overline{\sigma }}^{\ast }(P)\neq \varnothing $ . Pick some $x\in
f_{n}(P)\diagdown \mu _{\overline{\sigma }}^{\ast }(P)$. As $x\in f_{n}(P)$, 
$x\in PO(P)$ as well. Thus, $x\notin \mu _{\overline{\sigma }}^{\ast }(P)$
because there exists $y\in PO(P)\diagdown \left\{ x\right\} $ with  $%
\overline{\sigma }(\lambda _{P}(y))<\overline{\sigma }(\lambda _{P}(x))$.
Given the choice of $\overline{\sigma }$, we have  $\overline{\sigma }%
(\lambda _{P}(y))=0$ and $\overline{\sigma }(\lambda _{P}(x))=1$. Thus, $%
r_{\succ _{i}}(y)=r_{\succ _{j}}(y)$ $\forall i,j\in N$. As $x\in f_{n}(P)$,
this contradicts $y\in PO(P)$.
\end{proof}

\end{document}
